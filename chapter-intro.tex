%!TEX root = main.tex
\chapter{Introducción}
\label{cap:introduccion}
Los sistemas de computación tienen un rol cada vez más importante en la sociedad actual: las comunicaciones entre personas (por ejemplo, la mensajería instantánea), las redes sociales, los sistemas de vuelo, el software incorporado en los automóviles, la comunicación entre dispositivos, el software en dispositivos médicos, etc. Es tan profunda la importancia de los sistemas informáticos en la actualidad que una falla en ellos puede llevar a fuertes consecuencias económicas y/o perdidas humanas. Por ejemplo, el 4 de Octubre del 2021 una caída del sistema WhatsApp, debido a un problema de configuración de sus servidores, dejó sin el servicio a millones de personas, esto causó una conmoción en gran parte de la sociedad y una interrupción en el normal funcionamiento de instituciones y compañías, ya que este servicio se utiliza normalmente para las comunicaciones entre personas y/o grupos. La noción de \emph{falla} puede ser definida de diferentes formas, pero una definición bastante aceptada es que una falla es un evento inesperado durante la ejecución del sistema \cite{Aviziens76}. La misma se diferencia de la noción de error o \textit{bug}, el cual es un error de implementación de los programadores. La tolerancia a fallas es el área de la computación que se dedica a estudiar aquellos mecanismos que permiten que los sistemas de software sigan ejecutándose de una forma aceptable aún bajo la ocurrencia de fallas.

La investigación en sistemas distribuidos tolerantes a fallas tiene como objetivo hacer que estos sistemas sean más confiables a la hora de manejar fallas en entornos informáticos complejos. Además, la creciente dependencia de la sociedad actual en sistemas informáticos que funcionen correctamente, condujo a una creciente demanda de sistemas confiables; particularmente en sistemas con propiedades de confiabilidad cuantificables. La necesidad de dicha cuantificación es especialmente evidente en aquellos entornos de misión crítica como pueden ser los sistemas de control de vuelo, o el software para controlar plantas de energía nuclear, para dar algunos pocos ejemplos.

El flujo usual para el diseño y verificación de sistemas tolerantes a fallas consiste en definir un modelo nominal (es decir, el programa “libre de fallas” o “ideal”) para luego considerar aquellos posibles comportamientos defectuosos que se desvían del comportamiento normal (prescrito por el modelo nominal), así como la maquinaria de tolerancia a fallas que será utilizada para tratar estos comportamientos no esperados. Este modelo extendido representa la forma en que opera el sistema ante la ocurrencia de fallas. Existen diferentes formas de extender el modelo nominal, el enfoque típico es la inyección de fallas, es decir, la introducción automática de fallas en el modelo. Una propiedad importante que debe satisfacer cualquier modelo extendido, es la preservación del comportamiento normal en ausencia de fallas. En \cite{Gartner99}, se propuso un enfoque formal alternativo para abordar el análisis de la tolerancia a fallas. Este enfoque permite un análisis completamente automatizado, y distingue adecuadamente los comportamientos defectuosos de los normales. Además, este marco teórico es adaptable a la inyección de fallas. En la presente tesis, se definen tres nociones de relaciones de simulación para caracterizar diferentes tipos de tolerancia a fallas: \emph{enmascarante}(\textit{masking}), \emph{no enmascarante} (\textit{non-masking}) y \emph{seguro ante fallos} (\textit{failsafe}), como se definió originalmente en \cite{Gartner99}.

Durante la última década, se han logrado avances significativos en la definición de métricas o distancias adecuadas para diversos tipos de modelos cuantitativos, incluidos los sistemas en tiempo real \cite{HenzingerMP05}, los modelos probabilísticos \cite{DesharnaisGJP04} y las métricas para sistemas lineales y ramificados \cite{AlfaroFS09, Henzinger13, LarsenFT11, ThraneFL10}. Algunos autores ya han señalado que estas métricas pueden ser útiles para razonar sobre la robustez de un sistema, noción relacionada con la tolerancia a fallas. En particular, en \cite{CernyHR12} se generaliza la noción tradicional de relación de simulación y se introducen tres distancias de simulación diferentes entre sistemas, a saber, corrección, cobertura y robustez. Estas se definen utilizando juegos cuantitativos con objetivos de suma descontada (\textit{discounted sum} en Inglés) y  pago promedio (\textit{mean payoff} en Inglés).


%Hasta los primeros años de la década de 1990, el trabajo en computación tolerante a fallas se centró en tecnologías y aplicaciones específicas, lo que resultó en subdisciplinas aparentemente no relacionadas con terminologías y metodologías distintas.
%Desde entonces, se ha avanzado mucho al ver el área de una manera más manera abstracta y formal. Esto ha llevado a una comprensión más clara de los problemas inherentes y esenciales en el campo y muestra lo que se puede hacer para aprovechar la complejidad de los sistemas que contrarrestan fallas. 
% En la práctica, se utilizan diversas técnicas para aumentar la confiabilidad de un algoritmo, por ejemplo: usando mecanismos de votación, roll-backs, protocolos randomizados, etcétera. Sin embargo, la mayoría de estas técnicas comúnmente se usan en una forma \textit{ad-hoc}, consecuentemente,  el análisis del grado de tolerancia a fallas que proveen tales técnicas es una tarea demandante y casi nunca es posible antes de que el software ya esté en uso, y por lo tanto las fallas se vuelvan visibles a los usuarios.

Más precisamente, en esta tesis se definen relaciones de simulación que capturan la tolerancia a fallas de un sistema, así como caracterizaciones correspondientes en términos de juegos. Luego, se definen medidas sobre estas relaciones entre sistemas, las cuales permiten cuantificar el grado de tolerancia a fallas exhibidas por un sistema. Por último se definen algoritmos que permiten aplicar estas métricas, así como herramientas de software que implementan estos algoritmos. Además, se estudia la aplicabilidad de estas herramientas a casos de estudios provenientes del área de tolerancia a fallas.

Primero, se exploran estas nociones para sistemas no probabilistas (sistemas que pueden ser descritos por estados y transiciones entre ellos) y luego se consideran sistemas probabilistas (estos se describen con procesos de decisión de Markov). Para los primeros, se utilizan juegos de dos jugadores con suma cero y objetivos booleanos para estudiar sus propiedades, mientras que para los segundos es necesario explorar propiedades sobre juegos estocásticos en los cuales el ambiente (representado como un jugador) se comporta de forma \textit{fair}, en donde la noción de \textit{fairness} tiene la semántica usual en verificación de software \cite{BaierK08}.  Esto es necesario para que los juegos que utilizamos como marco teórico estén bien definidos, y por ende, los resultados obtenidos sean correctos.

El principal objetivo del \textit{framework} definido en esta tesis es  ofrecer soporte a los ingenieros en el análisis y diseño de sistemas tolerantes a fallas. Más precisamente, se ha definido funciones de distancia de enmascaramiento computable de modo que un ingeniero pueda medir la tolerancia de enmascaramiento de una implementación tolerante a fallas dada, es decir, la cantidad de fallas que se pueden enmascarar. De este modo, el ingeniero puede medir y comparar la distancia de tolerancia a fallas enmascarante en diversas implementaciones tolerantes a fallas alternativas y seleccionar la que mejor se adapte a sus preferencias.

Todas las técnicas desarrolladas en esta tesis se han implementado en herramientas de código abierto (las cuales se encuentran en repositorios disponibles al público) y a su vez se han realizado evaluaciones experimentales para validar nuestros resultados sobre diversos casos de estudio. Los resultados experimentales se muestran en los Capítulos \ref{cap:maskingMeasure}, \ref{cap:fairAdversaries}, y \ref{cap:maskProb}.  En el Capítulo \ref{cap:tool} se encuentran más detalles sobre las herramientas desarrolladas.

\section{Motivación y Objetivos}
\label{sec:intro.objetivos}

La tolerancia a fallas es una característica importante del software crítico, y puede ser definida como la capacidad de un sistema para lidiar con eventos inesperados, que pueden ser causados por \textit{bugs} de programación, interacciones con un ambiente poco cooperativo, mal funcionamiento de hardware, etcétera. Se pueden encontrar ejemplos de sistemas tolerantes a fallas en casi cualquier parte: protocolos de comunicación, circuitos de hardware, sistemas de aviónica, criptomonedas, etcétera. Por lo tanto, el incremento en la relevancia del software crítico en la vida cotidiana ha llevado a que se renueve el interés en la verificación automática de propiedades de tolerancia a fallas. Sin embargo, una de las dificultades principales a la hora de razonar sobre estos tipos de propiedades se da en su naturaleza cuantitativa, lo cual vale incluso en sistemas no probabilistas.
Un ejemplo simple se da con la introducción de redundancia en sistemas críticos. Esta es, sin lugar a dudas, una de las técnicas más utilizadas en tolerancia a fallas.
En la práctica, se sabe que al añadir más redundancia en un sistema incrementa su fiabilidad. Medir este incremento de fiabilidad es un problema central a la hora de evaluar software tolerante a fallas. Por otro lado, no hay un método \textit{de-facto} para caracterizar formalmente propiedades tolerantes a fallas, y por ello se suelen codificar utilizando mecanismos \textit{ad-hoc} como parte del diseño general.

La Teoría de Juegos \cite{MorgensternNeuman42} ofrece una teoría matemática elegante y profunda. 
En las ultimas décadas, ha recibido gran atención de la comunidad de ciencias de la computación, ya que tiene importantes aplicaciones en la verificación y síntesis de software. 
La analogía es atractiva, la operación de un sistema bajo un ambiente no cooperativo (hardware defectuoso, agentes maliciosos, canales de comunicación poco confiables, etc.) puede ser modelada como un juego entre dos jugadores (el sistema y el ambiente), en el cual el sistema trata de alcanzar ciertos objetivos, mientras que el ambiente pretende prevenir que esto suceda. 
Esta visión es particularmente útil para \textit{síntesis de controladores}, es decir, generación automática de políticas de toma de decisiones a partir de una especificación de alto nivel. 
Por lo tanto, sintetizar un controlador consiste de computar las estrategias óptimas para un juego dado. Al mismo tiempo, la teoría de juegos permite desarrollar métricas que pueden ser útiles para razonar sobre el nivel de tolerancia a fallas de un sistema dado.
En esta tesis nos enfocamos en juegos de dos jugadores, de suma cero, por turnos y de información perfecta con recompensas (no negativas)\cite{FilarV96}. 

Considerando los problemas mencionados anteriormente, los objetivos específicos de la investigación llevada a cabo en esta tesis son los siguientes:

\paragraph{Desarrollar Medidas para Tolerancia a Fallas}
\begin{itemize}
\item Obtener una caracterización formal de \textit{tolerancia a fallas enmascarante} utilizando conceptos de teoría de juegos.
    
\item Desarrollar algoritmos que permitan cuantificar el grado de \textit{tolerancia a fallas enmascarante} exhibido por un sistema.

\item Implementar, y evaluar, el rendimiento de dichas técnicas en herramientas de software de código abierto que puedan ayudar a un ingeniero a diseñar sistemas tolerantes a fallas, y evaluar su nivel de tolerancia.
\end{itemize}

\paragraph{Investigar propiedades sobre Juegos Estocásticos}
\begin{itemize}
\item Caracterizar formalmente juegos estocásticos entre un sistema y un entorno que se comporta de forma \textit{fair}. 
    
\item Desarrollar algoritmos que permitan derivar estrategias óptimas para un sistema estocástico.

\item Implementar y evaluar el rendimiento de dichas técnicas en herramientas de software de código abierto.
\end{itemize}


\section{Contribuciones}
\label{sec:intro.contribuciones}

%Se pueden mencionar papers publicados
%El trabajo de esta tesis dio fruto a tres artículos en conferencias de primer nivel \cite{CastroDDP18b} \cite{PutrueleDCD22} \cite{?}.
%A su vez se derivan de aquí múltiples trabajos que aún están bajo revisión y que no se presentan en esta tesis.
%\myworries{add intro mencionando trabajo previo/relacionado}
%En los últimos años, hubo avances en cuanto a generalizaciones cuantitativas de la noción booleana de correctitud y cuestiones relacionadas a verificación cuantitativa \cite{BokerCHK14,CernyHR12,Henzinger10,Henzinger13}.
%Esta tesis elabora sobre estos conceptos adaptándolos y extendiéndolos a la cuantificación de masking-tolerancia a fallas.
%En \cite{DemasiCMA17}, se introduce una caracterización de masking-tolerancia a fallas en términos de relaciones de (bi)-simulación basada en estados. Aquí proveemos una caracterización basada en transiciones, la cuál se extiende de manera simple a la construcción de juegos.
A continuación se enumeran las contribuciones principales del trabajo realizado en esta tesis:

\begin{enumerate}
	
\item Definimos la relación de enmascaramiento entre dos sistemas por medio de relaciones de simulación y damos una caracterización correspondiente en términos de juegos. Consecuentemente, agregamos objetivos cuantitativos al juego para así presentar una noción de distancia de tolerancia a fallas enmascarante entre sistemas. 
Utilizando esta distancia, es posible medir el grado de tolerancia de una implementación dada y compararla con otras para así seleccionar la más adecuada. 

\item Investigamos las propiedades de los juegos estocásticos con \textit{payoff} de recompensa total bajo la suposición de un entorno \textit{fair}. Bajo este escenario, mostramos que se preserva la determinación y que los jugadores poseen estrategias óptimas sin memoria y deterministas, además, mostramos que el valor del juego puede  ser calculado en tiempo polinomial.

\item Introducimos una noción formal de \textit{tolerancia a fallas enmascarante} entre sistemas probabilistas basándonos en una variante de bisimulación probabilista. También damos la caracterización correspondiente en términos de juegos estocásticos. Desarrollamos una medida para cuantificar el nivel de tolerancia a fallas enmascarante exhibido por sistemas \textit{que fallan casi-seguramente}, es decir, sistemas que llegan a un estado de falla con probabilidad $1$.

\item Desarrollamos herramientas automáticas diseñadas para medir el nivel de tolerancia a fallas enmascarante entre componentes de software (tanto estocásticos y no estocásticos), descritos por medio de un lenguaje de comandos con guardas.

%La herramienta se enfoca en medir componentes \textit{masking-tolerantes a fallas}, es decir, programas que enmascaran fallas de tal manera que no puedan ser observadas por el ambiente\cite{PutrueleDCD22}. 
 
\end{enumerate}

\section{Organización}
\label{sec:intro.organizacion}

A continuación se realiza un vistazo general a la estructura del resto de la tesis:
\begin{itemize}
	
\item Capítulo \ref{cap:preliminares}: Se introducen conceptos preliminares necesarios a lo largo de la tesis.
\item Capítulo \ref{cap:maskingMeasure}: Se define una relación de enmascaramiento entre sistemas y presentamos la noción de distancia de enmascaramiento entre sistemas tolerantes a fallas, y luego se provee una evaluación experimental. Este capítulo está basado en el artículo \textit{Measuring Masking Fault-Tolerance}\cite{CastroDDP18b} publicado en \textit{International Conference on Tools and Algorithms for the Construction and Analysis of Systems (TACAS) 2019}.
\item Capítulo \ref{cap:fairAdversaries}: Se describen resultados sobre juegos estocásticos que asumen un entorno no cooperativo pero \textit{fair} y evaluamos los mismos en una herramienta prototipo que extiende al \textit{model checker} {\Prism}~\cite{DBLP:conf/cav/KwiatkowskaN0S20,DBLP:conf/cav/KwiatkowskaNP11}. Este capítulo está basado en el artículo \textit{Playing Against Fair Adversaries in Stochastic Games with Total Rewards}\cite{CastroDDP22} publicado en \textit{International Conference on Computer Aided Verification (CAV) 2022}.
\item Capítulo \ref{cap:maskProb}: Se extiende la noción de relación de enmascaramiento a sistemas probabilistas, donde el entorno se asume \textit{fair}, y presentamos entonces una forma de medir \textit{tolerancia a fallas enmascarante} en este contexto. Luego se provee una evaluación experimental. Este capítulo está basado en un artículo que aun se encuentra bajo revisión, sin embargo se puede encontrar una versión preliminar en \cite{DBLP:journals/corr/abs-2207-02045}.
\item Capítulo \ref{cap:tool}: Se presentan dos herramientas de código abierto que implementan los conceptos de los Capítulos \ref{cap:maskingMeasure} y \ref{cap:maskProb}, respectivamente. Parte de este capítulo se basa en el artículo \textit{MaskD: A Tool for Measuring Masking Fault-Tolerance}\cite{PutrueleDCD22} publicado en \textit{International Conference on Tools and Algorithms for the Construction and Analysis of Systems (TACAS) 2022}. El resto del capítulo se basa en trabajo que aun está bajo revisión.
\item Capítulo \ref{cap:conclusiones}: Concluimos con discusiones finales y direcciones para trabajo futuro.
\end{itemize}
%En el capítulo \ref{cap:preliminares}, se introducen conceptos preliminares necesarios a lo largo de la tesis.
%Definimos en el capítulo \ref{cap:maskingMeasure} una relación de masking entre sistemas y presentamos la noción de distancia de masking entre sistemas tolerantes a fallas, y luego se introduce una herramienta de software de código abierto que implementa estas nociones.
%En el capítulo \ref{cap:fairAdversaries} describimos resultados sobre juegos estocásticos que asumen un entorno no cooperativo pero fair y evaluamos los mismos en una herramienta prototipo. 
%Los capítulos anteriores nos permiten finalmente en el capítulo \ref{cap:maskProb} extender las noción de relación de masking a sistemas probabilistas y presentamos entonces una forma de medir masking-tolerancia a fallas en este contexto.
%Finalmente, discutimos conclusiones y direcciones para trabajo futuro en \ref{cap:conclusiones}.


