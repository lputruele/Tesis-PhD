%!TEX root = main.tex
\chapter{Introducción}
\label{cap:introduccion}

La investigación en sistemas distribuidos tolerantes a fallas tiene como objetivo hacer que estos sistemas sean más confiables a la hora de manejar fallas en entornos informáticos complejos. Además, la creciente dependencia de la sociedad actual en sistemas informáticos que funcionen correctamente condujo a una creciente demanda de sistemas confiables; particularmente en sistemas con propiedades de confiabilidad cuantificables. La necesidad de dicha cuantificación es especialmente evidente en entornos de misión crítica como sistemas de control de vuelo, o software para controlar plantas de energía nuclear.

%Hasta los primeros años de la década de 1990, el trabajo en computación tolerante a fallas se centró en tecnologías y aplicaciones específicas, lo que resultó en subdisciplinas aparentemente no relacionadas con terminologías y metodologías distintas.
%Desde entonces, se ha avanzado mucho al ver el área de una manera más manera abstracta y formal. Esto ha llevado a una comprensión más clara de los problemas inherentes y esenciales en el campo y muestra lo que se puede hacer para aprovechar la complejidad de los sistemas que contrarrestan fallas. 
% En la práctica, se utilizan diversas técnicas para aumentar la confiabilidad de un algoritmo, por ejemplo: usando mecanismos de votación, roll-backs, protocolos randomizados, etcétera. Sin embargo, la mayoría de estas técnicas comúnmente se usan en una forma \textit{ad-hoc}, consecuentemente,  el análisis del grado de tolerancia a fallas que proveen tales técnicas es una tarea demandante y casi nunca es posible antes de que el software ya esté en uso, y por lo tanto las fallas se vuelvan visibles a los usuarios.

En esta tesis se definen relaciones de simulación que capturan la tolerancia a fallas de un sistema, así como caracterizaciones correspondientes en términos de juegos. Luego, se definen medidas que permitan cuantificar el grado de tolerancia a fallas exhibidas por un sistema. Primero se exploran estas nociones para sistemas no probabilistas y luego se consideran sistemas probabilistas. Para esto es necesario explorar propiedades sobre juegos estocásticos en los cuales el ambiente (representado como un jugador) se comporta de forma fair, en donde la noción de \textit{fairness} tiene la semántica usual en verificación de software \cite{BaierK08}.

Todas las técnicas desarrolladas en esta tesis se han implementado en herramientas de código abierto (las cuales se encuentran en repositorios disponibles al público) y a su vez se han realizado evaluaciones experimentales para validar nuestros resultados sobre diversos casos de estudio.

\section{Motivación y Objetivos}
\label{sec:intro.objetivos}

La tolerancia a fallas es una característica importante del software crítico, y puede ser definida como la capacidad de un sistema para lidiar con eventos inesperados, que pueden ser causados por bugs de programación, interacciones con un ambiente poco cooperativo, mal funcionamiento de hardware, etcétera. Se pueden encontrar ejemplos de sistemas tolerantes a fallas en casi cualquier parte: protocolos de comunicación, circuitos de hardware, sistemas aviónicas, cripto-monedas, etcétera. Por lo tanto, el incremento en la relevancia del software crítico en la vida cotidiana ha llevado a que se renueve el interés en la verificación automática de propiedades de tolerancia a fallas. Sin embargo, una de las dificultades principales a la hora de razonar sobre estos tipos de propiedades se da en su naturaleza cuantitativa, lo cual vale incluso en sistemas no probabilistas.
Un ejemplo simple se da con la introducción de redundancia en sistemas críticos. Esta es, sin lugar a dudas, una de las técnicas más utilizadas en tolerancia a fallas.
En la práctica, se sabe que al añadir más redundancia en un sistema incrementa su fiabilidad. Medir este incremento de fiabilidad es un problema central a la hora de evaluar software tolerante a fallas. Por otro lado, no hay un método \textit{de-facto} para caracterizar formalmente propiedades tolerantes a fallas, y por ello se suelen codificar utilizando mecanismos \textit{ad-hoc} como parte del diseño general.

La Teoría de Juegos \cite{MorgensternNeuman42} ofrece una teoría matemática elegante y profunda. 
En las ultimas décadas, ha recibido gran atención de la comunidad de ciencias de la computación, ya que tiene importantes aplicaciones en la verificación y síntesis de software. 
La analogía es atractiva, la operación de un sistema bajo un ambiente no cooperativo (hardware defectuoso, agentes maliciosos, canales de comunicación poco confiables, etc.) puede ser modelada como un juego entre dos jugadores (el sistema y el ambiente), en el cual el sistema trata de alcanzar ciertos objetivos, mientras que el ambiente pretende prevenir que esto suceda. 
Esta visión es particularmente útil para \textit{síntesis de controladores}, i.e., generación automática de políticas de toma de decisiones a partir de una especificación de alto nivel. 
Por lo tanto, sintetizar un controlador consiste de computar las estrategias óptimas para un juego dado. Al mismo tiempo, la teoría de juegos permite desarrollar métricas que pueden ser útiles para razonar sobre el nivel de tolerancia a fallas de un sistema dado.
En esta tesis nos enfocamos en juegos de dos jugadores, de suma cero, por turnos y de información perfecta con recompensas (no negativas)\cite{FilarV96}. 

Considerando los problemas mencionados anteriormente, los objetivos específicos de la investigación llevada a cabo en esta tesis son los siguientes:

\paragraph{Desarrollar Medidas para Tolerancia a Fallas}
\begin{itemize}
\item Obtener una caracterización formal de ciertas formas de tolerancia a fallas utilizando conceptos de teoría de juegos, haciendo énfasis en la \textit{masking-tolerancia a fallas}. 
    
\item Desarrollar algoritmos que permitan cuantificar el grado de \textit{masking-tolerancia a fallas} exhibido por un sistema.

\item Implementar, y evaluar, el rendimiento de dichas técnicas en herramientas de software de código abierto que puedan ayudar a un ingeniero a diseñar sistemas tolerantes a fallas, y evaluar su nivel de tolerancia.
\end{itemize}

\paragraph{Investigar propiedades sobre Juegos Estocásticos}
\begin{itemize}
\item Caracterizar formalmente juegos estocásticos entre un sistema y un entorno que se comporta de forma \textit{fair}. 
    
\item Desarrollar algoritmos que permitan derivar estrategias óptimas para un sistema estocástico.

\item Implementar y evaluar el rendimiento de dichas técnicas en herramientas de software de código abierto.
\end{itemize}


\section{Contribuciones}
\label{sec:intro.contribuciones}

%Se pueden mencionar papers publicados
%El trabajo de esta tesis dio fruto a tres artículos en conferencias de primer nivel \cite{CastroDDP18b} \cite{PutrueleDCD22} \cite{?}.
%A su vez se derivan de aquí múltiples trabajos que aún están bajo revisión y que no se presentan en esta tesis.
\myworries{add intro mencionando trabajo previo/relacionado}
En los últimos años, hubo avances en cuanto a generalizaciones cuantitativas de la noción booleana de correctitud y cuestiones relacionadas a verificación cuantitativa \cite{BokerCHK14,CernyHR12,Henzinger10,Henzinger13}.
Sin embargo, esta tesis elabora sobre estos conceptos adaptándolos y extendiéndolos a la cuantificación de masking-tolerancia a fallas.
A continuación se enumeran las contribuciones de esta tesis:

\begin{enumerate}
	
\item Definimos la relación de \textit{masking} entre dos sistemas por medio de relaciones de simulación y damos una caracterización correspondiente en términos de juegos. Consecuentemente, agregamos objetivos cuantitativos al juego para asi presentar una noción de distancia de masking-tolerancia a fallas entre sistemas. 
Utilizando esta distancia, es posible medir el grado de tolerancia de una implementación dada y compararla con otras para así seleccionar la más adecuada \cite{CastroDDP18b}.
	
\item Desarrollamos una herramienta automática diseñada para medir el nivel de tolerancia a fallas entre componentes de software, descritos por medio de un lenguaje de comandos con guarda. La herramienta se enfoca en medir componentes \textit{masking-tolerantes a fallas}, es decir, programas que enmascaran fallas de tal manera que no puedan ser observadas por el ambiente\cite{PutrueleDCD22}. 

\item Investigamos las propiedades de los juegos estocásticos con payoff de recompensa total bajo la suposición de un entorno \textit{fair}. Bajo este escenario, mostramos que se preserva la determinación y que los jugadores poseen estrategias óptimas sin memoria y deterministas, además, mostramos que el valor del juego puede  ser calculado en tiempo polinomial. Por último, validamos los resultados con una evaluación experimental \cite{CastroDDP22}

\item Introducimos una noción formal de \textit{masking-tolerancia a fallas} entre sistemas probabilistas basándonos en una variante de bisimulación probabilista. También damos la caracterización correspondiente en términos de juegos estocásticos. Desarrollamos una medida para cuantificar el nivel de tolerancia exhibido por sistemas almost-sure failing, i.e., sistemas que llegan a un estado inválido con probabilidad $1$.  Implementamos estas ideas en una herramienta prototipo y realizamos una evaluación experimental. \cite{DBLP:journals/corr/abs-2207-02045}
 
\end{enumerate}

\section{Organización}
\label{sec:intro.organizacion}
A continuación se realiza un vistazo general a la estructura del resto de la tesis:
\begin{itemize}
	
\item Capítulo \ref{cap:preliminares}: Se introducen conceptos preliminares necesarios a lo largo de la tesis.
\item Capítulo \ref{cap:maskingMeasure}: Se define una relación de \textit{masking} entre sistemas y presentamos la noción de distancia de \textit{masking} entre sistemas tolerantes a fallas, y luego se introduce una herramienta de software de código abierto que implementa estas nociones.
\item Capítulo \ref{cap:fairAdversaries}: Se describen resultados sobre juegos estocásticos que asumen un entorno no cooperativo pero \textit{fair} y evaluamos los mismos en una herramienta prototipo.
\item Capítulo \ref{cap:maskProb}: Se extiende la noción de relación de \textit{masking} a sistemas probabilistas y presentamos entonces una forma de medir \textit{masking-tolerancia a fallas} en este contexto.
\item Captítulo \ref{cap:conclusiones}: Finalmente, se discuten conclusiones y direcciones para trabajo futuro.
\end{itemize}
%En el capítulo \ref{cap:preliminares}, se introducen conceptos preliminares necesarios a lo largo de la tesis.
%Definimos en el capítulo \ref{cap:maskingMeasure} una relación de masking entre sistemas y presentamos la noción de distancia de masking entre sistemas tolerantes a fallas, y luego se introduce una herramienta de software de código abierto que implementa estas nociones.
%En el capítulo \ref{cap:fairAdversaries} describimos resultados sobre juegos estocásticos que asumen un entorno no cooperativo pero fair y evaluamos los mismos en una herramienta prototipo. 
%Los capítulos anteriores nos permiten finalmente en el capítulo \ref{cap:maskProb} extender las noción de relación de masking a sistemas probabilistas y presentamos entonces una forma de medir masking-tolerancia a fallas en este contexto.
%Finalmente, discutimos conclusiones y direcciones para trabajo futuro en \ref{cap:conclusiones}.


