\chapter{Medidas para Tolerancia Enmascarante en Sistemas Estocásticos}
\label{cap:maskProb}

En este capítulo introducimos una noción formal de tolerancia a fallas enmascarante entre sistemas de transición probabilistas que se basa en el concepto de bisimulación probabilista (al igual que en el capítulo \ref{cap:maskingMeasure} la llamamos simulación de enmascaramiento). También proveeremos la caracterización correspondiente en términos de un juego estocástico. A pesar de que estos juegos pueden tener una cantidad infinita de estados, proponemos una forma simbólica de representarlos, de tal manera que se pueda decidir en tiempo polinomial si existe una simulación de enmascaramiento entre dos sistemas de transición probabilistas.
%they can be solved in
%polynomial time.
%
Utilizamos esta noción de enmascaramiento para cuantificar el nivel de tolerancia a fallas enmascarante exhibido por sistemas \emph{que fallan casi-seguramente}, es decir, aquellos sistemas que eventualmente fracasan (llegan a un estado erróneo) con probabilidad $1$. En la práctica, esto abarca una gran cantidad de sistemas, ya que la ocurrencia de fallas durante de la ejecución de un sistema es, en general, un fenómeno probabilista; si consideramos que esta probabilidad es mayor a cero, entonces durante una ejecución lo suficientemente larga, es esperable que la  ocurrencia de fallas irá degradando al sistema hasta que este exhiba un estado de error. Más aún, es común asumir que el ambiente (y la forma de ocurrencia de fallas) se comporte de una forma justa o \emph{fair}, es decir, si un falla es posible que ocurra en cada momento, entonces ocurrirá. Debido que en lo sucesivo el ambiente es representado en los juegos por el minimizador, es decir, el comportamiento justo será estudiado solo sobre este jugador.
La asunción del comportamiento justo de las fallas es una observación que ya ha sido realizada en la literatura, por ejemplo en \cite{DBLP:conf/icse/DIppolitoBPU11}, en donde se ha mostrado utilizada para modelar sistemas falibles.

Técnicamente, el nivel de tolerancia a fallas enmascarante de sistemas que fallan casi-seguramente se puede calcular resolviendo una colección de ecuaciones funcionales.
%En lo que reste del capítulo denominaremos a los sistemas que fallan casi-seguramente, como sistemas \emph{casi-seguro terminantes}.
%

%% Moreover,  we generalize this approach to cope with systems where  the environment  behaves in a strong fair way, 
%% an assumption that has been useful  for designing critical software.
Además, se han implementado estas ideas en una herramienta, en este capítulo  realizamos una evaluación experimental de la misma  sobre varias instancias de tres casos de estudio generales.

\section{Introducción} \label{sec:intro_prob}

La tolerancia a fallas es una característica importante del software moderno. Esto es particularmente cierto para el software crítico como el software bancario,
aplicaciones de automoción, protocolos de comunicación, software de aviónica, etc.
solo por mencionar algunos ejemplos.
Sin embargo, en la práctica, es difícil cuantificar el nivel de tolerancia a fallas que brindan los sistemas informáticos. En la mayoría de los casos, los sistemas tolerantes a fallas se construyen usando técnicas ad-hoc que son basados en la experiencia y, muchas veces, carecen de fundamento matemático.
Además, las fallas suelen tener un carácter probabilístico. Por lo tanto, los conceptos provenientes de la teoría de la probabilidad se vuelven necesarios al desarrollar software tolerante a fallas. En este capítulo proporcionamos un framework destinado a analizar la tolerancia a fallas exhibida por sistemas probabilistas concurrentes. Esto abarca la probabilidad de ocurrencia de fallas, así como el uso de algoritmos aleatorios para mejorar la tolerancia a fallas de los sistemas.

 En la práctica, existen diferentes tipos de tolerancia a fallas, \emph{masking-tolerancia a fallas} (cuando tanto las propiedades de safety como las de liveness se conservan ante la ocurrencia de fallas),
\emph{nonmasking-tolerancia a fallas} (cuando solo se conservan las propiedades de liveness) y \emph{failsafe-tolerancia a fallas} (cuando solo se conservan las propiedades de safety).
Entre ellos, la masking-tolerancia a fallas se reconoce a menudo como el tipo de tolerancia a fallas más deseable, porque todas las propiedades del sistema nominal (es decir, no defectuoso) se conservan bajo un comportamiento defectuoso.
Sin embargo, en muchos entornos, no es realista exigir masking-tolerancia a fallas total. En particular, para aquellos sistemas que no están diseñados para terminar y, que con la degradación del hardware o software, los componentes
 eventualmente conducen a un error (es decir, un comportamiento que se desvía del comportamiento esperado del sistema). Una de las principales aplicaciones del framework descrito en las próximas secciones es la cuantificación de la
 cantidad de masking-tolerancia a fallas proporcionada por los sistemas antes de que entren a un estado de error. Esta medida proporciona una herramienta para seleccionar un mecanismo de tolerancia a fallas o para equilibrar múltiples mecanismos (por ejemplo, hasta qué punto vale la pena el costo de la redundancia de hardware eficiente frente a los artefactos de software que demandan tiempo).
	
Durante la última década, se han logrado avances significativos en cuanto a la definición de métricas o distancias adecuadas para diversos tipos de medidas para modelos cuantitativos, incluyendo sistemas en tiempo real \cite{HenzingerMP05}, modelos probabilísticos \cite{Bacci0LM17,BacciBLMTB19,DesharnaisGJP04,DesharnaisLT11,TangB18},
y métricas para sistemas lineales y ramificados \cite{CernyHR12,AlfaroFS09,Henzinger13,LarsenFT11,ThraneFL10}.
Algunos autores ya han señalado que estas métricas pueden ser útiles para razonar
sobre la robustez y corrección de un sistema, nociones relacionadas con la tolerancia a fallas.
En el capítulo~\ref{cap:maskingMeasure} se introdujo una noción de masking-tolerancia a fallas
entre sistemas, la cual se construye sobre una relación de simulación y una correspondiente
representación del juego con objetivos cuantitativos. En este capítulo ampliamos estas ideas a un entorno probabilístico y definimos una versión probabilística de
esta caracterización de masking-tolerancia a fallas.

Más específicamente, comenzamos caracterizando la masking-tolerancia a fallas probabilista a través de una variante de bisimulación probabilista. Esta \emph{simulación de enmascaramiento} relaciona dos sistemas de transición probabilistas. El primero actúa como una especificación del comportamiento previsto (es decir, modelo nominal) y el
el segundo como la implementación tolerante a fallas (es decir, el modelo extendido con comportamiento defectuoso). La existencia de una simulación de enmascaramiento implica
que la implementación enmascara las fallas. Esta relación de simulación se puede capturar como un juego estocástico jugado por un
\emph{Verificador} y \emph{Refutador}. Si el \emph{Verificador} gana, hay una simulación de enmascaramiento probabilista.
Si, en cambio, gana el \emph{Refutador}, la implementación no es masking-tolerante a fallas.
Estos juegos se basan en la noción de couplings entre distribuciones probabilísticas y, como consecuencia, el número de vértices de sus grafos de juego es infinita. Para abordar este problema, presentamos una representación simbólica de estos juegos en donde los couplings son capturados simbólicamente por medio de sistemas de ecuaciones.
El tamaño de estos grafos simbólicos
es polinomial en el tamaño de los sistemas de entrada. Además, los juegos de simulación
pueden resolverse a través de su representación simbólica.

En la práctica, la masking-tolerancia a fallas tiene una naturaleza cuantitativa, y así enriquecemos nuestros juegos con objetivos cuantitativos. Esto permite cuantificar la
cantidad de masking-tolerancia proporcionada por implementaciones tolerantes a fallas.
%
Nos enfocamos en juegos que llegan a un estado de error \textit{almost-surely} si el Refutador juega
de manera \textit{fair} (es decir, estos sistemas eventualmente llegaran a un estado de error con probabilidad $1$). Debido a la naturaleza infinita de los juegos, nos restringimos a estrategias aleatorias sin memoria y demostramos que nuestros juegos están
determinados bajo estas condiciones.
%
Probamos que el problema de decidir si un juego es \textit{almost-sure failing} bajo \textit{fairness} se puede resolver en tiempo polinomial. Además, el valor de un juego dado se puede calcular
resolviendo una colección de ecuaciones funcionales a través de una versión adaptada del algoritmo de \emph{Value Iteration} \cite{ChatterjeeH08,Condon90,Condon92,KelmendiKKW18}. Tomamos dicho valor como una medida de masking-tolerancia a fallas.

En resumen,
\begin{enumerate}
\item%
  definimos una noción de simulación de enmascaramiento probabilista,
\item%
  proveemos una caracterización en términos de juegos, y
\item%
  demostramos que los juegos resultantes se pueden decidir en tiempo polinomial.
%
Además, en la Sección~\ref{sec:almost_sure_prob},
\item%
  definimos una extensión de los juegos introducidos al considerar recompensas y proveemos una función de payoff que cuenta la cantidad de ``hitos'' lograda por una implementación;
\item%
  demostramos que estos juegos están determinados bajo la condición de que sean almost-surely
  failing bajo fairness y estrategias sin memoria, y
\item%
  proveemos un algoritmo para calcular el valor de estos juegos.
%
Adicionalmente,
\item%
  proporcionamos un algoritmo polinomial para decidir si un juego es almost-surely failing bajo fairness.
%
\item%
  Presentamos una evaluación experimental sobre algunos casos de estudio conocidos (Sec.~\ref{sec:experimental_eval_prob}).
\end{enumerate}
%

\section{Simulación de Masking Probabilista} \label{sec:mask_dist_prob}

En esta sección introducimos la simulacion de masking probabilista, la cual es una extensión probabilista de la relación introducida en el capítulo~\ref{cap:maskingMeasure}. Además proporcionamos una versión simbólica de la caracterización de juego de la relación, y proveemos un algoritmo para resolverlo.

%\subsection{The relation.}%
\subsection{La relación.}%
%
A grandes rasgos, una simulación de masking probabilista es una relación entre PTSs que extiende a la bisimulación probabilista~\cite{Larsen91}
para además tener en cuenta el enmascaramiento de fallas.  Intuitivamente, uno de los PTSs
cumple el rol de modelo nominal (o especificación), mientras que el otro PTS modela una implmentación del sistema bajo la ocurrencia de fallas. El modelo nominal describe el comportamiento del sistema cuando no se consideran fallas, mientras que la implementación describe una versión de la misma que es tolerante a fallas, donde la ocurrencia de fallas se toma en cuenta a través de un mecanismo de tolerancia a fallas que actúa sobre ellas.  La simulación de masking probabilista le permite a uno analizar si la implementación es capaz de enmascarar las fallas mientras se preserva el comportamiento de la especificación. Más especificamente, para transiciones no defectuosas, la relación de comporta como una bisimulación probabilista~\cite{Larsen91}, la cuál es capturada por medio de couplings y relaciones que respetan estos couplings. % (as done for instance in \cite{Larsen91}).
La parte novedosa está dada por la ocurrencia de fallas: si la implementación realiza una falla, el modelo nominal copia el movimiento haciendo nada.
%% then this step must be matched with a
%% Dirac self-transition by the specification, this represents a masking
%% step.  The formal definition is as follows.

%Before providing the formal definition, we fix some notation.
Para un conjunto de acciones $\Sigma$, y un conjunto (finito) de \emph{etiquetas de falla} $\faults$, con $\faults\cap\Sigma=\emptyset$, definimos $\SigmaF
= \Sigma \cup \faults$.  Intuitivamente, los elementos de $\faults$
indican la ocurrencia de una falla en una implementación defectuosa.
Además, consideramos el conjunto $\Sigma^i = \{ e^i
\mid e \in \Sigma\}$, que contiene los elementos de 
 $\Sigma$ indexado con el superíndice $i$.
%\remarkPC{No estoy seguro si hace falta definir $\faults$.}

\begin{definition} \label{def:masking_rel_prob}
  Sean $A =( S, \Sigma, {\rightarrow}, s_0 )$ y
  $A' =( S', \SigmaF, {\rightarrow'}, s_0' )$ dos PTSs que representan al modelo nominal y el modelo de implementación, respectivamente.
  %, where $\faults=\{F_0,\dots,F_n\}$ is a set of fresh labels.
  $A'$ es \emph{probabilisticamente masking-tolerante a fallas (fuerte)} con respecto a $A$ si y solo si existe una relación $\M \subseteq S \times S'$
  tal que:
  \begin{enumerate}[(a)]
  \item%
    $s_0 \M s'_0$, y
  \item%
    para todo $s \in S, s' \in S'$ con $s \M s'$ y todo $e \in \Sigma$
    y $F \in \faults$ se cumple que:
  \end{enumerate}%
 \begin{enumerate}[(1)]
 %\begin{inparaenum}[(1)]
  \item%
    si $s \xrightarrow{e} \mu$, entonces $s' \xrightarrowprime{e} \mu'$ y
    $\mu \MaskCoup \mu'$ for some $\mu'$;
  \item%
    si $s' \xrightarrowprime{e} \mu'$, entonces $s \xrightarrow{e} \mu$ y
    $\mu \MaskCoup \mu'$ for some $\mu$;
  \item%
    si $s' \xrightarrowprime{F} \mu'$, entonces $\Dirac_s \MaskCoup \mu'$.
  \end{enumerate}
   %\end{inparaenum}
  %
  Si tal relación existe decimos que $A'$ es una \emph{implementación probabilisticamente masking-tolerante a fallas (fuerte)} de $A$, 
    denotado por $A \Masking A'$.
\end{definition}

\begin{figure}[t]
\centering
\begin{minipage}[t]{.47\textwidth}
\fontsize{10}{10}\selectfont\ttfamily
\begin{tabbing}
x\=xxxxxxx\=xxxxxxxxxxxxx\=x\=xxx\= \kill    
module NOMINAL\\[1ex]
\>b : [0..1] init 0;\\
\>m : [0..1] init 0; \>\>// 0 = normal,\\
\>                   \>\>// 1 = refreshing\\[1ex]
\>[w0]   \>(m=0)          \>\>-> \>(b'= 0);\\
\>[w1]   \>(m=0)          \>\>-> \>(b'= 1);\\
\>[r0]   \>(m=0) \& (b=0) \>\>-> \>true;\\
\>[r1]   \>(m=0) \& (b=1) \>\>-> \>true;\\
\>[tick] \>(m=0)          \>\>-> \>p: (m'= 1) +\\
\>       \>               \>\>   \>(1-p): true;\\
\>[rfsh] \>(m=1)          \>\>-> \>(m'= 0);\\[1ex]
endmodule\\[-2em]
\end{tabbing}
 
\end{minipage}
\vspace{2ex}
\caption{Celda de Memoria probabilista: modelo nominal.} \label{fig:exam_1_mem_cell:nom}
\end{figure}


%
% BE AWARE OF THE HACKING FOR WRAPFIGURE TO WORK PROPERLY
%\begin{example}\label{example:memory}
\medskip\par\noindent\textit{Ejemplo~\refstepcounter{example}\label{example:memory}\theexample.}
Consideremos una celda de memoria que almacena un bit de información y que refresca su valor de manera periódica.  La memoria soporta operaciones de lectura y escritura, mientras que una operación de refresco realiza una lectura y sobre-escribe el valor de la celda con el valor leído.
%
%% Obviously, in this system, the result of a reading depends on the value stored in the cell. 
%% Thus, a property associated with the system is that the value read from the cell coincides with that of the last performed writing.
%% %
%% This is captured by the nominal model of
%% Fig.~\ref{fig:exam_1_mem_cell:nom} in  {\PRISM} 
%% notation~\cite{DBLP:conf/cav/KwiatkowskaNP11}.  Actions $\texttt{r}i$
%% and $\texttt{w}i$ (for $i=0,1$) represent the actions of reading or
%% writing  value $i$.  The bit stored in the memory is saved in
%% variable \texttt{b}.  A \texttt{tick} action indicates the passing of
%% one time unit and in doing so, with probability \texttt{p}, it
%% enables the refresh action (\texttt{rfsh}).  Variable \texttt{m}
%% indicates whether the system is in write/read mode, or producing a
%% refresh.
Este comportamiento está capturado por el modelo nominal de la Figura~\ref{fig:exam_1_mem_cell:nom} en notación {\PRISM}~\cite{DBLP:conf/cav/KwiatkowskaNP11}.  %Actions
$\texttt{r}i$
y $\texttt{w}i$ (para $i=0,1$) representan las acciones de lectura y escritura del valor $i$.  El bit almacenado en la memoria se guarda en la variable~\texttt{b}. La acción \texttt{tick} marca que pasó una unidad de tiempo y, con probabilidad \texttt{p}, permite una acción de refresco (\texttt{rfsh}). La variable \texttt{m}
indica si el sistema está operando en modo lectura/escritura o realizando un refresco.


%% \begin{figure}[t]
%% \begin{minipage}[t]{.45\textwidth}
%% \fontsize{6.6}{6.6}\selectfont\ttfamily
%% \begin{tabbing}
%% x\=xxxxxxx\=xxxxxxxxxxxxx\=x\=xxx\= \kill    
%% module NOMINAL\\[1ex]
%% \>b : [0..1] init 0;\\
%% \>m : [0..1] init 0; \>\>// 0 = normal,\\
%% \>                   \>\>// 1 = refreshing\\[1ex]
%% \>[w0]   \>(m=0)          \>\>-> \>(b'= 0);\\
%% \>[w1]   \>(m=0)          \>\>-> \>(b'= 1);\\
%% \>[r0]   \>(m=0) \& (b=0) \>\>-> \>true;\\
%% \>[r1]   \>(m=0) \& (b=1) \>\>-> \>true;\\
%% \>[tick] \>(m=0)          \>\>-> \>p : (m'= 1) +\\
%% \>       \>               \>\>   \>(1-p) : true;\\
%% \>[rfsh] \>(m=1)          \>\>-> \>(m'= 0);\\[1ex]
%% endmodule\\
%% \end{tabbing}
%% \end{minipage}
%% \hfill
%% \begin{minipage}[t]{.55\textwidth}
%% \fontsize{6.6}{6.6}\selectfont\ttfamily
%% \begin{tabbing}
%% x\=xxxxxxxx\=xxxxxxxxxxxxx\=xxx\=xxx\= \kill    
%% module FAULTY\\[1ex]
%% \>v : [0..3] init 0;\\
%% \>s : [0..2] init 0; \>\>// 0 = normal, 1 = faulty,\\
%% \>                   \>\>// 2 = refreshing\\
%% \>\textcolor{red}{f : [0..1] init 0;} \>\>\textcolor{red}{// fault limiting artifact}\\[1ex]
%% \>[w0]    \>(s!=2)           \>\>-> \>(v'= 0) \& (s'= 0);\\
%% \>[w1]    \>(s!=2)           \>\>-> \>(v'= 3) \& (s'= 0);\\
%% \>[r0]    \>(s!=2) \& (v<=1) \>\>-> \>true;\\
%% \>[r1]    \>(s!=2) \& (v>=2) \>\>-> \>true;\\
%% \>[tick]  \>(s!=2)           \>\>-> \>p : (s'= 2) + q : (s'= 1) \\
%% \>        \>                 \>\>   \>+ (1-p-q) : true;\\
%% \>[rfsh]  \>(s=2)            \>\>-> \>(s'=0)\\
%% \>        \>                 \>\>   \>\& (v'= (v<=1) ? 0 : 3);\\
%% \>[fault] \>(s=1) \textcolor{red}{\& (f<1)}   \>\>-> \>(v'= (v<3) ? (v+1) : 2) \\
%% \>        \>                 \>\>   \>\& (s'= 0) \textcolor{red}{\& (f'= f+1)};\\
%% \>[fault] \>(s=1) \textcolor{red}{\& (f<1)}    \>\>-> \>(v'= (v>0) ? (v-1) : 1) \\
%% \>        \>                 \>\>   \>\& (s'= 0) \textcolor{red}{\& (f'= f+1)};\\[1ex]
%% endmodule\\
%% \end{tabbing}
%% \end{minipage}

%% \caption{Nominal and fault-tolerant models for the memory cell.} \label{fig:exam_1_mem_cell}
%% %% \caption{Models for the memory cell. On the left is the Nominal Model and on the right is the Fault-tolerant Model.} \label{fig:exam_1_mem_cell}
%% \end{figure}

Una potencial falla en este escenario ocurre cuando la celda cambia de valor inesperadamente. % (e.g.,  as a consequence of some electromagnetic interference).
En la práctica, la ocurrencia de tal falla ocurre con una determinada probabilidad. Una técnica típica para lidiar con esta situación es a través de \emph{redundancia}, %for instance
e.g., utilizando tres bits en lugar de uno. Entonces, las operaciones de escritura se realizan simultaneamete sobre todos los bits mientras que una lectura devuelve el valor de la mayoría (o \emph{votación}).
\begin{figure}[t]
\centering
\begin{minipage}[t]{.47\textwidth}
\fontsize{10}{10}\selectfont\ttfamily
\begin{tabbing}
x\=xxxxxxxx\=xxxxxxxxxxxxx\=xxx\=xxx\= \kill    
module FAULTY\\[1ex]
\>v : [0..3] init 0;\\
\>s : [0..2] init 0; \>\>// 0 = normal, 1 = faulty,\\
\>                   \>\>// 2 = refreshing\\
\>\textcolor{red}{f : [0..1] init 0;} \>\>\textcolor{red}{// fault limiting artifact}\\[1ex]
\>[w0]    \>(s!=2)           \>\>-> \>(v'= 0) \& (s'= 0);\\
\>[w1]    \>(s!=2)           \>\>-> \>(v'= 3) \& (s'= 0);\\
\>[r0]    \>(s!=2) \& (v<=1) \>\>-> \>true;\\
\>[r1]    \>(s!=2) \& (v>=2) \>\>-> \>true;\\
\>[tick]  \>(s!=2)           \>\>-> \>p: (s'= 2) + q: (s'= 1) \\
\>        \>                 \>\>   \>+ (1-p-q): true;\\
\>[rfsh]  \>(s=2)            \>\>-> \>(s'=0)\\
\>        \>                 \>\>   \>\& (v'= (v<=1) ? 0 : 3);\\
\>[fault] \>(s=1) \textcolor{red}{\& (f<1)}   \>\>-> \>(v'= (v<3) ? (v+1) : 2) \\
\>        \>                 \>\>   \>\& (s'= 0) \textcolor{red}{\& (f'= f+1)};\\
\>[fault] \>(s=1) \textcolor{red}{\& (f<1)}    \>\>-> \>(v'= (v>0) ? (v-1) : 1) \\
\>        \>                 \>\>   \>\& (s'= 0) \textcolor{red}{\& (f'= f+1)};\\[1ex]
endmodule\\[-2em]
\end{tabbing}
 
\end{minipage}
\vspace{2ex}
\caption{Celda de Memoria probabilista: modelo tolerante a fallas.} \label{fig:exam_1_mem_cell:ft}
\end{figure}
%

%

%
%The right hand-side model of
La Figura~\ref{fig:exam_1_mem_cell:ft} modela esta implementación con la ocurrencia de la falla modelada implícitamente (ignoremos, por el momento, la parte roja).  La variable \texttt{v} cuenta los votos para el valor 1.
%% Thus writing 1 (\texttt{w1}) sets \texttt{v} in 3, and writing 0
%% (\texttt{w0}) sets it in 0.  The read action returns 1
%% (\texttt{r1}) if $\texttt{v}\geq 2$ and 0 (\texttt{r0}) otherwise.
%
Además de permitir la acción de refrescar, un \texttt{tick} puede también permitir la ocurrencia de una falla con probabilidad \texttt{q}, con
$\texttt{p}+\texttt{q}\leq 1$.
%
La variable \texttt{s} indica si el sistema está en modo normal
($\texttt{s}=0$), en un estado donde una falla puede ocurrir ($\texttt{s}=1$),
o produciendo un refresco ($\texttt{s}=2$).
%% Notice that reading and
%% writing are allowed as long as the system is not producing a refresh.
%
El texto de color rojo en la Figura~\ref{fig:exam_1_mem_cell:ft} es un artefacto para limitar el número de fallas a 1.  Bajo esta condición, la relación
%
$\M = $ $\{{\langle(b,m),(v,s,f)\rangle} \mid {{2b\leq v\leq 2b{+}1} \wedge (m=1 \Leftrightarrow s=2)}\}$
%
es una simulación de masking probabilista ($b$, $m$, $v$, $s$, y
$f$ representan los valores de las variables \texttt{b}, \texttt{m},
\texttt{v}, \texttt{s}, y \texttt{f}, respectivamente.)
%
Debería ser evidente que, si la parte roja del texto no estuviera presente, entonces
\texttt{FAULTY} no seria una implementación masking-tolerante a fallas de \texttt{NOMINAL}. 



%\subsection{A characterization in terms of a stochastic game.}
\subsection{Una caracterización en términos de juegos estocásticos.}

%In the following, we
Definimos un juego estocástico de simulación de masking para cualquier modelo nominal
$A = ( S, \Sigma, {\rightarrow}, s_0 )$ y modelo de implementación $A' = ( S', \Sigma_{\mathcal{F}}, {\rightarrow'}, s'_0 )$ dados. 
El juego es similar a un juego de bisimulación \cite{Stirling98}, y es jugado por dos jugadores, 
denominados por conveniencia, el Refutador ($\Refuter$) y el Verificador ($\Verifier$). El Verificador quiere probar que $s \in S$ y $s' \in S'$ son probabilisticamente masking-similares, 
y la intención del Refutador es la de refutar esto.
El juego comienza desde el par de estados $(s,s')$ y se cicla sobre los siguientes pasos:
\begin{enumerate}
\item[1)]
  $\Refuter$ elige una transición $s \xrightarrow{a} \mu$ del modelo nominal o una transición $s' \xrightarrow{a} \mu'$ de la implementación;
\item[2a)]
  Si $a \notin \faults$, $\Verifier$ elige una transición que empareje la acción $a$ del modelo opuesto, i.e., una transición $s'
  \xrightarrow{a} \mu'$ si la elección del $\Refuter$ fué en el modelo nominal,
  o una transición $s \xrightarrow{a} \mu$ en caso contrario.  Además,
  $\Verifier$ elige un coupling $w$ para $(\mu, \mu')$;
\item[2b)]
  Si $a \in \faults$, el $\Verifier$ solo puede seleccionar la distribución Dirac $\Dirac_{s}$ y el único coupling posible $w$ para
  $(\Dirac_{s}, \mu')$;
\item[3)]
  El par de estados sucesores $(t, t')$ se elige probabilísticamente de acuerdo a $w$.
%% \item[1)] $\Refuter$ chooses action $a$ and distribution $\mu$ (resp. $\mu'$) such that $s \xrightarrow{a} \mu$ 
%% 		(resp. $s' \xrightarrow{a} \mu'$);
%% \item[2a)] If $a \notin \faults$, $\Verifier$ chooses a matching action $a$ and distribution $\mu'$ (resp. $\mu$) and 
%% 		a coupling $w$ such that $s' \xrightarrow{a} \mu'$ (resp. $s \xrightarrow{a} \mu$) and $w$ is coupling for $(\mu, \mu')$;
%% \item[2b)] If $a \in \faults$, $\Verifier$ selects the Dirac distribution  $\Dirac_{s}$ and a coupling $w$, such that 
%% 		$w$ is coupling for  $(\Dirac_{s}, \mu')$;
%% \item[3)] The successor pair of states $(t, t')$ is chosen probabilistically according to $w$.
\end{enumerate}
%
Si la jugada continua eternamente, entonces gana el Verificador; en caso contrario, el Refutador gana. (Observemos, particularmente, que el Verificador pierde si no puede emparejar la etiqueta de una transición, ya que siempre es posible la elección de un coupling arbitrario.)  El paso 2b es el único que difiere del juego de bisimulación típico.  Esto es necesario por la asimetría causada por las transiciones etiquetadas como fallas. Intuitivamente,  si el Refutador elige jugar con una falla en la implementación, entonces el Verificador debe enmascarar la falla, y no puede moverse libremente en el modelo nominal.  Resumiendo, el paso probabilista de una falla solo puede ser emparejado por una distribución Dirac sobre el estado correspondiente de la especificación.

A continuación definimos el grafo de juego de masking estocástico que formaliza esta idea.

\begin{definition} \label{def:strong_masking_game_graphi}
  Sean $A =( S, \Sigma, {\rightarrow}, s_0 )$ y
  $A' = ( S', \Sigma_\faults , {\rightarrow'}, s'_0 )$ dos PTSs.
  El \emph{grafo de juego de masking estocástico} de dos jugadores
  $\StochG_{A,A'} = (V^\StochG, E^\StochG, V^\StochG_\Refuter, V^\StochG_\Verifier, V^\StochG_\Probabilistic, \InitVertex, \delta^\StochG)$,
  %with initial state $\InitVertex$,
   se define de la siguiente manera:
%
  {\footnotesize%
  \begin{align*}
    V^\StochG = \
    & V^\StochG_\Refuter \cup V^\StochG_\Verifier \cup V^\StochG_\Probabilistic, \text{donde: }\\
    V^\StochG_\Refuter = \
    & \{ (s, \mhyphen, s', \mhyphen, \mhyphen, \mhyphen, \Refuter) \mid
          s \in S \wedge s' \in S' \} \cup
      \{\ErrorSt\}\\
    V^\StochG_\Verifier = \
    & \{ (s, \sigma^1, s', \mu, \mhyphen, \mhyphen, \Verifier) \mid
         s \in S \wedge s' \in S'
%         \wedge \sigma \in (\Sigma^1 \cup \SigmaF^2 )
         \wedge \sigma \in \Sigma
         \wedge (s, \sigma, \mu) \in {\rightarrow} \} \cup {} \\
    & \{ (s, \sigma^2, s', \mhyphen, \mu', \mhyphen, \Verifier) \mid
         s \in S \wedge s' \in S'
%         \wedge \sigma \in ( \Sigma^1 \cup \SigmaF^2)
         \wedge \sigma \in \SigmaF
         \wedge (s', \sigma, \mu') \in {\rightarrow'}\}\\
    V^\StochG_\Probabilistic = \
    & \{ (s, \mhyphen, s', \mu, \mu', w, \Probabilistic) \mid
         s \in S \wedge s' \in S' \wedge %{} \\
    % & \phantom{\{ (s, \mhyphen, s', \mu, \mu', w, \Probabilistic) \mid {}}
         \mu \in \Dist(S) \wedge \mu' \in \Dist(S')
         \wedge w \in \couplings{\mu}{\mu'}\}\\
    \InitVertex = \
    & ( s_0, \mhyphen, s'_0, \mhyphen, \mhyphen, \mhyphen, \Refuter )
    \text{ \ (el Refutador comienza a jugar)}\\
    & \hspace{-2.6em}
    \delta^\StochG : V^\StochG_\Probabilistic \rightarrow \Dist(V^\StochG_\Refuter),
      \text{ definido por } 
      \delta^\StochG((s, \mhyphen, s', \mu, \mu', w, \Probabilistic))((t, \mhyphen, t', \mhyphen, \mhyphen, \mhyphen, \Refuter)) = w(t,t')
      \text{,}
  \end{align*}
  }%
%
  y $E^\StochG$ es el conjunto mínimo que satisface las siguientes reglas:
%
  {\footnotesize%
  \begin{align*}
    s \xrightarrow{\sigma} \mu
    & \Rightarrow \tuple{(s, \mhyphen, s', \mhyphen, \mhyphen, \mhyphen, \Refuter), (s, \sigma^{1}, s', \mu, \mhyphen, \mhyphen, \Verifier)}\in E^\StochG
    \tag{1$_1$}\label{play:r:1}\\
    s' \xrightarrowprime{\sigma} \mu'
    & \Rightarrow \tuple{(s, \mhyphen, s', \mhyphen, \mhyphen, \mhyphen, \Refuter),(s, \sigma^{2}, s', \mhyphen, \mu', \mhyphen, \Verifier)}\in E^\StochG
    \tag{1$_2$}\label{play:r:2}\\
    {s' \xrightarrowprime{\sigma} \mu'} \wedge {w \in \couplings{\mu}{\mu'}}
    & \Rightarrow \tuple{(s, \sigma^1, s', \mu, \mhyphen, \mhyphen, \Verifier),(s, \mhyphen, s', \mu, \mu', w, \Probabilistic)}\in E^\StochG
    \tag{2a$_1$}\label{play:v:1}\\
    \!\!\!{\sigma \notin \faults} \wedge {s \xrightarrow{\sigma} \mu} \wedge {w \in \couplings{\mu}{\mu'}}
    & \Rightarrow \tuple{(s, \sigma^2, s', \mhyphen, \mu', \mhyphen, \Verifier), (s, \mhyphen, s', \mu, \mu', w, \Probabilistic)}\in E^\StochG
    \tag{2a$_2$}\label{play:v:2a}\\
    {F \in \faults} \wedge {w \in \couplings{\Dirac_s}{\mu'}}
    & \Rightarrow \tuple{(s, F^2, s', \mhyphen, \mu', \mhyphen, \Verifier), (s, \mhyphen, s', \Dirac_s, \mu', w, \Probabilistic)}\in E^\StochG
    \tag{2b}\label{play:v:2b}\\
    (s, \mhyphen, s', \mu, \mu', w, \Probabilistic) \in V^\StochG_\Probabilistic \wedge & \\ 
    (t,t') \in \support{w} 
    & \Rightarrow \tuple{(s, \mhyphen, s', \mu, \mu', w, \Probabilistic), (t, \mhyphen, t', \mhyphen, \mhyphen, \mhyphen, \Refuter)}\in E^\StochG
    \tag{3}\label{play:p}\\
    {v\in (V^\StochG_\Verifier{\cup}\{\ErrorSt\})} \wedge ({\nexists v'} & {}\neq\ErrorSt : {\tuple{v,v'}\in E^\StochG})
    \Rightarrow  \tuple{v,\ErrorSt}\in E^\StochG
    \tag{err}\label{play:err}
  \end{align*}
  }%
\end{definition}



La definición sigue la idea de juego previamente descrita. Una ronda del juego comienza en el estado del Refutador $\InitVertex$.  Observemos que, en este punto, solo los estados corrientes del modelo nominal y el modelo de implementación son relevantes (la información restante está aún indefinida en esta ronda y por lo tanto se marca con ``$\mhyphen$'').  El Paso 1 de este juego se codifica en las reglas
(\ref{play:r:1}) y (\ref{play:r:2}), donde el Refutador elige una transición, por lo tanto definiendo la acción y la distribución que necesita ser emparejada,  esto mueve el juego a un estado del Verificador. Un estado del Verificador en $V^\StochG_\Verifier$ es una tupla que contiene la acción y la distribución a emparejar, y que modelo movilizó el Refutador. El Paso 2a del juego está dado por las reglas (\ref{play:v:1}) y
(\ref{play:v:2a}) en las cuales el Verificador elige un movimiento emparejador del modelo opuesto (y asi definiendo la otra distribución) y un coupling apropiado, lo que nos lleva a un estado probabilista.  El Paso 2b
del juego está codificado en la regla (\ref{play:v:2b}).  Aquí, el Verificador no tiene alternativa ya que está obligado a elegir la distribución Dirac
$\Dirac_s$ y el único coupling disponible en
$\couplings{\Dirac_s}{\mu'}$.  Un estado probabilista en
$V^\StochG_\Probabilistic$ contiene la información necesaria para resolver probabilisticamente el próximo paso a través de la función $\delta^\StochG$ (rule
(\ref{play:p})).
%% For completness, the corresponding edges are added in the sixth rule.
Finalmente, si un jugador no tiene movimientos disponibles, solo se puede mover al estado de error $\ErrorSt$ (regla (\ref{play:err})). Esto solo puede ocurrir en un estado del Verificador o en
$\ErrorSt$.


La noción de simulación de masking probabilista se puede capturar por el juego de masking estocástico correspondiente con el objetivo Booleano apropiado.

\begin{figure}
\begin{center}
  \scalebox{0.7}{
    \begin{tikzpicture}[on grid,auto,align at top,,
        hvh path/.style={to path={-- ++(#1,0) |- (\tikztotarget)},rounded corners}]

      \node (ar) {$\left((0,0),\mhyphen,(0,0,\textcolor{red}{0}),\mhyphen,\mhyphen,\mhyphen,\Refuter\right)$};
      \node (av) [below=0.75 of ar]  {$\left((0,0),\texttt{tick},(0,0,\textcolor{red}{0}),\mu,\mhyphen,\mhyphen,\Verifier\right)$};
      \node (ap) [below=0.8 of av]  {$\left((0,0),\texttt{tick},(0,0,\textcolor{red}{0}),\mu,\mu',w_0,\Probabilistic\right)$};
      \node[dot] (apdot) [below=.7 of ap] {};
      \node (br) [below right=.7 and 1.3 of apdot] {$\left((0,0),\mhyphen,(0,1,\textcolor{red}{0}),\mhyphen,\mhyphen,\mhyphen,\Refuter\right)$};
      \node (brign) [below left=.7 and 2.3 of apdot] {$\left((0,1),\mhyphen,(0,2,\textcolor{red}{0}),\mhyphen,\mhyphen,\mhyphen,\Refuter\right)$};
      \node (bv) [below left=.7 and 1.3 of br] {$\left((0,0),\texttt{fault},(0,1,\textcolor{red}{0}),\mhyphen,\Dirac_{(1,0,\textcolor{red}{1})},\mhyphen,\Verifier\right)$};
      \node (bp) [below =.8 of bv] {$\left((0,0),\texttt{fault},(0,1,\textcolor{red}{0}),\Dirac_{(0,0)},\Dirac_{(1,0,\textcolor{red}{1})},w_1,\Probabilistic\right)$};
      \node (cr) [below =.8 of bp] {$\left((0,0),\mhyphen,(1,0,\textcolor{red}{1}),\mhyphen,\mhyphen,\mhyphen,\Refuter\right)$};
      \node (cv) [below=.75 of cr]  {$\left((0,0),\texttt{tick},(1,0,\textcolor{red}{1}),\mu,\mhyphen,\mhyphen,\Verifier\right)$};
      \node (cp) [below=.8 of cv]  {$\left((0,0),\texttt{tick},(1,0,\textcolor{red}{1}),\mu,\mu'',w_2,\Probabilistic\right)$};
      \node[dot] (cpdot) [below=.7 of cp] {};
      \node (dr) [below right=.7 and 1.3 of cpdot] {$\left((0,0),\mhyphen,(1,1,\textcolor{red}{1}),\mhyphen,\mhyphen,\mhyphen,\Refuter\right)$};
      \node (drign) [below left=.7 and 2.3 of cpdot] {$\left((0,1),\mhyphen,(1,2,\textcolor{red}{1}),\mhyphen,\mhyphen,\mhyphen,\Refuter\right)$};
      \node (dv) [below left=.7 and 1.3 of dr] {$\left((0,0),\texttt{fault},(1,1),\mhyphen,\Dirac_{(2,0)},\mhyphen,\Verifier\right)$};
      \node (dp) [below =.8 of dv] {$\left((0,0),\texttt{fault},(2,1),\Dirac_{(0,0)},\Dirac_{(2,0)},w_3,\Probabilistic\right)$};
      \node (er) [below =.8 of dp] {$\left((0,0),\mhyphen,(2,0),\mhyphen,\mhyphen,\mhyphen,\Refuter\right)$};
      \node (ev) [below =.8 of er] {$\left((0,0),\texttt{r0},(2,0),\Dirac_{(0,0)},\mhyphen,\mhyphen,\Verifier\right)$};
      \node (err) [below =.75 of ev] {$\ErrorSt$};


      \node (label) [above right=.2 and .8 of apdot] {$1{-}\texttt{p}{-}\texttt{q}$};
      \node (label) [above right=.2 and .8 of cpdot] {$1{-}\texttt{p}{-}\texttt{q}$};
      
      \path[-,line width=0.5pt]
      (ap) edge[] (apdot)
      (cp) edge[] (cpdot)
      ;
      
      \path[-{Latex[length=1.35mm,width=0.9mm]},line width=0.5pt]
      (ar) edge[] (av)
      (av) edge[] (ap)
      (apdot) edge[] node[below,pos=0.15] {$\texttt{q}$} (br)
      (apdot) edge[] node[above] {$\texttt{p}$} (brign)
      (apdot.east) edge[hvh path=24mm] (ar.east)   
      (br) edge[] (bv)
      (bv) edge[] (bp)
      (bp) edge[] (cr)
      (cr) edge[] (cv)
      (cv) edge[] (cp)
      (cpdot) edge[] node[below,pos=0.15] {$\texttt{q}$} (dr)
      (cpdot) edge[] node[above] {$\texttt{p}$} (drign)
      (cpdot.east) edge[hvh path=24mm] (cr.east)   
      (dr) edge[] (dv)
      (dv) edge[] (dp)
      (dp) edge[] (er)
      (er) edge[] (ev)
      (ev) edge[] (err)
      (err) edge[loop,out=0,in=45,looseness=4] (err)
      ;

      \scoped[on background layer]{
        \node (fznw) [above left=.4 and 2.8 of dv] {};
        \node (fzse) [below right=3.8 and 5.6 of fznw] {};

        \path[fill=red!10]
        (fznw) rectangle (fzse)
        ;
      }
    \end{tikzpicture}
  }
  
  \caption{Fragmento de un grafo de juego de masking}\label{fig:masking:game:graph}
\end{center}
\end{figure}


\begin{theorem} \label{thm:wingame_strat_prob}
  Sean $A= ( S, \Sigma, {\rightarrow}, s_0 )$ y $A'=( S', \SigmaF, {\rightarrow'}, s_0' )$ dos PTSs.
  Tenemos que $A \Masking A'$ si y solo si el Verificador posee una estrategia \textit{sure-ganadora (o almost-sure-ganadora)} para el grafo de juego de masking estocástico
  $\mathcal{G}_{A,A'}$ con el objetivo Booleano $\neg \Diamond \ErrorSt$.
%$\Phi = \{ \omega = \omega_0,\omega_1, \dots  \in \Omega \mid \forall {i \geq 0} : \omega_i \neq \ErrorSt \}$.
\end{theorem}

\noindent
%\textbf{Proof of Theorem \ref{thm:wingame_strat_prob}.}
%  Sean $A= ( S, \Sigma, \rightarrow, s_0 )$ y $A'=( S', \SigmaF, \rightarrow', s_0' )$ dos PTSs.
%  Tenemos que $A \Masking A'$ si y solo si el Verificador posee una estrategia ganadora sure (o almost-sure) para el grafo de juego de masking estocástico
%  $\mathcal{G}_{A,A'}$ con el objetivo Booleano
%$\Phi = \{ \omega_0,\omega_1, \dots  \in \Omega \mid \forall {i \geq 0} : \omega_i \neq \ErrorSt \}$. \\
\noindent
\begin{proof} 
 Vale la pena destacar que para objetivos de safety (e.g., $\neg \Diamond \ErrorSt$) las estrategias sure-ganadoras y almost-sure-ganadoras son equivalentes. Por lo que solo probamos el teorema para las estrategias sure-ganadoras.
  
\noindent ``Solo Si'': Asumamos $A \Masking A'$, por lo tanto existe una simulación de masking probabilista $\M \subseteq S \times S'$.
Definamos una estrategia sure-ganadora  $\strat{\Verifier}$ para el Verificador de la siguiente manera.
Dado un estado $(s, \sigma^1, s', \mu, \mhyphen, \mhyphen, \Verifier)$ (resp. $(s, \sigma^2, s', \mhyphen, \mu', \mhyphen, \Verifier)$), si $s \M  s'$, $\strat{\Verifier}$ selecciona una transición
$\langle (s, \sigma^1, s', \mu, \mhyphen, \mhyphen, \Verifier), (s, \sigma^1, s', \mu, \mu', w, \Probabilistic)\rangle$ (resp. $\langle (s, \sigma^2, s', \mhyphen, \mu', \mhyphen, \Verifier),(s, \sigma^2, s', \mu, \mu', w, \Probabilistic)\rangle$) tal que $w$ es un coupling $\M$-respetuoso para ($\mu,\mu'$) 
(cuya existencia está garantizada por la Def.~\ref{def:masking_rel_prob}). En caso contrario, $\strat{\Verifier}$ selecciona un vértice arbitrario. 
Vamos a demostrar que esta estrategia es sure-ganadora para el verificador en el estado inicial.
Tenemos que probar que, para cualquier estrategia del Refutador $\strat{\Refuter}$, tenemos $\out(\strat{\Verifier}, \strat{\Refuter}) \subseteq\Omega \setminus \Phi$, donde $\out(\strat{\Verifier}, \strat{\Refuter})$ denota el conjunto de caminos generados cuando las estrategias $\strat{\Refuter}$ y $\strat{\Verifier}$ son utilizadas.  Sea $\strat{\Refuter}$ una estrategia para el Refutador, y $\omega = \omega_0, \omega_1,  \dots$ una jugada en $\text{out}(\strat{\Verifier}, \strat{\Refuter})$. 
Demostramos por inducción sobre $i$ que:
\begin{align} 
\forall i \geq 0: & \omega_i \neq \ErrorSt \wedge (\pr{6}{\omega_i} = \Verifier \Rightarrow \pr{0}{\omega_i} \M  \pr{2}{\omega_i}) \nonumber \\ 
&\wedge (\pr{6}{\omega_i} = \Probabilistic \Rightarrow \text{ $\pr{5}{\omega_i}$ es un coupling $\M$-respetuoso para $(\pr{3}{\omega_i},  \pr{4}{\omega_i})$}). \label{eq1:thm:wingame_strat_prob}
\end{align}
Para $i=0$, la prueba es directa. Asumamos que la propiedad vale para $\omega_i$, si $\omega_i$ es un vértice del Verificador y 
$\pr{1}{\omega} = \sigma^1$ (resp. $\sigma^2$) con $\sigma \notin \faults$, entonces por definición de $\strat{\Verifier}$, Def.~\ref{def:masking_rel_prob} e hipótesis inductiva, 
tenemos que $\omega_{i+1} =  (s, \sigma^1, s', \mu, \mu', w,  \Probabilistic)$
(resp. $\omega_{i+1} =  (s, \sigma^2, s', \mu, \mu', w, \Probabilistic)$) donde $w$ es un coupling
$\M$-respetuoso para $(\pr{3}{\omega_i},  \pr{4}{\omega_i})$ y también que  
$\omega_{i+1} \neq \ErrorSt$. Si $\pr{1}{\omega_i} \in \faults$,
entonces la prueba es similar, pero tomando en cuenta que $\mu = \Dirac_s$.
Si $\omega_i$ es un vértice del Refutador, entonces $\omega_{i+1}$ es un vértice del Verificador, y no puede ser $\ErrorSt$ porque, por construcción, solo los nodos del Verificador son adyacentes a $\ErrorSt$.
Si $\omega_i$ es un vértice probabilista, entonces por hipótesis inductiva, tenemos que $\support{\pr{5}{\omega_i}} \neq \emptyset$ y por lo tanto $\ErrorSt \neq \omega_{i+1}$,
y además como $\pr{5}{\omega_i}$ es un coupling $\M$-respetuoso para $(\pr{3}{\omega_i},  \pr{4}{\omega_i})$, también obtenemos  
$ \pr{0}{\omega_{i+1}}  \M \pr{2}{\omega_{i+1}}$. Por lo tanto la propiedad (\ref{eq1:thm:wingame_strat_prob}) queda demostrada,  eso implica que:
$\forall i \geq 0 : \omega_i \neq \ErrorSt$. Por lo tanto, $\omega \in \Omega \setminus \Phi$.

``Si'': Primero, es necesario introducir un poco de notación, dada una función $f: A \rightarrow \Dist(B)$,  y $S \subseteq A$, 
consideramos el conjunto $f(S)= \{ b \in B \mid \exists a \in S: f(a)(b) > 0 \}$.
De forma similar, para $T \subseteq B$, definimos $f^{-1}(T) = \{ a \in A \mid \exists b \in T : f(a)(b) > 0\}$

Ahora bien,  supongamos que el Verificador posee una estrategia sure-ganadora $\strat{\Verifier}$
desde el estado inicial. Entonces, definimos una relación de masking probabilista de la siguiente manera: 
\[
\M =\{(s,s') \mid \post((s, \mhyphen, s', \mhyphen, \mhyphen, \mhyphen, \Refuter)) \subseteq \strat{\Verifier}^{-1}(V^{\StochG}_\Probabilistic) \text{ para alguna estrategia sure-ganadora} \strat{\Verifier} \}
\]
%$\M =\{(s,s') \mid (s, \mhyphen, s', \mu, \mu', w, \Probabilistic) \in \strat{\Verifier}(V^{\StochG}_\Verifier)$ for some sure winning strategy $\strat{\Verifier} \}$. 
Sabemos, por nuestro supuesto, que este conjunto no es vacío y que es directo de ver que $(s_0,s'_0) \in \M$. 
Primero, vamos a demostrar que para cualquier $(s, \mhyphen, s', \mu, \mu', w, \Probabilistic) \in \strat{\Verifier}(V^{\StochG}_\Verifier)$,  tal que   $\strat{\Verifier}$ es una estrategia sure-ganadora,  
tenemos que  $\mu \MaskCoup \mu'$. 
Asumamos $(s, \mhyphen, s', \mu, \mu', w, \Probabilistic) \in \strat{\Verifier}(V^{\StochG}_\Verifier)$ para alguna estrategia sure-ganadora $\strat{\Verifier}$, y no es el caso que $\mu \MaskCoup \mu'$, o equivalentemente 
$\exists t,t' : w(t,t') > 0 \wedge \neg (t \M t')$. 
Por lo tanto, tenemos un sucesor $(t, \mhyphen,t', \mhyphen,\mhyphen,\mhyphen, \Refuter)$ de
$(s, \mhyphen, s', \mu, \mu', w, \Probabilistic)$ el cual puede ser escogido con probabilidad mayor a $0$ y  $(t,t') \notin \M$. 
Además, existe un $t \xrightarrow{\sigma} \mu$ 
(o $t' \xrightarrowprime{\sigma} \mu'$)tal que $(t, \sigma^1, t', \mu, \mhyphen, \mhyphen, \Verifier) \in \post((t, \mhyphen,t', \mhyphen,\mhyphen,\mhyphen, \Refuter))$ 
(resp. $(t, \sigma^2, t', \mhyphen,  \mu', \mhyphen, \Verifier) \in \post((t, \mhyphen,t', \mhyphen,\mhyphen,\mhyphen, \Refuter))$). 
Este estado no es el estado $\ErrorSt$, y el Verificador no posee una estrategia sure-ganadora desde el mismo, ya que $(t,t') \notin \M$.  Por lo tanto,  desde 
$(t, \mhyphen,t', \mhyphen,\mhyphen,\mhyphen, \Refuter)$, el Refutador siempre posee una forma de jugar de tal manera que la probabilidad de alcanzar el estado de error sea mayor que $0$.  Entonces, 
$(s, \mhyphen, s', \mu, \mu', w, \Probabilistic) \notin \strat{\Verifier}(V^{\StochG}_\Verifier)$ lo cual contradice nuestro supuesto inicial. 
Por lo tanto, $\mu \MaskCoup \mu'$.

Ahora vamos a probar que $\M$ es una simulación de masking probabilista. Demostramos esto para cualquier $(s,\mhyphen,s',\mhyphen, \mhyphen,  \mhyphen, \Refuter)$ alcanzable desde el estado inicial que satisfaga $s \M s'$, valen las condiciones (b) de la Def.~\ref{def:masking_rel_prob}.  Para la condición (b)(1), sea $(s,\mhyphen,s',\mhyphen, \mhyphen,  \mhyphen, \Refuter) \in V^{\StochG}_\Refuter$ tal que 
$s \M s'$ y $s \xrightarrow{\sigma} \mu$ (resp. $s \xrightarrowprime{\sigma} \mu'$). Además, consideremos el estado correspondiente 
$(s,\sigma^1,s', \mu, \mhyphen, \mhyphen, \Verifier)$ (resp.  $(s,\sigma^2,s',  \mhyphen, \mu', \mhyphen, \Verifier)$),  como tenemos que $s \M s'$, existe una estrategia sure-ganadora para este vértice, por lo tanto existe algún 
$(s,\mhyphen,s', \mu,  \mu', w,\Probabilistic) \in \post((s,\mhyphen,s', \mu, \mhyphen, \mhyphen, \Verifier))$, y por la propiedad demostrada arriba tenemos que 
$\mu \MaskCoup \mu'$. Los casos (b)(2) y (b)(3) son similares.  Ahora, como esto se cumple para todos los estados alcanzables, y 
$\Verifier$ tiene una estrategia ganadora desde $(s_0, \mhyphen, s'_0, \mhyphen, \mhyphen,  \mhyphen, \Refuter)$, en particular, tenemos que $s_0 \M s'_0$. 
Por lo tanto, todos los requisitos de la Def.~\ref{def:masking_rel_prob} valen, y entonces $\M$ es una relación de masking probabilista. 
\end{proof} \\



% BE AWARE OF THE HACKING FOR WRAPFIGURE TO WORK PROPERLY
%\begin{example}
\medskip\par\noindent\textit{Ejemplo~\refstepcounter{example}\theexample.}
%
  Consideremos el grafo de la Figura~\ref{fig:masking:game:graph}. 
  Este representa un fragmento de grafo de juego de masking entre
  \texttt{NOMINAL} y \texttt{FAULTY} del  Ejemplo~\ref{example:memory}.
  Los vértices representan los valores de las variables en el siguiente orden
  $((\texttt{b},\texttt{m}),\_,(\texttt{v},\texttt{s},\textcolor{red}{\texttt{f}}),\_,\_,\_,\_)$.
  Las distribuciones allí son las siguientes:\vspace{-2ex}%
  
  {\scriptsize
  \begin{align*}
    \mu   &= \texttt{p}\cdot(0,1)+(1{-}\texttt{p})\cdot(0,0)\\
    \mu'  &= \texttt{p}\cdot(0,2,\textcolor{red}{0})+\texttt{q}\cdot(0,1,\textcolor{red}{0})+(1{-}\texttt{p}{-}\texttt{q})\cdot(0,0,\textcolor{red}{0})\\
    \mu'' &= \texttt{p}\cdot(1,2,\textcolor{red}{1})+\texttt{q}\cdot(1,1,\textcolor{red}{1})+(1{-}\texttt{p}{-}\texttt{q})\cdot(1,0,\textcolor{red}{1})\\
    w_0   &= \begin{cases}\texttt{p}\cdot((0,1),(0,2,\textcolor{red}{0}))+\texttt{q}\cdot((0,0),(0,1,\textcolor{red}{0}))+{}\\(1{-}\texttt{p}{-}\texttt{q})\cdot((0,0),(0,0,\textcolor{red}{0}))\end{cases}\\
    w_1   &= \Dirac_{((0,0),(1,0,\textcolor{red}{1}))}\\
    w_2   &= \begin{cases}\texttt{p}\cdot((0,1),(1,2,\textcolor{red}{1}))+\texttt{q}\cdot((0,0),(1,1,\textcolor{red}{1}))+{}\\(1{-}\texttt{p}{-}\texttt{q})\cdot((0,0),(1,0,\textcolor{red}{1}))\end{cases}\\
    w_3   &= \Dirac_{((0,0),(2,0))}
  \end{align*}
  }%
  %% $\mu   = \texttt{p}\,{\cdot}\,(0,1)+(1{-}\texttt{p})\,{\cdot}\,(0,0)$,
  %% $\mu'  = \texttt{p}\,{\cdot}\,(0,2,\textcolor{red}{0})+\texttt{q}\,{\cdot}\,(0,1,\textcolor{red}{0})+(1{-}\texttt{p}{-}\texttt{q})\,{\cdot}\,(0,0,\textcolor{red}{0})$,
  %% $\mu'' = \texttt{p}\,{\cdot}\,(1,2,\textcolor{red}{0})+\texttt{q}\,{\cdot}\,(1,1,\textcolor{red}{0})+(1{-}\texttt{p}{-}\texttt{q})\,{\cdot}\,(1,0,\textcolor{red}{0})$,
  %% $w_0   = \texttt{p}\,{\cdot}\,((0,1),(0,2,\textcolor{red}{0}))+\texttt{q}\,{\cdot}\,((0,0),(0,1,\textcolor{red}{0}))+(1{-}\texttt{p}{-}\texttt{q})\,{\cdot}\,((0,0),(0,0,\textcolor{red}{0}))$,
  %% $w_1   = \Dirac_{((0,0),(1,0,\textcolor{red}{1}))}$,
  %% $w_2   = \texttt{p}\,{\cdot}\,((0,1),(1,2,\textcolor{red}{1}))+\texttt{q}\,{\cdot}\,((0,0),(1,1,\textcolor{red}{1}))+(1{-}\texttt{p}{-}\texttt{q})\,{\cdot}\,((0,0),(1,0,\textcolor{red}{1}))$, and
  %% $w_3   = \Dirac_{((0,0),(2,0))}$.

  \noindent%
  Primero consideremos el grafo ignorando los números en color rojo. Por ejemplo, el vértice inicial 
  $\left((0,0),\mhyphen,(0,0,\textcolor{red}{0}),\mhyphen,\mhyphen,\mhyphen,\Refuter\right)$
  debería ser considerado como
  $\left((0,0),\mhyphen,(0,0),\mhyphen,\mhyphen,\mhyphen,\Refuter\right)$.
  En este caso obtenemos el grafo de juego de masking cuando la parte roja en
  \texttt{FAULTY} está ausente.
  %
  Observemos que en la mayoría de los vértices, se omiten muchos arcos salientes. En particular, el vértice del Verificador $\left((0,0),\texttt{tick},(0,0),\mu,\mhyphen,\mhyphen,\Verifier\right)$
  (recordemos que estamos ignorando el dígito rojo) tiene infinitos arcos salientes que llevan a estados probabilistas de la forma
  $\left((0,0),\texttt{tick},(0,0),\mu,\mu',w,\Probabilistic\right)$,
  donde $w$ es un coupling para $(\mu,\mu')$.
  %
  En el grafo, elegimos distinguir el coupling $w_0$, el cual es óptimo para el Verificador (similarmente luego para $w_2$).
  %
  El grafo destaca el camino que lleva al estado de error $\ErrorSt$.
  Notemos que esto ocurre como consecuencia de la elección del Refutador de ejecutar una segunda falla en el vértice
  $\left((0,0),\mhyphen,(1,1),\mhyphen,\mhyphen,\mhyphen,\Refuter\right)$
  llevando el juego a la parte del grafo remarcada en rojo.  Luego, el Refutador escoge leer el valor 0 en el modelo \texttt{NOMINAL}  (en el vértice
  $\left((0,0),\mhyphen,(2,0),\mhyphen,\mhyphen,\mhyphen,\Refuter\right)$),
  movimiento que el Verificador es incapaz de emparejar.
  

  Ahora bien, consideremos el grafo de juego de masking entre \texttt{NOMINAL} y
  el modelo con fallas limitadas \texttt{FAULTY} (i.e., el modelo incluyendo la parte en rojo).  Este grafo incluye los valores en rojo correspondientes a la variable $\textcolor{red}{\texttt{f}}$.  Observemos aqui que, el Refutador no es capaz de producir una transición de \texttt{fault} desde el vértice
  $\left((0,0),\mhyphen,(1,1,\textcolor{red}{\texttt{1}}),\mhyphen,\mhyphen,\mhyphen,\Refuter\right)$.
  Por lo tanto el Verificador logra evitar alcanzar el estado de error
  $\ErrorSt$.
%\end{example}
\medskip\par
%%%%%



Observemos que el grafo para el juego de masking estocástico podria ser infinito ya que cada vértice probabilista incluye un coupling entre dos distribuciones, y puede haber una cantidad innumerable de ellos.
%The stochastic masking game, though infinite, induces an algorithm as
Aun así, induce el algoritmo a continuación.
%
%% For $i\in \mathbb{N}$, we define regions $W^i$ of the graph vertices.
%% Intuitively, each $W^i$ represents a collection of vertices from which
%% the Refuter has a strategy (in the infinite game) with probability
%% greater than $0$ of reaching the error state in at most $i$ steps
%% (these sets can be thought of as a probabilistic version of
%% \emph{attractors} \cite{Jurd11}).
Para $i\in \mathbb{N}$, definimos regiones $W^i$ que contienen la colección de vértices desde los cuales el Refutador tiene una estrategia (en el juego infinito) con probabilidad mayor a $0$ de alcanzar el estado de error en a lo sumo $i$ pasos (estos conjuntos se pueden pensar como una versión probabilista de los \emph{atractores} \cite{Jurd11}).


\begin{definition}\label{def:W}
  Sea
  $\StochG_{A,A'} = (V^\StochG, E^\StochG, V^\StochG_\Refuter, V^\StochG_\Verifier, V^\StochG_\Probabilistic, \InitVertex, \delta^\StochG)$
  un grafo de juego de masking estocástico para los PTSs $A$ y $A'$.
  Definimos los conjuntos $W^i$ (para $i \geq 0$) de la siguiente manera:
  {\footnotesize%
  \begin{align*}
 %   W^0 = {} & \emptyset,  \\
    W^0= {} & \{\ErrorSt\}, \\
    W^{i+1} = {}
    & \textstyle
      \{v' \mid v' \in V^\StochG_\Refuter \wedge \post(v') \cap W^i \neq \emptyset \} \cup {} \\
    & \textstyle
      \{v' \mid {v' \in V^\StochG_\Verifier} \wedge  {\vertices{\post(v')} \subseteq  \bigcup_{j \leq i}W^{j}} \wedge {\vertices{\post(v')} \cap W^i \neq \emptyset} \} \cup {} \\
    & \textstyle
      \{v' \mid v' \in V^\StochG_\Probabilistic \wedge \sum_{v'' \in \post(v') \cap W^i} \delta^\StochG(v')(v'') > 0 \}
  \end{align*}
  }%
  %
  donde
  $\vertices{V'}= \{ (s, \mhyphen, s', \mu, \mu', w, \Probabilistic) \in V' \cap V^\StochG_\Probabilistic \mid w \in \vertices{\couplings{\mu}{\mu'}}\}$, y
  $W = \bigcup_{i  \geq 0} W^i$.
\end{definition}
%
Los conjuntos $W^i$ pueden ser utilizados para resolver el juego $\StochG_{A,A'}$.
Observemos, en particular, que no tenemos en cuenta todos los couplings posibles, solo aquellos que son vértices del politopo
$\couplings{\pr{3}{v}}{\pr{4}{v}}$. ($\pr{i}{(x_0,\dots,x_n)}$ es la $(i+1)$-ésima proyección, i.e., \ $x_i$.)
%% This makes the set set of graph
%% vertices finite and hence $W$ can be computed with a fix point
%% algorithm.
%
Esto es suficiente para determinar el ganador del juego ya que todos los couplings en $\couplings{\pr{3}{v}}{\pr{4}{v}}$ pueden ser expresados como una combinación convexa de sus vértices.  Por lo tanto, si existe una probabilidad positiva de alcanzar el estado de error con algún coupling, entonces también existe una probabilidad positiva de alcanzarlo a través del coupling de un vértice del politopo.
%
Al tomar solo los couplings vértices, solo una cantidad finita de vértices del grafo son compilados en cada $W_i$ y entonces $W$ puede ser computado efectivamente a través de un algoritmo de punto fijo.

El siguiente resultado es una adaptación directa de los resultados para juegos de alcanzabilidad sobre grafos finitos \cite{ChatterjeeH12}.
	
\begin{theorem}\label{th:strat-W} 
  Sea
  $\StochG_{A,A'} = (V^\StochG, E^\StochG, V^\StochG_\Refuter, V^\StochG_\Verifier, V^\StochG_\Probabilistic, \InitVertex, \delta^\StochG)$
  un grafo de juego de masking estocástico para los PTSs $A$ y $A'$.  Entonces, el
  Verificador posee una estrategia sure-ganadora (o almost-sure-ganadora) desde el vértice $v$ si y solo si
  $v \notin W$.
\end{theorem}

%\textbf{Proof of Theorem \ref{th:strat-W}.}
%Let $\StochG_{A,A'} = (V^\StochG, E^\StochG, V^\StochG_\Refuter, V^\StochG_\Verifier, V^\StochG_\Probabilistic, \InitVertex, \delta^\StochG)$ 
%be a stochastic masking game graph for some PTSs $A$ and $A'$, we have that the Verifier has a sure (or almost-sure) winning strategy from vertex 
%$v$ iff $v \notin W$. \\
\noindent
\begin{proof} Primero, podemos definir un juego de alcanzabilidad de $2$ jugadores a partir de  $\StochG_{A,A'}$ al considerar a los nodos probabilistas como estados del Refutador, e ignorando la distribución de probabilidad, denotemos a este nuevo juego por $\mathcal{H}^\StochG_{A,A'}$. 
Está claro que una estrategia del Verificador es sure-ganadora en $\StochG_{A,A'}$ si y solo si esta estrategia es ganadora en $\mathcal{H}^\StochG_{A,A'}$. 
Entonces, la demostración se reduce a probar que los conjuntos $W^i$ determinan las estrategias ganadoras del Verificador en $\mathcal{H}^\StochG_{A, A'}$ (recordemos que solo los vértices de los politopos se tienen en cuenta para definir los conjuntos $W^i$). 

``Solo si'': Si el Verificador posee una estrategia ganadora desde el vértice $v$ veamos que $v \notin W^k$ para todo $k$ por inducción. 
Para $k=1$ es directo. Ahora bien, asumamos que la propiedad se cumple para $W^k$, sea $v$ un vértice arbitrario tal que el Verificador posee una estrategia ganadora $\strat{\Verifier}$ desde $v$.  Procedemos por casos:

Si $v$ es un nodo del Verificador, supongamos por contradicción que $v \in W^{k+1}$, entonces $\vertices{\post(v))}  \subseteq W^k$. 
Por lo tanto, por hipótesis inductiva $\strat{\Verifier}(v) \notin \vertices{\post(v)}$, es decir, $\strat{\Verifier}(v)$ es un vértice probabilista cuyo coupling no es un vértice del politopo. Además, es un nodo del Refutador en $\mathcal{H}^\StochG_{A,A'}$. Para este nodo tenemos $v' \in \post(\strat{\Verifier}(v))$ si y solo si $\pr{5}{\strat{\Verifier}(v)}(\pr{0}{v'}, \pr{2}{v'})>0$. 
Observemos que $\pr{5}{\strat{\Verifier}(v)}$ es un punto en el politopo definido por $\couplings{\pr{3}{\strat{\Verifier}(v)}}{\pr{4}{\strat{\Verifier}(v)}}$, 
como los politopos no contienen lineas polytopes do not contain lines, o bien $\pr{5}{\strat{\Verifier}(v)}$ es un vértice o existe un vértice del politopo $w'$ tal que   $w'(\pr{3}{\strat{\Verifier}(v)}, \pr{4}{\strat{\Verifier}(v)})>0$ si y solo si $\pr{5}{\strat{\Verifier}(v)}(\pr{3}{\strat{\Verifier}(v)}, \pr{4}{\strat{\Verifier}(v)})>0$. 
Por lo tanto, existe un $v'' \in \vertices{\post(v)}$ tal que $\post(v'') = \post(\strat{\Verifier}(v))$, es decir, $v'' \in W^k$ implica que $\strat{\Verifier}(v) \in W^{k}$ lo cual es una contradicción, entonces $v \notin W^{k+1}$. 

Si $v$ es un nodo del Refutador, por contradicción asumamos $v \in W^{k+1}$, es decir, existe un vértice del Verificador $v' \in \post(v)$ tal que 
$v' \in W^k$, por inducción no existe una estrategia sure-ganadora desde $v'$, pero entonces el Verificador no tiene una estrategia sure-ganadora desde $v$, ya que
el Refutador puede jugar a $v'$ desde $v$, esto es una contradicción, y el resultado se deduce.

``Si'':  Definamos una estrategia $\strat{\Verifier}$ la cual es una estrategia ganadora en $\mathcal{H}^\StochG_{A,A'}$ para cualquier nodo del Verificador $v \notin W$. 
Si $v \notin W$, entonces $\strat{\Verifier}(v) = v'$ para algún $v' \in \post(v) \cap (V^\StochG \setminus W)$ (cuya existencia está garantizada por nuestro supuesto), 
además, si $v \in W$, entonces $\strat{\Verifier}(v) = v'$ para un nodo arbitrario $v'$. 
Probemos que para cualquier jugada generada por $\strat{\Verifier}$: $v_0, v_1, \dots$ tenemos $v_i \notin W$, la prueba es por inducción sobre $i$.
Para $i=0$ esto es directo, asumiendo que $v_i \notin W$ vamos a probar que $v_{i+1} \notin W$. 
Si $v_i$ es un nodo del Refutador, por Def.~\ref{def:W}  
$\post(v_i) \cap W = \emptyset$, y entonces, $v_{i+1} \notin W$. 
Si $v_i$ es un nodo del Verificador, por definición de $\strat{\Verifier}$: $v_{i+1} = \strat{\Verifier}(v_i) \notin W$ y por lo tanto el resultado se deduce.  
\end{proof} \\


%
Es de destacar que, para un vértice probabilista
$(s, \mhyphen, s', \mu, \mu', w, \Probabilistic) \in V^\StochG_\Probabilistic$,
el politopo \textit{two-way transportation} $\couplings{\mu}{\mu'}$ tiene al menos $\frac{\max\{m,n\}!}{(\max\{m, n\}-\min\{m,n\}+1)!}$ vértices
(y a lo sumo $m^{n-1}n^{m-1}$ vértices)~\cite{KleeWitzgall}, donde
$m=|\support{\mu}|$ y $n=|\support{\mu'}|$.  Por lo tanto, no seria práctico computar estos conjuntos.





%\subsection{A symbolic game graph.}
\subsection{Un grafo de juego simbólico.}

%% Notice that the graph for a stochastic masking game could be infinite.
%% Indeed, each probabilistic node of the graph include a coupling
%% between the two contending distributions, and there can be uncountably
%% many of them.

A continuación, vamos a introducir una descripción finita de los juegos de masking estocásticos a través de una representación simbólica, la cual permite un algoritmo mas eficiente.
%
La definición del grafo de juego simbólico tiene dos partes.  La primera captura el comportamiento no estocástico del juego al remover la elección estocástica ($\delta^\StochG$) del grafo de juego, asi como los couplings en la información de los vértices. La segunda parte anexa un sistema de ecuaciones a cada vértice probabilista, cuyo espacio de soluciones es el politopo definido por el conjunto de todos los couplings entre las distribuciones contendientes.


\begin{definition} \label{def:symbolic_game_graph}
  Sean $A = ( S, \Sigma, {\rightarrow}, s_0 )$
  y $A' = ( S', \Sigma_\faults, {\rightarrow'}, s'_0 )$
  dos PTSs.
  El \emph{grafo de juego simbólico} para el juego de masking estocástico
  $\mathcal{G}_{A,A'}$ está definido por la estructura
  $\SymbG_{A,A'} = ( V^{\SymbG}, E^{\SymbG}, V^{\SymbG}_\Refuter, V^{\SymbG}_\Verifier, V^{\SymbG}_\Probabilistic, v_0^{\SymbG} )$,
  %with initial state $v_0^{\SymbG}$,  
  donde:
%  \remarkPRD{Modifiqu\'e la definici\'on de $V^\SymbG_\Probabilistic$ para que sea finito y no venga otro (o el mismo) referee cornudo a romper las bolas}
  {\footnotesize%
  \begin{align*}
    V^\SymbG = \
    & V^\SymbG_\Refuter \cup V^\SymbG_\Verifier \cup V^\SymbG_\Probabilistic, \text{ donde: }\\
    V^\SymbG_\Refuter = \
    & \{ (s, \mhyphen, s', \mhyphen, \mhyphen, \Refuter) \mid
          s \in S \wedge s' \in S' \} \cup
      \{\ErrorSt\}\\
    V^\SymbG_\Verifier = \
    & \{ (s, \sigma^1, s', \mu, \mhyphen, \Verifier) \mid
         s \in S \wedge s' \in S'
         \wedge \sigma \in \Sigma
         \wedge (s, \sigma, \mu) \in {\rightarrow} \} \cup {} \\
    & \{ (s, \sigma^2, s', \mhyphen, \mu', \Verifier) \mid
         s \in S \wedge s' \in S'
         \wedge \sigma \in \SigmaF
          \wedge (s', \sigma, \mu') \in {\rightarrow'} \} \\
    V^\SymbG_\Probabilistic = \
    & \{ (s, \mhyphen, s', \mu, \mu', \Probabilistic) \mid
         s \in S \wedge s' \in S' \wedge
    %     \mu \in \Dist(S) \wedge \mu' \in \Dist(S')\}\\
    %    \exists \sigma{\in}\SigmaF : (s\xrightarrow{\sigma}\mu \vee (\sigma{\in}\faults \wedge \mu=\Dirac_s)) \wedge s'\xrightarrowprime{\sigma}\mu'\}\\
         \exists \sigma{\in}\SigmaF : (s\xrightarrow{\sigma}\mu \vee \mu=\Dirac_s) \wedge s'\xrightarrowprime{\sigma}\mu'\}\\
    v_0^{\SymbG} = \
    & ( s_0, \mhyphen, s'_0, \mhyphen, \mhyphen, \Refuter ),
    % \text{ (the Refuter starts playing)}
  \end{align*}
  }%
%
  y $E^\SymbG$ es el conjunto mínimo que satisface las siguientes reglas:
%
  {\footnotesize%
  \begin{align*}
    s \xrightarrow{\sigma} \mu
    & \Rightarrow \tuple{(s, \mhyphen, s', \mhyphen, \mhyphen, \Refuter), (s, \sigma^{1}, s', \mu, \mhyphen, \Verifier)}\in E^\SymbG \\
    s' \xrightarrowprime{\sigma} \mu'
    & \Rightarrow \tuple{(s, \mhyphen, s', \mhyphen, \mhyphen, \Refuter),(s, \sigma^{2}, s', \mhyphen, \mu', \Verifier)}\in E^\SymbG \\
    {s' \xrightarrowprime{\sigma} \mu'}
    & \Rightarrow \tuple{(s, \sigma^1, s', \mu, \mhyphen, \Verifier),(s, \mhyphen, s', \mu, \mu', \Probabilistic)}\in E^\SymbG \\
    {\sigma \notin \faults} \wedge {s \xrightarrow{\sigma} \mu}
    & \Rightarrow \tuple{(s, \sigma^2, s', \mhyphen, \mu', \Verifier), (s, \mhyphen, s', \mu, \mu', \Probabilistic)}\in E^\SymbG\\
    {F \in \faults}
    & \Rightarrow \tuple{(s, F^2, s', \mhyphen, \mu', \Verifier), (s, \mhyphen, s', \Dirac_s, \mu', \Probabilistic)}\in E^\SymbG \\
     (s, \mhyphen, s', \mu, \mu', \Probabilistic) \in V^\SymbG_\Probabilistic \wedge {} \qquad & \\ 
   ((t,t') \in \support{\mu}\times \support{\mu'})
    & \Rightarrow \tuple{(s, \mhyphen, s', \mu, \mu', \Probabilistic), (t, \mhyphen, t', \mhyphen, \mhyphen,  \Refuter)}\in E^\SymbG \\
    & \hspace{-12.4em} {v\in (V^\SymbG_\Verifier{\cup}\{\ErrorSt\})} \wedge {(\nexists {v'\neq\ErrorSt} : {\tuple{v,v'}\in E^\SymbG})}
    \Rightarrow  \tuple{v,\ErrorSt}\in E^\SymbG
  \end{align*}
  }%
%% \begin{itemize}
%%  %   \item $\Sigma^{SG} = \Sigma$
%%   \item $V^\SymbG = (S \times ( \Sigma^1 \cup \Sigma^2 \cup\{\#\}) \times S' \times \Dist(S) \times \Dist(S') \times \{ \Refuter, \Verifier, \Probabilistic \}) 
%%   \cup \{\ErrorSt\}$
%%   \item $v_0^{\SymbG} = ( s_0, \#, s'_0, \mhyphen, \mhyphen, \Refuter )$, 
%%   \item $V_\Refuter^{\SymbG} = \{ (s, \#, s', \mhyphen, \mhyphen,  \Refuter) \mid s \in S \wedge s' \in S' \wedge \mu \in \Dist(S) \wedge \mu' \in \Dist(S')  \} 
%%   \cup \{\ErrorSt\}$,
%%   \item $V_\Verifier^{\SymbG} = \{ (s, \sigma, s', \mu, \mhyphen,  \Verifier) \mid s \in S \wedge s' \in S' \wedge \sigma \in (\Sigma^1 \cup \SigmaF^2) \wedge \mu \in \Dist(S)\} \cup \{ (s, \sigma, s', \mhyphen, \mu',  \Verifier) \mid s \in S \wedge s' \in S' \wedge \sigma \in (\Sigma^1 \cup \SigmaF^2) \wedge \mu' \in \Dist(S')\}$,
%%   \item $V_\Probabilistic^{\SymbG} = \{ (s, \#, s', \mu, \mu', \Probabilistic) \mid s \in S \wedge s' \in S' \wedge \mu \in \Dist(S) \wedge \mu' \in \Dist(S')\}$
%% \end{itemize}
%% and $E^\SymbG$ is the minimal set satisfying:
%% \begin{itemize}
%%   \item $\{ ((s, \#, s', \mhyphen, \mhyphen,  \Refuter),(s, \sigma^{1}, s', \mu, \mhyphen,  \Verifier)) \mid \exists\;\sigma \in \Sigma: s \xrightarrow{\sigma} \mu\} \subseteq E^\SymbG$,
%%
%%   \item $\{ ((s, \#, s', \mhyphen, \mhyphen, \Refuter), (s, \sigma^{2}, s', \mhyphen, \mu',  \Verifier)) \mid \exists\;\sigma \in \Sigma: s' \xrightarrowprime{\sigma} \mu' \} \subseteq E^\SymbG$,
%%
%%   \item $\{ ((s, \#, s', \mu, \mu',  \Probabilistic), (t, \#, t', \mhyphen, \mhyphen,  \Refuter)) \mid \exists w \in \couplings{\mu}{\mu'} : w(t,t')>0 \} \subseteq E^\SymbG$,
%%
%%  % \item $\{ ((s, \#, s', \mu, \mu',  \Probabilistic), (t, \sigma^{2}, t', \mhyphen, \nu',  \Verifier)) \mid (\exists\;\sigma \in \Sigma: t' \xrightarrowprime{\sigma} \nu')  \} \subseteq E^\mathcal{SG}$,
%%   \item $\{ ((s, \sigma^2, s', \mhyphen, \mu',  \Verifier), (s, \#, s', \mu, \mu',  \Probabilistic)) \mid (\exists\;\sigma \in \Sigma: s' \xrightarrow{\sigma} \mu) \} \subseteq E^\SymbG$,
%%
%%   \item $\{ ((s, \sigma^1, s', \mu, \mhyphen, \Verifier), (s, \#, s', \mu, \mu',  \Probabilistic)) \mid (\exists\;\sigma \in \Sigma: s' \xrightarrowprime{\sigma} \mu') \} \subseteq E^\SymbG$.
%%   \item  $(\ErrorSt, \ErrorSt) \in E^\SymbG$
%%   \item  for those vertices $v$ such that $\post(v)= \emptyset$,  transitions $(v, \ErrorSt)$  are added 
%%   to $E^\SymbG$.
%% \end{itemize}
%% \end{definition}	
%%
%% Intuitively, a symbolic graph abstracts away the probabilistic part of a game, leaving the underlying graph of the game.
%% Furthermore, we label every vertex in $V_\Probabilistic^{\SymbG}$ with a system of equations:
%% \begin{definition} Let $\SymbG_{A,A'} = \langle V^{\SymbG},  E^{\SymbG}, V^{
%%     \SymbG}_\Refuter, V^{\SymbG}_\Verifier, V^{\SymbG}_\Probabilistic, v_0^{\SymbG} \rangle$
%%       be a symbolic graph game.
  Adicionalmente, para cada
  $v=(s, \mhyphen, s', \mu, \mu', \Probabilistic) \in V^\SymbG_\Probabilistic$,
  se define el conjunto de variables 
  $X(v)=\{x_{s_i,s_j} \mid s_i \in \text{Supp}(\mu) \wedge s_j \in \text{Supp}(\mu')\}$,
  y el sistema de ecuaciones 
  %
  {\footnotesize%
  \begin{align*}
    \Eq(v) = {}
    & \textstyle
    \big\{ \sum_{s_j \in \support{\mu'}} x_{s_k,s_j}=\mu(s_k) \mid s_k \in \support{\mu} \big\} \cup {} \\
    & \textstyle
    \big\{ \sum_{s_k \in \support{\mu}} x_{s_k,s_j}=\mu'(s_j) \mid s_j \in \support{\mu'} \big\} \cup {} \\
    & \textstyle
    \big\{ x_{s_k,s_j} \geq 0 \mid s_k \in \support{\mu} \wedge s_j \in \support{\mu'} \big\}
  \end{align*}
  }%
%	$\text{Eq}(v)$ is the following $(|\text{Supp}(\mu)|+|\text{Supp}(\mu')|)$-dimensional vector of equations: 
%	\begin{equation*}
%		\text{Eq}(v)_{k} = \begin{cases*}
%      							  {\displaystyle (\sum^{m-1}_{j=0} x_{s_k,s_j})=\mu(s_k)} 		      & if $k<n$, \\
%     							  {\displaystyle (\sum^{n-1}_{i=0} x_{s_i,s_{k-n}})=\mu'(s_{k-n}) }	      & otherwise.
%   						      \end{cases*}
%	\end{equation*}
%Where  $n=|\text{Supp}(\mu)|$ and $m=|\text{Supp}(\mu')|$
\end{definition} 


Observemos que $\{\bar{x}_{s_k,s_j}\}_{s_k,s_j}$ es una solución de 
$\Eq(v)$ si y solo si existe un coupling
$w\in\couplings{\mu}{\mu'}$ tal que  $w(s_k,s_j)=\bar{x}_{s_k,s_j}$
para todo $s_k \in \support{\mu}$ y $s_j \in \support{\mu'}$.

Adicionalmente, dado un conjunto de vértices de juego
$V' \subseteq V^\SymbG_\Refuter$,
definimos $\Eq(v, V')$ al extender $\Eq(v)$ con una ecuación que limita los couplings de tal forma que los vértices en $V'$ \emph{no} son alcanzados.
Formally,
%
$\Eq(v, V') = \Eq(v) \cup \big\{\sum_{(s, \mhyphen, s', \mhyphen, \mhyphen, \Refuter) \in V'} x_{s,s'} = 0\big\}$.
%% Formally:
%% %
%% \[ \textstyle
%% \Eq(v, V') = \Eq(v) \cup \big\{\sum_{(s, \mhyphen, s', \mhyphen, \mhyphen, \Refuter) \in V'} x_{s,s'} = 0\big\}.
%% \]
%
Al definir apropiadamente una familia de conjuntos $V'$, mostraremos que la simulación de masking probabilista puede ser verificada en tiempo polinomial a través del grafo de juego simbólico.



A continuación proponemos utilizar el grafo de juego simbólico para resolver el juego infinito. De esta forma, obtenemos un procedimiento de tiempo polinomial.
Similarmente a la Definición~\ref{def:W}, proporcionamos una construcción inductiva de los vértices ganadores para el Refutador utilizando sistemas de ecuaciones en lugar de conjuntos de vértices del politopo, de la siguiente manera.

\begin{definition}\label{def:U_prob}
  Sea
  $\SymbG_{A,A'} = ( V^{\SymbG},  E^{\SymbG}, V^{\SymbG}_\Refuter, V^{\SymbG}_\Verifier, V^{\SymbG}_\Probabilistic, v_0^{\SymbG} )$
  un grafo de juego simbólico para los PTSs $A$ y $A'$.
  Los conjuntos $U^i$ (para $i \geq 0$) se definen de la siguiente manera:
  {\footnotesize%
  \begin{align*}
    %U^0 = {} & \emptyset,  \label{def:U_probji} \\
    U^0 = {} & \{\ErrorSt\},  \label{def:U_probji}\\
    U^{i+1} = {}
    & \textstyle 
    \{v' \mid v' \in V^\SymbG_\Refuter \wedge \post(v') \cap U^{i} \neq \emptyset \} \cup {}\\
    & \textstyle 
    \{v' \mid v' \in V^\SymbG_\Verifier \wedge \post(v') \subseteq \bigcup_{j\leq i}U^{j} \wedge \post(v') \cap U^i \neq \emptyset  \}  \cup {}\\
    & \textstyle 
    \{v' \mid v' \in V^\SymbG_\Probabilistic \wedge \post(v') \cap U^{i} \neq \emptyset \wedge \Eq(v', \post(v') \cap U^{i}) \text{ no tiene solución}\}
  \end{align*}
  }%
  Además, definimos $U = \bigcup_{i \geq 0} U^i$.
\end{definition}
%

La construcción de cada $U^{i+1}$ sigue una idea similar a la construcción de $W^{i+1}$, solo variando significativamente en el caso de los vértices probabilistas.  La primer y segunda linea corresponden a los jugadores Refutador y Verificador, respectivamente. La última linea corresponde al jugador probabilista.
%
Observemos que, si $\Eq(v', \post(v') \cap U^{i})$ no tiene solución, significa que todo coupling posible inevitablemente lleva, con cierta probabilidad, a un estado ``perdedor'' de nivel menor, ya que, en
particular, la ecuación
$\sum_{(s, \mhyphen, s', \mhyphen, \mhyphen, \Refuter) \in (\post(v') {\cap} U^{i})} x_{s,s'} = 0$
no puede ser satisfecha.

Existe una fuerte conexión entre los conjuntos  $W^i$ y $U^i$: un vértice está en $W^i$ si y solo si su versión abstracta está en $U^i$.  Esto se formaliza en el siguiente teorema.
	
\begin{theorem}\label{th:U-and-W}
  Sea $\mathcal{G}_{A,A'}$ un grafo de juego de masking estocástico
  para los PTSs $A$ y $A'$ y sea $\SymbG_{A,A'}$
  el grafo de juego simbólico correspondiente.
%
  Para cada $v \in V^\StochG$, $u \in V^\SymbG$ tal que 
  $\pr{i}{v} = \pr{i}{u}$ (para $0 \leq i \leq 4$) y
  $\pr{6}{v} = \pr{5}{u}$, y para todo $k\geq 0$,
  $v \in U^k$ si y solo si $u \in W^k$.
\end{theorem}

El siguiente teorema es una consecuencia directa de los Teoremas
\ref{th:strat-W} y \ref{th:U-and-W}.

\begin{theorem}
  Sea $\StochG_{A,A'}$ un grafo de juego estocástico para los PTSs $A$
  y $A'$, y sea $\SymbG_{A,A'}$ el grafo de juego simbólico correspondiente.  Entonces, el Verificador posee una estrategia sure-ganadora (o almost-sure-ganadora) en
  $\StochG_{A,A'}$ si y solo si $v^\SymbG_0 \notin U$.
\end{theorem}

\noindent
%\textbf{Proof of Theorem \ref{th:U-and-W}.}
%Given a stochastic masking game graph $\mathcal{G}_{A,A'}$ for some PTSs $A$ and $A'$ and the corresponding symbolic game $\SymbG_{A,A'}$. 
%For any states $v \in V^\StochG$, $u \in V^\SymbG$ such that $\pr{i}{v} = \pr{i}{u}$ (for $0 \leq i \leq 4$) and for any $k>0$ 
%we have that: $u \in U^k$ iff $v \in W^k$. \\
\noindent
\begin{proof}  
La prueba es por inducción sobre $k$. Para k=0, tenemos que $W^0 = \{ \ErrorSt \} = U^0$. Para el caso inductivo, consideremos los nodos arbitrarios $u \in V^\SymbG$ y $v \in V^{G}$,
tales que $\pr{i}{v} = \pr{i}{u}$ para $0 \leq i \leq 4$. Observemos que estos nodos también coinciden en sus últimas componentes, es decir, ambos son nodos del Refutador, del Verificador o probabilistas. 
Asumamos que son del Refutador, si $u \in U^{k}$ entonces $\post(u) \cap U^{k-1} \neq \emptyset$. 
Por lo tanto, existe algún $u' \in \post(u) \cap U^{k-1}$ que es un nodo del Verificador.  Por Def.~\ref{def:symbolic_game_graph}, tenemos un $v' \in \post(v)$ tal que $\pr{i}{v'}=\pr{i}{u'}$ (para $0\leq i \leq 4$), por inducción tenemos que $v' \in W^{k-1}$ 
y entonces $v \in W^k$. 
De manera similar, si $v \in W^k$ tenemos que $\post(v) \cap W^{k-1} \neq \emptyset$, y procedemos como se hizo anteriormente. 
Si $v$ y $u$ son nodos del Verificador la prueba es similar.
Ahora bien, asumamos que $v$ y $u$ son nodos probabilistas. 
Si $u \in U^k$, entonces $\post(u) \cap U^{k-1} \neq \emptyset$. Como se probó mas arriba, también tenemos que  $\post(v) \cap W^{k-1} \neq \emptyset$. 
Además, si $\Eq(v, \post(u) \cap U^{k-1})$ no tiene solución, entonces tenemos al menos un coupling (llamemosle $w$) para las distribuciones
$\pr{3}{u}$ y $\pr{4}{u}$ que satisface las ecuaciones y también $w(\pr{0}{u'},\pr{2}{u'})>0$ para algún $u' \in \post(u) \cap U^{k-1}$. 
Por Def.~\ref{def:strong_masking_game_graphi}, tenemos un vértice $v' \in \post(v)$ tal que $\pr{5}{v'}=w$, y por lo tanto $\sum_{v' \in \post(v) \cap W^{j-1}} \delta^\StochG(v)(v') >0$, lo que significa que $v' \in W^k$. 
Similarmente, si $\sum_{v' \in \post(v) \cap W^{j-1}} \delta^\StochG(v)(v') >0$  entonces $\Eq(u, \post(u) \cap U^{k-1})$ no tiene solución, y entonces
$v\in W^k$ implica $u \in U^k$.
\end{proof} \\
%% As a consequence of this theorem and Theorem~\ref{thm:wingame_strat_prob},
%% in order to decide if there is a probabilistic masking simulation
%% between two PTSs $A$ and $A'$, we can calculate set $U$ over
%% $\SymbG_{A,A'}$, and check whether its initial state belongs to $U$.
Como consecuencia de este último teorema y el Teorema~\ref{thm:wingame_strat_prob}, basta con calcular el conjunto $U$
sobre $\SymbG_{A,A'}$ para decidir si existe una simulación de masking probabilista
entre $A$ y $A'$.
%
Esto se puede hacer en tiempo polinomial, ya que $\Eq(v,C)$ puede ser resuelto en tiempo polinomial (e.g, utilizando programación lineal) y el número de iteraciones para construir $U$ está acotado por $|V^\SymbG|$.  Como $V^\SymbG$ depende linearmente de las transiciones de los PTSs involucrados, tenemos el siguiente teorema.

\begin{theorem}
  Sean $A$ y $A'$ dos PTSs.  $A \Masking A'$ se puede decidir en tiempo 
  $O(\textit{Poly}(m\cdot m'))$, donde $m$ y $m'$ son 
  los tamaños de las transiciones de $A$ y $A'$, respectivamente.
  %% Let $A$ and $A'$ be PTSs.  $A \Masking A'$ can be decided in time
  %% $O((n*n')^4)$, where $n$ is the number of states of $A$ and $n'$ is
  %% the number of states of $A'$.
\end{theorem} 







\section{Cuantificando la Tolerancia a Fallas} \label{sec:almost_sure_prob}
%\section{Quantifying Fault Tolerance in casi-seguro terminante Systems} \label{sec:prob_almost_sure}
%\section{Quantifying Fault Tolerance in casi-seguro terminante Systems under fairness} \label{sec:prob_almost_sure}

La simulación de enmascaramiento probabilista determina si una implementación tolerante a fallas es capaz de enmascarar las fallas completamente, o no. Sin embargo, en la práctica, este tipo de tolerancia a fallas enmascarante no es común. Usualmente, los sistemas tolerantes a fallas son capaces de enmascarar una cantidad de fallas determinada antes de alcanzar un estado de error. Es decir, se necesita una cuantificar la tolerancia a fallas, para poder estimar la utilidad de cada mecanismo de tolerancia a fallas, y poder comparar diferentes técnicas entre sí, para poder elegir la que provea una mejor tolerancia a fallos.
%
%% Our main goal in this section is to extend the
%% game theory presented in the previous section to be able to measure
%% the amount of masking tolerance exhibited by a system before it fails.
%
En esta sección extendemos la teoría de juegos presentada anteriormente para proporcionar una medida para la cantidad de fallas que un sistema puede enmascarar antes de entrar en un estado de error.
%
Para hacer esto, extendemos el juego de enmascaramiento estocástico con una función de objetivo cuantitativo.
El valor esperado de esta función indica la cantidad de ``hitos'' que se espera que la implementación tolerante a fallas concrete antes de entrar en un estado de error.  Un hito es cualquier evento de interés que pueda ocurrir durante la ejecución de un sistema.  Por ejemplo, se puede considerar hito al enmascaramiento de una falla.
En este caso, la medida reflejará la cantidad de fallas que fueron toleradas por el sistema antes de romperse. Otro hito podría ser un \textit{acknowledgement} exitoso en un protocolo de transmisión. Esto mide el número esperado de mensajes que el protocolo es capaz de transferir antes de entrar a un estado de falla.

%
%% To do this type of measuring,  we need  stochastic games with 
%% quantitative objective functions.  Intuitively, these objective functions count the
%% number of ``milestones'' observed during a play.
%% %
%% Therefore, we first extend stochastic masking games as follows.
%
%% \begin{definition}%
%%   Let
%%   $A =( S, \Sigma, \rightarrow, s_0 )$ and $A' =( S', \SigmaF, \rightarrow', s_0' )$
%%   be two PTSs.
%% %
%%   A \emph{stochastic masking game graph with milestones} $\milestones$
%%   is a tuple
%%   $\MilestoneG_{A,A'} = (V^\StochG, E^\StochG, V^\StochG_\Refuter, V^\StochG_\Verifier, V^\StochG_\Probabilistic, \InitVertex, \delta^\StochG,  \milestones)$
%%   where:
%%   \begin{inparaenum}[(i)]
%%   \item%
%%     $(V^\StochG, E^\StochG, V^\StochG_\Refuter, V^\StochG_\Verifier, V^\StochG_\Probabilistic, \InitVertex, \delta^\StochG)$  is a stochastic masking game graph,
%%   \item%
%%     $\milestones \subseteq \SigmaF^2$ is the set of \emph{milestones}, and
%%   \item%
%%     $\reward^\StochG(v) = \chi_{\milestones}(\pr{1}{v})$ is a \emph{reward function}.
%%   \end{inparaenum}
%% \end{definition}
%% %
%% In this definition, $\chi_B$ is the characteristic function over
%% set $B$ defined as usual:
%% %$\chi_B(a) = {(a{\in}B)}\mathbin{?}1\mathbin{:}0$.
%% $\chi_B(a) = 1$ whenever $(a{\in}B)$ and $\chi_B(a) = 0$ otherwise.
%% If $B=\{b\}$ is a singleton set, we simply write $\chi_b$.
%
%% Given a stochastic masking game graph with milestones and reward function $r^\StochG$, for any play 
%% $\rho = \rho_0, \rho_1,  \dots$,
%% we define the \emph{masking payoff function} by
%% %
%% $\FMask(\rho) = \lim_{n \rightarrow \infty} (\sum^{n}_{i=0} \reward^\StochG(\rho_i))$.
%% %% we define the \emph{masking payoff function} as follows:
%% %% \[
%% %% 	\FMask(\rho) = \lim_{n \rightarrow \infty} (\sum^{n}_{i=0} \reward^\StochG(\rho_i)).
%% %% \]
%
Por lo tanto, un hito es simplemente una etiqueta de una acción designada en el modelo de implementación.


\begin{definition}%
  Sea $A' =( S', \SigmaF, \rightarrow', s_0' )$ un PTS que modela una implementación.
  Un \emph{conjunto de hitos} es un conjunto $\milestones\subseteq\SigmaF$.
\end{definition}

%\remarkPRD{Este cambio en la definici\'on de la recompensa soluciona directamente el error se\~nalado por Luciano en el Teorema 6, solo que habria que cambiar alli a $\chi_{\milestones}$ por $\mreward^\StochG$ (ya lo hice).  Adem\'as habr\'ia que revisar el apendice, por este cambio, pero tambi\'en porque desaparecio el stochastic game graph with milestones $\MilestoneG_{A,A'}$}
Dado un conjunto de hitos $\milestones$ para $A'$, definimos la recompensa
$\mreward^\StochG$ sobre el juego de enmascaramiento estocástico
$\StochG_{A,A'} = (V^\StochG, E^\StochG, V^\StochG_\Refuter, V^\StochG_\Verifier, V^\StochG_\Probabilistic, \InitVertex, \delta^\StochG)$ para los PTS $A$ y $A'$ de la siguiente manera:
%
$\mreward^\StochG(v) = 1$ si $v\in V^\StochG_\Verifier$ y
$\pr{1}{v}\in(\milestones^1\cup\milestones^2)$; en caso contrario,
$\mreward^\StochG(v) = 0$.
%
$\mreward^\StochG$ recolecta hitos (cuando puede) solo una vez por cada ronda del juego
. Esto solo se puede hacer en los vértices del Verificador ya que solo en estos se almacena la etiqueta que se esta jugando en la ronda.
%
Luego, la \emph{función de payoff de enmascaramiento} está definida por
%
$\FMask(\rho) = \lim_{n \rightarrow \infty} (\sum^{n}_{i=0} \mreward^\StochG(\rho_i))$.
%
Intuitivamente, la función de \emph{payoff} $\FMask$ caracteriza la cantidad de hitos que una implementación tolerante a fallas es capaz de lograr hasta que se alcanza el estado de error. Este tipo de funciones de \emph{payoff} se suelen llamar \emph{recompensas totales} en la literatura.
Esto se puede pensar como un juego entre el mecanismo de tolerancia a fallas y un jugador (malicioso) que escoge la manera en la que las fallas ocurren. En este juego, el Verificador es quien maximiza (pretende alcanzar tanto hitos como sea posible) y el Refutador es quien minimiza (quiere prevenir al Verificador de coleccionar recompensas por hitos).  

    

% Desde aca incluimos las cosas de fairness
Para que el juego de recompensa total esperada esté determinado, el juego estocástico tiene que ser casi-seguro terminante (\textit{almost-surely stopping} en Ingles), es decir, el juego debe alcanzar un vértice terminal con probabilidad
1~\cite{FilarV96}.  En el Capítulo~\ref{cap:fairAdversaries} se ha extendido la propiedad de determinación a juegos que son casi-seguro terminantes (o terminantes) bajo la condición de que el minimizador juega de manera \emph{fair}.
%
%% In our setting, this amounts to considering \emph{casi-seguro terminante
%% masking games}, that is, games in which the error state $\ErrorSt$ is
%% reached with probability 1.  Moreover, we require that the Refuter
%% plays fair.  This is necessary to prevent the Refuter from stalling the game
%% in an unproductive loop.
En nuestro contexto, esto nos lleva a considerar \emph{juegos de enmascaramiento que fallan casi-seguramente bajo fairness}, es decir, juegos en los que el estado de error
$\ErrorSt$ es alcanzado con probabilidad 1 dado que el Refutador juega de forma \emph{fair}.
%
El supuesto de \emph{fairness} es necesario para prevenir que el Refutador frene el progreso del juego en un bucle improductivo.
%
En efecto, consideremos el escenario descrito en el Ejemplo~\ref{example:memory}, y el juego de enmascaramiento estocástico entre los modelos nominal y con fallas de las Figuras~\ref{fig:exam_1_mem_cell:nom} y~\ref{fig:exam_1_mem_cell:ft} (omitiendo la parte en rojo).
Uno esperaría que el juego lleve a un estado de falla con probabilidad $1$.
%`
Sin embargo, el Refutador posee estrategias para las cuales la probabilidad de alcanzar $\ErrorSt$ es menor a $1$.  Por ejemplo, el Refutador podría siempre jugar con una acción de lectura, y por lo tanto, el Verificador tiene que imitar esta acción para siempre y entonces la probabilidad de alcanzar el estado de error es~$0$.
%
Observemos que, en este escenario, el Refutador se está comportando de forma benevolente, jugando de tal forma que se esquiva el estado de error.
Claramente, esto va en contra de la intención del comportamiento de las fallas, las cuales uno esperaría que ocurran luego de esperar lo suficiente.
%
Por lo tanto, la suposición de que el Refutador juega de forma (fuertemente) \emph{fair}, es decir, 
si alguna acción o falla está habilitada con infinita frecuencia para el Refutador, entonces este está obligado a jugarla eventualmente.
%it will eventually play such action or fault.

El contexto para los juegos estocásticos con función de \emph{payoff} de enmascaramiento como objetivo que presentaremos ahora se basa en el Capítulo~\ref{cap:fairAdversaries}, solo que con la necesidad de un cuidado especial debido a la naturaleza infinita de nuestros grafos de juego estocásticos.
%
Por esta razón limitamos los resultados del resto de la sección a estrategias sin memoria (aleatorias) y posponemos el resultado general como trabajo futuro.
%
Por lo tanto, vamos usar $\Strategies{\Verifier}^{\Memoryless}$ y
$\Strategies{\Refuter}^{\Memoryless}$ para denotar los conjuntos de todas las estrategias sin memoria (aleatorias) para el Verificador y el Refutador,
respectivamente, y de forma similar, utilizaremos
$\Strategies{\Verifier}^{\Memoryless\Deterministic}$ y
$\Strategies{\Refuter}^{\Memoryless\Deterministic}$ para denotar los conjuntos de todas las estrategias sin memoria puras (o deterministas).


%%    We are interested in systems that eventually fail, this allows us to model an interesting set of systems, and it is also mathematically convenient to ensure
%%  the well-definedness of our games.	However,  In practice, the restriction to casi-seguro terminante systems  could exclude interesting  case studies.
%% Consider, for instance,  the scenario described in  Example~\ref{example:memory}.  The fault-tolerant memory implementation described therein is not 
%% casi-seguro terminante, as the Refuter has strategies for which the probability of reaching the error state is less than $1$. 
%% For instance,  the Refuter may always play the reading action,  and hence the Verifier have to mimic this action forever, 
%% this yields a probability of $0$ of reaching the error state. 
%% Observe that,  in this scenario, the Refuter is behaving in a benevolent manner, 
%% playing in such a way that the error state is avoided. A natural assumption in this case is  that the environment (es decir, the Refuter) behaves in a fair way, in the sense that, if some action or fault is 
%%  infinitely often enabled, then it will be eventually played by this player.  fairness properties  are particularly useful  in fault-tolerance in order to adopt a more realistic model of faults \cite{DBLP:conf/icse/DIppolitoBPU11}.
%%      In this section, we extends the framework presented above to systems that almost sure fails under strong fairness assumptions.  As shown in \cite{DBLP:conf/cav/CastroDDP22}, the value of
%% stochastic games under fair strategies of one player can be computed using greatest fixpoints. We extend these results to our setting. First, consider the notion of fair plays.




Una jugada \emph{fair} del Refutador se define como una jugada en la cuál el Refutador se compromete a seguir un patrón \emph{fair} fuerte, es decir, que incluye con infinita frecuencia cualquier transición que es habilitada con infinita frecuencia. Una estrategia \emph{fair} para el Refutador, es una estrategia que siempre mide 1 sobre el conjunto de todas las jugadas \emph{fair} del Refutador, sin importar la estrategia del Verificador. La definición a continuación sigue el estilo de~\cite{DBLP:journals/dc/BaierK98,BaierK08,CastroDDP22}.



%
\begin{definition}
  Sea $\StochG_{A,A'} = (V^\StochG, E^\StochG, V^\StochG_\Refuter, V^\StochG_\Verifier, V^\StochG_\Probabilistic, \InitVertex, \delta^\StochG)$ un juego de enmascaramiento probabilista,
  el \emph{conjunto de todas las jugadas \emph{fair} del Refutador} está definido como
  $ 
	\RFP = \{ \omega \in \Omega \mid v \in \inf(\omega) \cap V^\StochG_\Refuter \Rightarrow \post(v) \subseteq \inf(\omega) \}
  $.
  %
  Una estrategia del Refutador $\strat{\Refuter}$ se dice que es \emph{casi-seguro fair} si y solo si, para toda estrategia del Verificador
  $\strat{\Verifier}$,
  $\Prob{\strat{\Refuter}}{\strat{\Verifier}}_{\StochG_{A,A'},\InitVertex}(\RFP) = 1$.
\end{definition}


%% %
%% \begin{definition} Given a masking game $\StochG_{A,A'} = (V^\StochG, E^\StochG, V^\StochG_\Refuter, V^\StochG_\Verifier, V^\StochG_\Probabilistic, \InitVertex, \delta^\StochG)$,
%% The set of Refuter's fair plays (denoted $\RFP$) is defined as follows:
%% \[
%% 	\RFP = \{ \omega \in \Omega \mid v \in \inf(\omega) \cap V^\StochG_\Refuter \Rightarrow \post(v) \subseteq \inf(\omega) \}
%% \]
%% \end{definition}
%% 	The almost-sure fair strategies for the Refuter are defined as follows.
%% \begin{definition} Given a masking game $\StochG_{A,A'} = (V^\StochG, E^\StochG, V^\StochG_\Refuter, V^\StochG_\Verifier, V^\StochG_\Probabilistic, \InitVertex, \delta^\StochG)$,
%% a Refuter strategy $\strat{\Refuter}$ is said to be \emph{almost-sure fair} iff, for any Verifier's strategy $\strat{\Verifier}$, 
%% $\Prob{\strat{\Refuter}}{\strat{\Verifier}}_{\StochG_{A,A'},  \InitVertex}(\RFP) = 1$.
%% \end{definition}

Bajo este concepto, el juego de enmascaramiento estocástico falla casi-seguramente bajo \emph{fairness} si para cada estrategia del Verificador y para cada estrategia \emph{fair} del Refutador, el juego lleva al estado de error con probabilidad 1. Esto se define formalmente a continuación.



\begin{definition}
  Sean $A$ y $A'$ dos PTS. Decimos que el juego de enmascaramiento estocástico $\StochG_{A,A'}$ \emph{falla casi-seguramente bajo fairness
  (y estrategias sin memoria)} si y solo si, para toda estrategia sin memoria
  $\strat{\Verifier} \in \Strategies{\Verifier}^{\Memoryless}$ y para cualquier estrategia \emph{fair} sin memoria
  $\strat{\Refuter} \in \Strategies{\Refuter}^{\Memoryless}$,
  $\Prob{\strat{\Verifier}}{\strat{\Refuter}}_{\StochG_{A,A'}, \InitVertex}(\Diamond \ErrorSt)=1$.
\end{definition}

%% Now, a model is an casi-seguro terminante implementation under fairness
%% of a given nominal model if for every Verifier's strategy and every
%% Refuter's fair strategy, the stochastich masking games leads to an
%% error with probability 1.  This is formally defined as follows.
%% %	We use  almost-sure fair strategies to define casi-seguro terminante implementations under fairness.

%% \begin{definition} Let $A =( S, \Sigma, \rightarrow, s_0 )$ and $A' =( S', \SigmaF, \rightarrow', s_0' )$ be two PTSs. We say that $A'$ is an \emph{casi-seguro terminante implementation under fairness}  of $A$ iff, 
%% for every strategy  $\strat{\Verifier} \in \Strategies{\Verifier}$ and any fair strategy $\strat{\Refuter} \in \Strategies{\Refuter}$, it holds: $\Prob{\strat{\Verifier}}{\strat{\Refuter}}_{\StochG_{A,A'},  \InitVertex}(\Diamond \ErrorSt)=1$.
%% \end{definition}



Es interesante observar que,  bajo supuestos de \emph{fairness} fuerte, la determinación de los juegos se preserva~\cite{CastroDDP22}. Además, en juegos estocásticos finitos con restricciones de \emph{fairness}, el valor del juego puede ser computado calculando el máximo punto fijo del siguiente funcional de Bellman. Adaptamos este resultado a nuestros juegos al usar los vértices de los politopos al computar los valores de los estados probabilistas. %vertices.
    
%% \begin{theorem}\label{theo:det-fairness} Let $\MilestoneG_{A,A'}$ be a stochastic game with milestones for some PTSs $A$ and $A'$ that is almost-sure 
%% failing for fair Refuter's strategies.  Then,  we have:
%% \[
%%  \inf_{\strat{\Refuter} \in \Strategies{\Refuter}^{\Memoryless}} \sup_{\strat{\Verifier} \in \Strategies{\Verifier}^{\Memoryless}}  \mathbb{E}^{\strat{\Verifier},\strat{\Refuter}}_{\MilestoneG_{A,A'}, v}[\FMask] 
%%     = \sup_{\strat{\Verifier \in \Strategies{\Verifier}^{\Memoryless}}} \inf_{\strat{\Refuter \in \Strategies{\Refuter}^{\Memoryless}}}    \mathbb{E}^{\strat{\Verifier},\strat{\Refuter}}_{\MilestoneG_{A,A'}, v}[\FMask] 
%%     < \infty.
%% \]
%% Moreover, 
%% the value of the game for memoryless strategies for the Verifier and  fair memoryless Refuter's strategies is the greatest fixpoint of the following functional $\Bellman$: 
%% %(see Theorem ~\ref{thm:solution_eq_stock_game}). 
%% \[
%%     \Bellman(f)(v) =
%%     \begin{cases}
%%            %\displaystyle \max_{w \in \vertices{\couplings{\pr{3}{v},\pr{4}{v}}}} \{\chi_{\milestones}(\pr{1}{v}) + \sum_{v' \in \post(v)} w(v,v')  \Bellman_{v'} \}& \text{ if }v \in V^{\SymbG}_\Verifier  \\
%%            %\displaystyle \max_{w \in \vertices{\couplings{\pr{3}{v}}{\pr{4}{v}}}} \{\chi_{\milestones}(\pr{1}{v}) + \sum_{v' \in \post(v)} w(\pr{0}{v'},\pr{2}{v'})  \Bellman_{v'} \}& \text{ if }v \in V^{\SymbG}_\Verifier  \\
%%            \displaystyle \min \{\upperbound,  \max_{w \in \vertices{\couplings{\pr{3}{v}}{\pr{4}{v}}}} \{ \chi_{\milestones}(\pr{1}{v})  + \sum_{v' \in \post(v)} w(\pr{0}{v'},\pr{2}{v'})  f(v') \} \}& \text{ if }v \in V^{\SymbG}_\Probabilistic  \\
%%            \displaystyle \min \{ \upperbound, \max \{\chi_{\milestones}(\pr{1}{v})  + f(v') \mid v' \in \post(v) \} \} & \text{ if } v \in  V^{\SymbG}_\Verifier, \\
%%            \displaystyle \min \{ \upperbound,  \min \{\chi_{\milestones}(\pr{1}{v})  + f(v') \mid v' \in \post(v) \} \} & \text{ if } v \in  V^{\SymbG}_\Refuter, \\
%%            \displaystyle 0 & \text{ if } v=\ErrorSt.
%%            \displaystyle 
%%     \end{cases}
%% \]
%% where $\upperbound$ is a number such that 
%% $\upperbound \geq \inf_{\strat{\Refuter} \in \Strategies{\Refuter}^{\Memoryless\Deterministic}} \sup_{\strat{\Verifier} \in \Strategies{\Verifier}^{\Memoryless\Deterministic}} \Expect{\strat{\Verifier}}{\strat{\Refuter}}_{\MilestoneG_{A,A'}, \InitVertex}[\FMask]$.
%% \end{theorem}
%% \reversemarginpar\marginnote{\color{blue}LP2ALL: The functional counts milestones in every case but it should only count on the V case.}[-4cm]


%\remarkPRD{Cambi\'e un poco el enunciado de este teorema y puede afectar cosas en el ap\'endice. Revisar}
\begin{theorem}\label{theo:det-fairness}
Sea $\milestones$ un conjunto de hitos para $A'$ y sea  $\StochG_{A,A'}$ un juego estocástico para $A$ y $A'$ que falla casi-seguramente para estrategias \emph{fair} del Refutador.  Entonces,  
\[\adjustlimits
 \inf_{\strat{\Refuter} \in \Strategies{\Refuter}^{\Memoryless}} \sup_{\strat{\Verifier} \in \Strategies{\Verifier}^{\Memoryless}}  \mathbb{E}^{\strat{\Verifier},\strat{\Refuter}}_{\StochG_{A,A'}, v}[\FMask] 
    = 
    \adjustlimits    
    \sup_{\strat{\Verifier} \in \Strategies{\Verifier}^{\Memoryless}} \inf_{\strat{\Refuter} \in \Strategies{\Refuter}^{\Memoryless}}    \mathbb{E}^{\strat{\Verifier},\strat{\Refuter}}_{\StochG_{A,A'}, v}[\FMask] 
    < \infty.
\]
Además, 
el valor del juego para estrategias sin memoria para el Verificador y estrategias \emph{fair} sin memoria para el Refutador es el punto fijo mayor del siguiente funcional  $\Bellman$: 
%
{\small
\[
    \Bellman(f)(v) =
    \begin{cases}
           \displaystyle \min \Big\{\upperbound,  \max_{w \in VC} \Big\{ \mreward^\StochG(v)  +\!\! \sum_{v' \in \post(v)}\!\! w(\pr{0}{v'},\pr{2}{v'})  f(v') \Big\} \Big\}& \text{ si }v \in V^{\SymbG}_\Probabilistic  \\
           \displaystyle \min \big\{ \upperbound, \max \big\{\mreward^\StochG(v)  + f(v') \mid v' \in \post(v) \big\} \big\} & \text{ si } v \in  V^{\SymbG}_\Verifier, \\
           \displaystyle \min \big\{ \upperbound,  \min \big\{\mreward^\StochG(v)  + f(v') \mid v' \in \post(v) \big\} \big\} & \text{ si } v \in  V^{\SymbG}_\Refuter, \\
           \displaystyle 0 & \text{ si } v=\ErrorSt.
           \displaystyle 
    \end{cases}
\]}%
donde el conjunto $VC=\vertices{\couplings{\pr{3}{v}}{\pr{4}{v}}}$ y $\upperbound$ es un número tal que
$\upperbound \geq \inf_{\strat{\Refuter} \in \Strategies{\Refuter}^{\Memoryless\Deterministic}} \sup_{\strat{\Verifier} \in \Strategies{\Verifier}^{\Memoryless\Deterministic}} \Expect{\strat{\Verifier}}{\strat{\Refuter}}_{\StochG_{A,A'}, \InitVertex}[\FMask]$.
\end{theorem}

%\textbf{Proof of Theorem \ref{theo:det-fairness}.} Let $\StochG_{A,A'}$ be a stochastic game that is almost-sure 
%failing for fair Refuter's strategies,  and  let $\milestones$ be a milestone set for $A'$.  Then,  
%\[
%    \inf_{\strat{\Refuter} \in \Strategies{\Refuter}^{\Memoryless}} \sup_{\strat{\Verifier} \in \Strategies{\Verifier}^{\Memoryless}}  \mathbb{E}^{\strat{\Verifier},\strat{\Refuter}}_{\StochG_{A,A'}, v}[\FMask] 
%    = \sup_{\strat{\Verifier \in \Strategies{\Verifier}^{\Memoryless}}} \inf_{\strat{\Refuter \in \Strategies{\Refuter}^{\Memoryless}}}    \mathbb{E}^{\strat{\Verifier},\strat{\Refuter}}_{\StochG_{A,A'}, v}[\FMask] 
%    < \infty
%\]
%Furthermore, the value of the game for memoryless strategies for the Verifier and  fair Memoryless Refuter's strategies is the greatest fixpoint of the following functional $\Bellman$: 
%(see Theorem ~\ref{thm:solution_eq_stock_game}). 
%\[\small
%    \Bellman(f)(v) =
%    \begin{cases}
           %\displaystyle \max_{w \in \vertices{\couplings{\pr{3}{v},\pr{4}{v}}}} \{\mreward^\StochG(v) + \sum_{v' \in \post(v)} w(v,v')  \Bellman_{v'} \}& \text{ if }v \in V^{\SymbG}_\Verifier,  \\
           %\displaystyle \max_{w \in \vertices{\couplings{\pr{3}{v}}{\pr{4}{v}}}} \{\mreward^\StochG(v) + \sum_{v' \in \post(v)} w(\pr{0}{v'},\pr{2}{v'})  \Bellman_{v'} \}& \text{ if }v \in V^{\SymbG}_\Verifier,  \\
%           \displaystyle \min \{\upperbound,  \max_{w \in \vertices{\couplings{\pr{3}{v}}{\pr{4}{v}}}} \{\mreward^\StochG(v) + \sum_{v' \in \post(v)} w(\pr{0}{v'},\pr{2}{v'})  f(v') \} \}& \text{ if }v \in V^{\SymbG}_\Probabilistic  \\
%           \displaystyle \min \{ \upperbound, \max \{\mreward^\StochG(v)  + f(v') \mid v' \in \post(v) \} \} & \text{ if } v \in  V^{\SymbG}_\Verifier, \\
%           \displaystyle \min \{ \upperbound,  \min \{\mreward^\StochG(v)  + f(v') \mid v' \in \post(v) \} \} & \text{ if } v \in  V^{\SymbG}_\Refuter, \\
%           \displaystyle 0 & \text{ if } v=\ErrorSt.
%           \displaystyle 
%    \end{cases}
%\]
%where $\SymbG_{A,A'}$ is the corresponding symbolic game, and $\upperbound$ is a number such that 
%$\upperbound \geq \inf_{\strat{\Refuter} \in \Strategies{\Refuter}^{\Memoryless\Deterministic}} \sup_{\strat{\Verifier} \in \Strategies{\Verifier}^{\Memoryless\Deterministic}} \Expect{\strat{\Verifier}}{\strat{\Refuter}}_{\StochG_{A,A'}, v}[\FMask]$, for every $v$.

% PRUEBA PASADA AL APENDICE
\iffalse
\noindent
\begin{proof}
Primero vamos a probar que podemos restringirnos a estrategias deterministas de manera segura al computar el valor del juego para estrategias sin memoria.  Para hacer esto, demostramos ahora que para todas estrategias sin memoria $\strat{\Verifier}$ y $\strat{\Refuter}$, hay una estrategia sin memoria y determinista
$\strat{\Verifier}'$ tal que: $\mathbb{E}^{\strat{\Verifier}',\strat{\Refuter}}_{\StochG_{A,A'}, v}[\FMask]  \geq \mathbb{E}^{\strat{\Verifier},\strat{\Refuter}}_{\StochG_{A,A'}, v}[\FMask]$. Primero, observemos que cualquier estrategia sin memoria satisface la siguiente ecuación para todo $v \in V^\StochG_\Verifier$:
\begin{align}
    \mathbb{E}_{\StochG_{A,A'},v}^{\strat{\Verifier},\strat{\Refuter}}[\FMask]&  \leq \mreward^\StochG(v) + \sum_{v' \in \post(v)} \delta^{\strat{\Refuter},\strat{\Verifier}}(v,v')  \mathbb{E}_{\StochG_{A,A'},v'}^{\strat{\Verifier},\strat{\Refuter}}[\FMask] \label{theo:det-fairness:eq3:l1}\\
    & \leq \mreward^\StochG(v)  + \max_{v' \in \post(v)} \{  \mathbb{E}_{\StochG_{A,A'},v'}^{\strat{\Verifier},\strat{\Refuter}}[\FMask] \},
    \label{theo:det-fairness:eq3:l12}
\end{align}
\sloppy donde $\delta^{\strat{\Refuter},\strat{\Verifier}}(v,v')$ denota la función de transición probabilista obtenida cuando las estrategias $\strat{\Refuter}$ y $\strat{\Refuter}$
están fijadas en el juego $\StochG_{A,A'}$. La primera desigualdad se deduce de la definición de valor esperado, la segunda desigualdad se deduce de que $ \sum_{v' \in \post(v)} \delta^{\strat{\Refuter},\strat{\Verifier}}(v,v') \mathbb{E}_{\StochG_{A,A'},v'}^{\strat{\Verifier},\strat{\Refuter}}[\FMask] $ es una combinación convexa. Es decir,  definiendo $\strat{\Verifier}'(v) = \argmax_{v' \in \post(v)}  \{ \mathbb{E}_{\StochG_{A,A'},v'}^{\strat{\Verifier},\strat{\Refuter}}[\FMask] \}$, para todo $v$,  obtenemos
$\mathbb{E}_{\StochG_{A,A'},v}^{\strat{\Verifier},\strat{\Refuter}}[\FMask] \leq \mathbb{E}_{\StochG_{A,A'},v}^{\strat{\Verifier}',\strat{\Refuter}}[\FMask]$. 
Similarmente podemos probar que para todas estrategias sin memoria $\strat{\Refuter}$ y $\strat{\Verifier}$, existe una estrategia sin memoria, determinista y \emph{fair} 
$\strat{\Refuter}'$ tal que  $\mathbb{E}_{\StochG_{A,A'},v}^{\strat{\Verifier},\strat{\Refuter}'}[\FMask] \leq \mathbb{E}_{\StochG_{A,A'},v}^{\strat{\Verifier},\strat{\Refuter}}[\FMask]$.  Estas propiedades implican que:
\[
    \inf_{\strat{\Refuter} \in \Strategies{\Refuter}^{\Memoryless\Deterministic}}  \sup_{\strat{\Verifier} \in \Strategies{\Verifier}^{\Memoryless\Deterministic}} \mathbb{E}_{\StochG_{A,A'},v'}^{\strat{\Verifier},\strat{\Refuter}}[\FMask] 
    =  \inf_{\strat{\Refuter} \in \Strategies{\Refuter}^{\Memoryless}}  \sup_{\strat{\Verifier} \in \Strategies{\Verifier}^{\Memoryless}} \mathbb{E}_{\StochG_{A,A'},v'}^{\strat{\Verifier},\strat{\Refuter}}[\FMask]
\]
y de forma similar:
\[
    \sup_{\strat{\Verifier} \in \Strategies{\Verifier}^{\Memoryless\Deterministic}}  \inf_{\strat{\Refuter} \in \Strategies{\Refuter}^{\Memoryless\Deterministic}} \mathbb{E}_{\StochG_{A,A'},v'}^{\strat{\Verifier},\strat{\Refuter}}[\FMask] 
    =  \sup_{\strat{\Verifier} \in \Strategies{\Verifier}^{\Memoryless}}  \inf_{\strat{\Refuter} \in \Strategies{\Refuter}^{\Memoryless}} \mathbb{E}_{\StochG_{A,A'},v'}^{\strat{\Verifier},\strat{\Refuter}}[\FMask].
\]

    Ahora demostraremos el teorema.  Podemos definir un juego (finito) restringido solo teniendo en cuenta los vértices del politopo definidos por los \emph{couplings}.  Consideremos el sub-juego $\mathcal{H}_{A,A'}$ obtenido de $\StochG_{A,A'}$ al restringir los sucesores de los vértices del Verificador a los siguientes conjuntos:
\begin{itemize}
	\item $\{ \langle (s, \sigma^2, s', \mhyphen, \mu', \mhyphen, \Verifier), (s, \mhyphen, s', \mu, \mu', w, \Probabilistic) \rangle \mid (\exists\;\sigma \in \Sigma: s \xrightarrow{\sigma} \mu) \wedge   w \in \vertices{\couplings{\mu}{\mu'}}\} \subseteq E^\StochG$ para todo $\sigma \notin \faults$,

 	 \item $\{ \langle (s, \sigma^1, s', \mu, \mhyphen, \mhyphen, \Verifier),(s, \mhyphen, s', \mu, \mu', w, \Probabilistic) \rangle \mid (\exists\;\sigma \in \Sigma: s' \xrightarrowprime{\sigma} \mu' ) \wedge  w \in \vertices{\couplings{\mu}{\mu'}} \} \subseteq E^\StochG$,
	 
	 \item $\{ \langle (s, F^2, s', \mhyphen, \mu', \mhyphen, \Verifier), (s, \mhyphen, s', \Dirac_s, \mu', w, \Probabilistic) \rangle \wedge w \in \vertices{\couplings{\Dirac_s}{\mu'}} \} \subseteq E^\StochG$ para todo $F \in \faults$,
\end{itemize}
Es decir, restringimos los \emph{couplings} a los vértices del politopo $\couplings{\mu}{\mu'}$.  Observemos que como el conjunto de vértices es finito, el juego  $\mathcal{H}_{A,A'}$ también es finito.  
Ahora vamos a probar que:
\begin{equation}\label{eq:theo:determined:eq2}
    \sup_{\strat{\Verifier} \in \Strategies{\Verifier}^{\Memoryless}} \inf_{\strat{\Refuter} \in \Strategies{\Refuter}^{\Memoryless}} \mathbb{E}_{\mathcal{H}_{A,A'},v'}^{\strat{\Verifier},\strat{\Refuter}}[\FMask]
    \leq 
     \sup_{\strat{\Verifier} \in \Strategies{\Verifier}^{\Memoryless}} \inf_{\strat{\Refuter} \in \Strategies{\Refuter}^{\Memoryless}} \mathbb{E}_{\mathcal{G}_{A,A'},v'}^{\strat{\Verifier},\strat{\Refuter}}[\FMask],
\end{equation}
y:
\begin{equation}\label{eq:theo:determined:eq3}
    \inf_{\strat{\Refuter} \in \Strategies{\Refuter}^{\Memoryless}} \sup_{\strat{\Verifier} \in \Strategies{\Verifier}^{\Memoryless}} \mathbb{E}_{\mathcal{G}_{A,A'},v'}^{\strat{\Verifier},\strat{\Refuter}}[\FMask]
    \leq 
     \inf_{\strat{\Refuter} \in \Strategies{\Refuter}^{\Memoryless}} \sup_{\strat{\Verifier} \in \Strategies{\Verifier}^{\Memoryless}} \mathbb{E}_{\mathcal{H}_{A,A'},v'}^{\strat{\Verifier},\strat{\Refuter}}[\FMask],
\end{equation}
    Observemos que, por la propiedad que probamos arriba, estas son equivalentes a:
\begin{equation}\label{eq:theo:determined:eq4}
    \sup_{\strat{\Verifier} \in \Strategies{\Verifier}^{\Memoryless\Deterministic}} \inf_{\strat{\Refuter} \in \Strategies{\Refuter}^{\Memoryless\Deterministic}} \mathbb{E}_{\mathcal{H}_{A,A'},v'}^{\strat{\Verifier},\strat{\Refuter}}[\FMask]
    \leq 
     \sup_{\strat{\Verifier} \in \Strategies{\Verifier}^{\Memoryless\Deterministic}} \inf_{\strat{\Refuter} \in \Strategies{\Refuter}^{\Memoryless\Deterministic}} \mathbb{E}_{\mathcal{G}_{A,A'},v'}^{\strat{\Verifier},\strat{\Refuter}}[\FMask],
\end{equation}
y:
\begin{equation}\label{eq:theo:determined:eq5}
    \inf_{\strat{\Refuter} \in \Strategies{\Refuter}^{\Memoryless\Deterministic}} \sup_{\strat{\Verifier} \in \Strategies{\Verifier}^{\Memoryless\Deterministic}} \mathbb{E}_{\mathcal{G}_{A,A'},v'}^{\strat{\Verifier},\strat{\Refuter}}[\FMask]
    \leq 
     \inf_{\strat{\Refuter} \in \Strategies{\Refuter}^{\Memoryless\Deterministic}} \sup_{\strat{\Verifier} \in \Strategies{\Verifier}^{\Memoryless\Deterministic}} \mathbb{E}_{\mathcal{H}_{A,A'},v'}^{\strat{\Verifier},\strat{\Refuter}}[\FMask],
\end{equation}
(\ref{eq:theo:determined:eq4}) vale ya que $\post^{\mathcal{H}_{A,A'}}(v) \subseteq \post^{\StochG_{A,A'}}(v)$ para $v \in V^{\mathcal{H}_{A,A'}}_\Verifier$ y
$\post^{\mathcal{H}_{A,A'}}(v) = \post^{\StochG_{A,A'}}(v)$ para $v \in V^{\mathcal{H}_{A,A'}}_\Refuter$.  Para probar (\ref{eq:theo:determined:eq5}), fijemos una estrategia \emph{fair} $\strat{\Refuter} \in \Strategies{\Refuter}^{\Memoryless \Deterministic}$, la estrategia óptima para el Verificador en el juego $\mathcal{G}_{A,A'}$ se obtiene solo en vértices probabilistas que son vértices de $\couplings{\mu}{\mu'}$, estos son vértices probabilistas de $\mathcal{H}_{A,A'}$, por lo tanto 
$  \sup_{\strat{\Verifier} \in \Strategies{\Verifier}^{\Memoryless\Deterministic}} \mathbb{E}_{\mathcal{G}_{A,A'},v'}^{\strat{\Verifier},\strat{\Refuter}}[\FMask]
    \leq 
     \sup_{\strat{\Verifier} \in \Strategies{\Verifier}^{\Memoryless\Deterministic}} \mathbb{E}_{\mathcal{H}_{A,A'},v'}^{\strat{\Verifier},\strat{\Refuter}}[\FMask],$
para cualquier estrategia \emph{fair} sin memoria $\strat{\Refuter}$,  (\ref{eq:theo:determined:eq5}) se deduce.

    Además, el valor del juego $\mathcal{H}_{A,A'}$ está dado por el mayor punto fijo de las ecuaciones \cite{CastroDDP22}:
\begin{equation}\label{eq:theo:determined:bellman1}\small
    \Bellman(f)(v) =
    \begin{cases}
           \displaystyle \min \{\upperbound, \mreward^\StochG(v) + \sum_{v' \in \post(v)} \delta(v)(v')  f(v') \} & \text{ if } v \in V^{\mathcal{H}}_\Probabilistic  \\
           \displaystyle \min \{\upperbound,  \max \{\mreward^\StochG(v) +f(v') \mid v' \in \post(v) \} \} & \text{ si } v \in  V^{\mathcal{H}}_\Verifier, \\
           \displaystyle \min \{\upperbound,  \min \{\mreward^\StochG(v)  + f(v') \mid v' \in \post(v) \} \} & \text{ si } v \in  V^{\mathcal{H}}_\Refuter, \\
           \displaystyle 0 & \text{ si } v=\ErrorSt.
           \displaystyle 
    \end{cases}
\end{equation}
    para algún $\upperbound \geq \inf_{\strat{\Refuter} \in \Strategies{\Refuter}^{\Memoryless}} \sup_{\strat{\Verifier} \in \Strategies{\Verifier}^{\Memoryless}} \mathbb{E}_{\mathcal{H}_{A,A'},v'}^{\strat{\Verifier},\strat{\Refuter}}[\FMask]$.
    Es decir, tenemos:
\begin{equation}\label{eq:theo:determined:eq6}
    \inf_{\strat{\Refuter} \in \Strategies{\Refuter}^{\Memoryless\Deterministic}} \sup_{\strat{\Verifier} \in \Strategies{\Verifier}^{\Memoryless\Deterministic}}  \mathbb{E}^{\strat{\Verifier},\strat{\Refuter}}_{\mathcal{H}_{A,A'}, v}[\FMask] 
    = \sup_{\strat{\Verifier \in \Strategies{\Verifier}^{\Memoryless\Deterministic}}} \inf_{\strat{\Refuter \in \Strategies{\Refuter}^{\Memoryless\Deterministic}}}    \mathbb{E}^{\strat{\Verifier},\strat{\Refuter}}_{\mathcal{H}_{A,A'}, v}[\FMask] 
    < \infty
\end{equation}
  Por lo tanto, a causa de (\ref{eq:theo:determined:eq4}),  (\ref{eq:theo:determined:eq5}) y (\ref{eq:theo:determined:eq6}) tenemos:
\[
 \inf_{\strat{\Refuter} \in \Strategies{\Refuter}^{\Memoryless}} \sup_{\strat{\Verifier} \in \Strategies{\Verifier}^{\Memoryless}}  \mathbb{E}^{\strat{\Verifier},\strat{\Refuter}}_{\StochG_{A,A'}, v}[\FMask] 
    = \sup_{\strat{\Verifier \in \Strategies{\Verifier}^{\Memoryless}}} \inf_{\strat{\Refuter \in \Strategies{\Refuter}^{\Memoryless}}}    \mathbb{E}^{\strat{\Verifier},\strat{\Refuter}}_{\StochG_{A,A'}, v}[\FMask] 
    < \infty.
\]
    Esto prueba una parte del teorema.
    Ahora, consideremos el siguiente funcional sobre el juego simbólico:
\[\small
\Bellman'(f)(v) =
    \begin{cases}
           \displaystyle \min \{ \upperbound, \max_{w \in \vertices{\couplings{\pr{3}{v}}{\pr{4}{v}}}} \{\mreward^\SymbG(v)  + \sum_{v' \in \post(v)} w(\pr{0}{v'},\pr{2}{v'})  f(v') \} \}& \text{ si }v \in V^{\SymbG}_\Probabilistic  \\
           \displaystyle \min \{ \upperbound, \max \{\mreward^\SymbG(v)  +f(v') \mid v' \in \post(v) \} \} & \text{ si } v \in  V^{\SymbG}_\Verifier, \\
           \displaystyle \min \{ \upperbound, \min \{\mreward^\SymbG(v) + f(v') \mid v' \in \post(v) \} \} & \text{ si } v \in  V^{\SymbG}_\Refuter, \\
           \displaystyle 0 & \text{ si } v=\ErrorSt.
           \displaystyle 
    \end{cases}
\]
Vamos a probar que este puede ser utilizado para resolver $\Bellman$. Primero, observemos que $\Bellman'$ es monótona,  está definida sobre el retículo completo $[0,\upperbound]$ y es Scott-completa. Por lo tanto, tiene un mayor punto fijo.  Sea $\nu \Bellman'$ el mayor punto fijo de $\Bellman',$  vamos a demostrar que
$\nu \Bellman(v) = \nu \Bellman'((v[0],v[1],v[2],v[3],v[4], v[6]))$, para todo $v \in V^{\mathcal{H}_{A,A'}}_\Verifier \cup V^{\mathcal{H}}_\Refuter$.
Para realizar esto, consideremos para cada vértice simbólico el siguiente \emph{mapping}:
\begin{itemize}
    \item $\llbracket (s,\sigma,s',\mu,\mu',X) \rrbracket = (s,\sigma,s',\mu,\mu',\mhyphen,X)$, para $X \in \{\Refuter, \Verifier\}$,
    \item $\llbracket (s,\mhyphen, s',\mu,\mu',\Probabilistic) \rrbracket =(s, \mhyphen,s', \mu,\mu', w,\Probabilistic)$,  \\ donde
              $w = \argmax_{w \in \vertices{\couplings{\mu}{\mu'}}} \{\sum_{v' \in \post(v)} w(\pr{0}{v'},\pr{2}{v'})  \nu \Bellman'(v') \}$
\end{itemize}
    De manera similar, podemos definir una asociación desde vértices concretos a vértices simbólicos:
\begin{itemize}
    \item $\llparenthesis (s,\sigma,s',\mu,\mu',Y ,X) \rrparenthesis = (s,\sigma,s',\mu,\mu',X)$, para $X \in \{\Refuter, \Verifier \}$ y $Y \in \{\mhyphen \} \cup \vertices{\couplings{\mu,\mu'}}$.
\end{itemize}
    Ahora bien, probemos que $\alpha(v) = \nu \Bellman'(\llparenthesis v \rrparenthesis)$ es un punto fijo de $\Bellman$. Procedemos por casos:
   
   Si $v$ es un vértice del Refutador,  entonces:
\begin{align}
   \Bellman(\alpha)(v) & =  \min \{\upperbound,  \min \{\mreward^\StochG(v)  + \alpha(v') \mid v' \in \post(v) \} \}  \\
                                    & =  \min \{\upperbound,  \min \{\mreward^\SymbG(v)  +\nu \Bellman'(\llparenthesis v'  \rrparenthesis) \mid v' \in \post(v) \} \} \\  
                                    & = \nu \Bellman'(\llparenthesis v  \rrparenthesis) \\
                                    & = \alpha(v)           
\end{align}
 La primer linea es por definición de $\Bellman$, la segunda linea se obtiene aplicando la definición de $\alpha$, la tercer linea se debe a la suryectividad de  $\llparenthesis \rrparenthesis$,  el hecho de que $\mreward^\SymbG(v) = \mreward^\StochG(v) $, la definición de $\Bellman'$ y ya que $\nu \Bellman'(\llparenthesis v  \rrparenthesis)$ es un punto fijo de $\Bellman'$.

    Si $v$ es un vértice del Verificador entonces:
\begin{align}
   \Bellman(\alpha)(v) & =   \min \{ \upperbound, \max \{\mreward^\StochG(v) + \alpha(v') \mid v' \in \post(v) \} \}  \\
                                    & =   \min \{ \upperbound, \mreward^\StochG(v) + \max_{w \in \vertices{\couplings{\pr{3}{v}}{\pr{4}{v}}}} \{\sum_{v' \in \post(v)} w(\pr{0}{v'},\pr{2}{v'})  \nu \Bellman'(\llparenthesis v' \rrparenthesis) \} \} \\  
                                    & = \nu \Bellman'(\llparenthesis v  \rrparenthesis) \\
                                    & = \alpha(v)           
\end{align}
    La segunda linea se debe a que los vértices probabilistas en $\mathcal{H}_{A,A'}$ son exactamente los vértices del politopo que define los \emph{couplings} posibles.

    Por esto, $\alpha$ es un punto fijo de $\Bellman$ para los vértices en $V^{\mathcal{H}}_\Verifier \cup V^{\mathcal{H}}_\Refuter$.  Además,  probaremos que es el mayor punto fijo. Asumamos por contradicción que existe algún $\alpha'$ tal que es un punto fijo de $\Bellman$ y $\alpha'(v) \geq  \alpha(v)$ para todo $v \in V^{\mathcal{H}}_\Verifier \cup V^{\mathcal{H}}_\Refuter$, y $\alpha'(v') >  \alpha(v')$ para algún $v' \in V^{\mathcal{H}}_\Verifier \cup V^{\mathcal{H}}_\Refuter$.  
    Podemos definir $\beta : V^{\SymbG} \rightarrow [0,\upperbound]$ como sigue: $\beta(v) = \alpha'(\llbracket v \rrbracket)$, como hicimos arriba podemos probar que es un punto fijo de $\Bellman'$ y, además, para todo vértice simbólico tenemos $\beta(v) =  \alpha'(\llbracket v \rrbracket) \geq \alpha(\llbracket v \rrbracket) = \nu \Bellman'(\llparenthesis \llbracket v \rrbracket \rrparenthesis)
    = \nu \Bellman'(v)$, y de forma similar podemos probar que existe un $v'$ tal que $\beta(v') >  \nu \Bellman'(v)$, lo cual es una contradicción ya que $\nu \Bellman'$ es el mayor punto fijo de $\Bellman'$.
\end{proof}\\
\fi
%% \reversemarginpar\marginnote{\color{blue}LP2ALL: The functional counts milestones in every case but it should only count on the V case.}[-4cm]
%% \remarkPRD{Notar que con el cambio de la Def. 6 queda solucionado el problema que se\~nala Luciano}
La constante $\upperbound$ es necesaria para que Knaster-Tarski aplique al retículo completo $[0,\upperbound]^V$~\cite{CastroDDP22}.



%%    Also, we can check whether  a game is casi-seguro terminante under fairness by computing predecessor sets in the symbolic game graph.  To do so, we define the symbolic version of the predecessor sets.
%% %\begin{align*}
%% %	\EFairpre(C) =& \{ v \in V^\StochG_\Probabilistic \mid \delta(v)(C) > 0\} \\
%% %		       & \cup \{ v \in  V^\StochG_\Verifier \cup V^\StochG_\Refuter  \mid \exists v' \in C : (v,v') \in E^\StochG \}\\
%% %	\AFairpre(C) = &\{ v \in V^\StochG_\Probabilistic \mid \delta(v)(C) > 0\} \\
%% %		      & \cup \{ v \in  V^\StochG_\Verifier   \mid \forall v' \in C : (v,v') \in E^\StochG \Rightarrow v' \in C \}\\
%% %		      &  \cup \{ v \in  V^\StochG_\Refuter  \mid \exists v' \in C : (v,v') \in E^\StochG \}
%% %\end{align*}
%% %	These sets allows us to check if a given game is casi-seguro terminante for $\starredstrat{\Refuter}$.
%% %\begin{theorem} Given a masking game $\StochG_{A,A'} = (V^\StochG, E^\StochG, V^\StochG_\Refuter, V^\StochG_\Verifier, V^\StochG_\Probabilistic, \InitVertex, \delta^\StochG)$,
%% %then $\Prob{\strat{\Verifier}}{\strat{\Refuter}}_{\StochG_{A,A'}}(\mathcal{A}) = 1$ for every fair Refuter's strategy $\strat{\Refuter}$ and Verifier's strategy $\strat{\Verifier}$, being $\mathcal{A} = \{ \rho \in \Omega \mid \rho_i = \ErrorSt \}$, iff
%% %$\InitVertex \notin \EFairpre^*(V^\StochG \setminus \AFairpre^*(\ErrorSt))$.
%% %\end{theorem}
%% %\begin{proof} Consider the MDP $\StochG^{\starredstrat{\Refuter}}_{A,A'}$, then note that $v \in  \AFairpre^*(C)$ (in game $\StochG_{A,A'}$) iff
%% %$v \in  \Apre^*(C)$ (in MDP $\StochG^{\starredstrat{\Refuter}}_{A,A'}$), so any successor of a Refuter's state in  $\StochG^{\starredstrat{\Refuter}}_{A,A'}$ has a positive probability.
%% %Then, $ v \in V^\StochG \setminus \AFairpre^*(C)$ (in game $\StochG_{A,A'}$) iff $v \in V^\StochG \setminus \Apre^*(C)$ (in MDP $\StochG^{\starredstrat{\Refuter}}_{A,A'}$), because $\starredstrat{\Refuter}$ is memoryless, which means that the states in both are the same. 
%% %That is, $ v \in \EFairpre^*(V^\StochG \setminus \AFairpre^*(C))$ (in game $\StochG_{A,A'}$)
%% %iff $ v \in \Epre^*(V^{\StochG^{\starredstrat{\Refuter}}} \setminus \Apre^*(C))$, since $\Epre^*(S)$ and $\EFairpre^*(S)$ coincide over 
%% %any set of vertices.
%% %\end{proof}\\
%%     Given a game $\StochG_{A,A'}$ and its symbolic version $\SymbG_{A,A'}$, let $\SymbEFairpre(C)$ and 
%% $\SymbAFairpre(C)$, for a given set $C$ of symbolic vertices, be defined as follows:

Además, podemos chequear si un juego falla casi-seguramente bajo \emph{fairness} al computar los conjuntos predecesores en el grafo de juego simbólico. Para llevar a cabo esto, definimos los conjuntos predecesores en el grafo de juego simbólico $\SymbG_{A,A'}$ para un conjunto dado $C$ de vértices simbólicos, como a continuación:
%
{\small
\begin{align*}
	\SymbEFairpre(C) ={}& \{ v \in V^\SymbG_\Probabilistic \mid  \exists v' \in C\cap V^\SymbG_\Refuter : \pr{0}{v'}\in\support{\pr{3}{v}} \land \pr{2}{v'}\in\support{\pr{4}{v}} \} \\
		       & \cup \{ v \in  V^\SymbG_\Verifier \cup V^\SymbG_\Refuter  \mid \exists v' \in C : (v,v') \in E^\SymbG \}\\
	\SymbAFairpre(C) = {}&\{ v \in V^\SymbG_\Probabilistic \mid \Eq(v,C) \textit{  no tiene solución }\} \\
		      & \cup \{ v \in  V^\SymbG_\Verifier   \mid \forall v' : (v,v') \in E^\SymbG \Rightarrow v' \in C \}\\
		      &  \cup \{ v \in  V^\SymbG_\Refuter  \mid \exists v' \in C : (v,v') \in E^\SymbG \}
\end{align*}}%
%% \begin{align*}
%% 	\SymbEFairpre(C) ={}& \{ v \in V^\SymbG_\Probabilistic \mid  \exists v' \in C\cap V^\SymbG_\Refuter : \pr{0}{v'}\in\support{\pr{3}{v}} \land \pr{2}{v'}\in\support{\pr{4}{v}} \} \\
%% 		       & \cup \{ v \in  V^\SymbG_\Verifier \cup V^\SymbG_\Refuter  \mid \exists v' \in C : (v,v') \in E^\SymbG_{A,A'} \}\\
%% 	\SymbAFairpre(C) ={}&\{ v \in V^\SymbG_\Probabilistic \mid \Eq(v,C) \textit{  has no solution }\} \cup \{ v \in  V^\SymbG_\Refuter  \mid \exists v' \in C : (v,v') \in E^\SymbG \}\\
%% 		      & \cup \{ v \in  V^\SymbG_\Verifier   \mid \forall v' \in C : (v,v') \in E^\SymbG \Rightarrow v' \in C \}
%% \end{align*}
%
En particular, el primer conjunto en la definición de $\SymbEFairpre(C)$
contiene todos los vértices probabilistas $v$ para los cuales existe un \emph{coupling} que lleva al vértice del Refutador $v'$ en $C$.  Para esto es suficiente chequear que los estados $\pr{0}{v'}$ y $\pr{2}{v'}$ que definen a $v'$
están en los respectivos conjuntos soporte de las probabilidades
$\pr{3}{v}$ y $\pr{4}{v}$ que definen a $v$ (como
siempre es posible definir un \emph{coupling} que asigne probabilidad positiva a un par de estados en los conjuntos soporte correspondientes).  El primer conjunto en la definición de $\SymbAFairpre(C)$ contiene todos los vértices probabilistas $v$ para los cuales no existe \emph{coupling} ``evite'' $C$, es decir, que no haya \emph{coupling} que lleve con probabilidad 0 al conjunto de todos los pares de estados que definen un vértice en $C$.  Un \emph{coupling} que evite
$C$ resuelve $\Eq(v,C)$.
%
%% By using $\SymbEFairpre$ and $\SymbAFairpre$ recursively, we can
%% decide whether a game is casi-seguro terminante under fairness as follows.
Utilizando $\SymbEFairpre$ y $\SymbAFairpre$, podemos decidir si un juego falla casi-seguramente bajo \emph{fairness}:

\begin{theorem}\label{theo:decide-stopping}
%% Given a masking game $\StochG_{A,A'}$ and its symbolic version $\SymbG_{A,A'}$,  we have that 
%%  $\StochG_{A,A'}$ is casi-seguro terminante under fairness  iff
%% $
%%   \InitVertex \in V^\SymbG \setminus {\SymbEFairpre}^*(V^\SymbG \setminus {\SymbAFairpre}^*(\{ \ErrorSt \})),
%% $
%% where $\InitVertex$ is the initial state of $\SymbG_{A,A'}$ and $V^\SymbG$ its sets of vertices.

El juego de enmascaramiento 
 $\StochG_{A,A'}$ falla casi-seguramente bajo \emph{fairness}  si y solo si
$
  \InitVertex \in V^\SymbG \setminus {\SymbEFairpre}^*(V^\SymbG \setminus {\SymbAFairpre}^*(\{ \ErrorSt \})),
$
donde $\InitVertex$ es el estado inicial de $\SymbG_{A,A'}$ (la versión simbólica de $\StochG_{A,A'}$) y $V^\SymbG$ son los conjuntos de vértices $\SymbG_{A,A'}$.
\end{theorem}
%\noindent
%\textbf{Proof of Theorem \ref{theo:decide-stopping}}
%Given a masking game $\StochG_{A,A'}$ and its symbolic version $\SymbG_{A,A'}$,  we have that 
% $\StochG_{A,A'}$ is stopping under fairness  iff
%$
%  \InitVertex \in V^\SymbG \setminus {\SymbEFairpre}^*(V^\SymbG \setminus {\SymbAFairpre}^*(\{ \ErrorSt \})),
%$
%where $\InitVertex$ is the initial state of $\SymbG_{A,A'}$ and $V^\SymbG$ its sets of vertices.
% PRUEBA PASADA AL APENDICE
\iffalse
\noindent
\begin{proof} 
Consideremos el juego $\mathcal{H}_{A,A'}$ como se definió en la prueba del Teorema~\ref{theo:det-fairness}.  Primero, vamos a mostrar que el juego $\StochG_{A,A'}$ falla casi-seguramente para estrategias \emph{fair} del Refutador si y solo si $\mathcal{H}_{A,A'}$ también es falla casi-seguramente para estrategias \emph{fair} del Refutador.  
Esto es equivalente a probar que $\inf_{\strat{\Verifier}} \Prob{\strat{\Verifier}}{\strat{\Refuter}}_{\StochG_{A,A'}, \InitVertex}(\Diamond \ErrorSt)=1$ 
si y solo si $\inf_{\strat{\Verifier}} \Prob{\strat{\Verifier}}{\strat{\Refuter}}_{\mathcal{H}_{A,A'}, \InitVertex}(\Diamond \ErrorSt)=1$ para toda estrategia fair sin memoria  $\strat{\Refuter}$. Ahora bien, observemos que para toda estrategia sin memoria del Verificador $\strat{\Verifier}$ tenemos:
\begin{align*}
     \Prob{\strat{\Verifier}}{\strat{\Refuter}}_{\mathcal{G}_{A,A'}, v}(\Diamond \ErrorSt) & \geq \min\{ \sum_{v' \post(v)} w(\pr{0}{v'}, \pr{2}{v'})\Prob{\strat{\Verifier}}{\strat{\Refuter}}_{\mathcal{G}_{A,A'}, v'}(\Diamond \ErrorSt) \mid w \in \couplings{\pr{3}{v}}{\pr{4}{v}} \}\\
         & = \min\{ \sum_{v' \post(v)} w(\pr{0}{v'}, \pr{2}{v'})\Prob{\strat{\Verifier}}{\strat{\Refuter}}_{\mathcal{G}_{A,A'}, v'}(\Diamond \ErrorSt) \mid w \in\vertices{\couplings{\pr{3}{v}}{\pr{4}{v}}} \}
\end{align*}
Por lo tanto, tenemos una estrategia determinista y sin memoria $\strat{\Verifier}'$ tal que:
\[
\Prob{\strat{\Verifier}'}{\strat{\Refuter}}_{\mathcal{G}_{A,A'}, v}(\Diamond \ErrorSt) = \inf_{\strat{\Verifier} \in \Strategies{\Verifier}^{\Memoryless}} \Prob{\strat{\Verifier}}{\strat{\Refuter}}_{\StochG_{A,A'}, \InitVertex}(\Diamond \ErrorSt),
\]
esta estrategia solo elige vértices probabilistas en  $V^{\mathcal{H}}_{\Probabilistic}$, y por lo tanto, $\strat{\Verifier}'$ es una estrategia 
en $\mathcal{H}_{A,A'}$. Entonces las cadenas de Markov  $\StochG^{\strat{\Verifier}',\strat{\Refuter}}$,$ \mathcal{H}^{\strat{\Verifier}',\strat{\Refuter}}$
son iguales para toda estrategia $\strat{\Refuter}$, por lo tanto tenemos: 
$\inf_{\strat{\Verifier}} \Prob{\strat{\Verifier}}{\strat{\Refuter}}_{\StochG_{A,A'}, v^{\mathcal{H}}_0}(\Diamond \ErrorSt)
= \inf_{\strat{\Verifier}} \Prob{\strat{\Verifier}}{\strat{\Refuter}}_{\mathcal{H}_{A,A'}, \InitVertex}(\Diamond \ErrorSt)$.

Ahora bien, demostraremos que podemos verificar si el juego $\mathcal{H}$ falla casi-seguramente bajo \emph{fairness} o no utilizando el juego simbólico. Definimos así los siguientes conjuntos sobre este juego:
\begin{align*}
  \EFairpre(C) = {}&\{ v \in V^{\mathcal{H}} \mid\exists v' \in C : \langle v,v' \rangle \in E^\mathcal{H} \} \\
  \AFairpre(C) = {}&\{ v \in V^{\mathcal{H}}_\Probabilistic \mid \delta(v,C)>0\} \\
                       & \cup \{ v \in  V^{\mathcal{H}}_\Verifier \mid \forall v' {\in} V^{\mathcal{H}} : \langle v,v' \rangle \in E^{\mathcal{H}} \Rightarrow v' {\in} C \} \\
                     & \cup \{v \in V^{\mathcal{H}}_\Refuter \mid \exists v'{\in} V^{\mathcal{H}} : \langle v,v' \rangle \in E^\mathcal{H} \} 
\end{align*}

Como se demostró en \cite{CastroDDP22} (Teorema 3) tenemos que: 
 $\Prob{\strat{\Verifier}}{\strat{\Refuter}}_{\mathcal{H}_{A,A'},v}(\Diamond \ErrorSt) = 1$ para toda estrategia 
 $\strat{\Verifier} \in \Strategies{\Verifier}$ y estrategia fair $\strat{\Refuter} \in \Strategies{\Refuter}$
  si y solo si $v \in V\setminus \EFairpre^*(V \setminus \AFairpre^*(\{ \ErrorSt \}))$.
    
 Definimos una asociación $\zeta : V^\SymbG \rightarrow 2^{V^{\mathcal{H}}}$ como a continuación:
 \[\small
     \zeta(v) = 
                    \begin{cases*}
                         \{(\pr{0}{v},\pr{1}{v},\pr{2}{v},\pr{3}{v},\pr{4}{v},\mhyphen, \pr{5}{v} )\} & if  $v \in V^{\SymbG}_\Refuter \cup V^{\SymbG}_\Verifier$, \\
                         \{ (\pr{0}{v},\pr{1}{v},\pr{2}{v},\pr{3}{v},\pr{4}{v},w, \pr{5}{v}) \mid w \in \vertices{\couplings{\pr{4}{v}}{\pr{5}{v}}}\} & en caso contrario.
                    \end{cases*}
 \]   
 Observemos que para los vértices del Refutador y el Verificador la función $\zeta$ retorna un conjunto unitario.  Para vértices probabilistas retorna los vértices del politopo correspondiente.
 
  Ahora bien, demostraremos que para todo $v \in V^\SymbG$  tenemos que:  $\zeta(v) \subseteq  \AFairpre^n(\{ \ErrorSt \})$ si y solo si $v \in {\SymbAFairpre}^n(\{ \ErrorSt \})$
  La prueba es por inducción sobre $n$. El caso base es directo. El caso inductivo es por casos:
 
 \sloppy      Si $v$ es un nodo del Refutador, entonces  $\zeta(v) = \{(\pr{0}{v},\pr{1}{v}, \pr{2}{v}, \pr{3}{v}, \pr{4}{v},\mhyphen, \pr{5}{v})\}$. 
        Ahora bien, vamos a mostrar la parte ``si'', la otra dirección es similar. $v \in  {\SymbAFairpre}^n(\{ \ErrorSt \})$ si y solo si para algún $v' \in  {\SymbAFairpre}^{n-1}(\{ \ErrorSt \})$ (*) tenemos 
        $\langle v,v' \rangle \in E^{\SymbG}$ (**),  por inducción y (*) tenemos que 
        $(\pr{0}{v'},\pr{1}{v'}, \pr{2}{v'}, \pr{3}{v'},\pr{4}{v'},\mhyphen, \pr{5}{v'}) \in  {\SymbAFairpre}^{n-1}(\{ \ErrorSt \})$,
        y por definición de $E^{\mathcal{H}}$ y (**) obtenemos 
        $\langle (\pr{0}{v},\pr{1}{v},\pr{2}{v},\pr{3}{v},\pr{4}{v},\mhyphen, \pr{5}{v}),(\pr{0}{v'},\pr{1}{v'},\pr{2}{v'},\pr{3}{v'},\pr{4}{v'},\mhyphen,\pr{5}{v'}) \rangle \in E^{\mathcal{H}}$ por lo tanto $(\pr{0}{v},\pr{1}{v},\pr{2}{v},\pr{3}{v},\pr{4}{v},\mhyphen, \pr{5}{v}) \in \AFairpre^n(\{ \ErrorSt \})$, y entonces $\zeta(v) \subseteq \AFairpre^n(\{ \ErrorSt \})$
        
    %  \item If $v$ is a Verifier's node,  we have $\zeta(v) = \{(v[0],v[1],v[2],v[3],v[4],\mhyphen, v[5])\}$.  
     %  $v \in  {\SymbAFairpre}^n(\ErrorSt)$ iff for all $v' \in  {\SymbAFairpre}^{n-1}(\ErrorSt)$
   
        Si $v$ es un nodo del Verificador, también tenemos  $\zeta(v) = \{( \pr{0}{v},\pr{1}{v},\pr{2}{v},\pr{3}{v},\pr{4}{v},\mhyphen, \pr{5}{v} )\}$.  
        De manera similar a lo anterior, demostraremos solo la parte ``si'' ya que la otra dirección es análoga.       
        Si $v \in  {\SymbAFairpre}^n(\{ \ErrorSt \})$, entonces 
        para todo $(v,v') \in E^{\SymbG}$ tenemos $v' \in  {\SymbAFairpre}^{n-1}(\{ \ErrorSt \})$.  Ahora bien, vamos a probar que $\zeta(v) \subseteq  \AFairpre^n(\{ \ErrorSt \})$, 
        sea $\langle ( \pr{0}{v},\pr{1}{v},\pr{2}{v},\pr{3}{v},\pr{4}{v},\mhyphen, \pr{5}{v} ),u' \rangle \in E^{\mathcal{H}_{A,A'}}$, por definición de $E^{\SymbG}$
        tenemos que $\langle v,  (\pr{0}{u'},  \pr{1}{u'}, \pr{2}{u'}, \pr{3}{u'}, \pr{4}{u'}, \pr{6}{u'}) \rangle \in E^{\SymbG}$, por lo tanto por nuestro supuesto obtenemos que 
        $(\pr{0}{u'},  \pr{1}{u'}, \pr{2}{u'}, \pr{3}{u'}, \pr{4}{u'}, \pr{6}{u'}) \in  {\SymbAFairpre}^{n-1}(\{ \ErrorSt \})$, y por inducción, tenemos que
        $u' \in  \AFairpre^{n-1}(\{ \ErrorSt \})$,  y por definición de $\AFairpre$ obtenemos que 
        $( \pr{0}{v},\pr{1}{v},\pr{2}{v},\pr{3}{v},\pr{4}{v},\mhyphen, \pr{5}{v} ) \in \AFairpre^n(\{ \ErrorSt \})$, por lo tanto $\zeta(v) \subseteq  \AFairpre^n(\{ \ErrorSt \})$.
       
        Si $v$ es un nodo probabilista, una vez más solo probamos la parte ``si''.   Si $v \in  {\SymbAFairpre}^n(\{ \ErrorSt \})$, entonces
        $\Eq(v)({\SymbAFairpre}^{n-1}(\{ \ErrorSt \}))$ no tiene solución.  Ahora bien, consideremos 
        $(\pr{0}{v},  \pr{1}{v}, \pr{2}{v}, \pr{3}{v}, \pr{4}{v}, w,\pr{5}{v}) \in \zeta(v)$, observemos que no podemos tener 
        $\delta((\pr{0}{v},  \pr{1}{v}, \pr{2}{v}, \pr{3}{v}, \pr{4}{v}, w,\pr{5}{v}),  \AFairpre^{n-1}(\{ \ErrorSt \})) = 0$,  en caso contrario
        $w$ tendría solución para  $\Eq(v)({\SymbAFairpre}^{n-1}(\{ \ErrorSt \}))$. Por lo tanto,  
        $(\pr{0}{v},  \pr{1}{v}, \pr{2}{v}, \pr{3}{v}, \pr{4}{v}, w,\pr{5}{v}) \in  {\SymbAFairpre}^n(\{ \ErrorSt \})$. La parte ``solo si'' es similar.
        
       Ahora vamos a probar que, para cualquier conjunto $S \subseteq V^{\SymbG}$ y $ \bigcup \zeta(S) \subseteq S'$ (en particular, notemos que $\bigcup \zeta(V^{\SymbG}) = V^{\mathcal{H}}$)
        tenemos que: 
       $\zeta(v) \subseteq  \EFairpre^n( S' ) \neq \emptyset$ si y solo si $v \in {\SymbEFairpre}^n( S )$. Como hicimos arriba, la prueba es por inducción sobre $n$.
       El caso base es directo.  Para el caso inductivo procedemos por casos.
       
       Si $v$ es un nodo del Refutador demostramos la parte ``si''.  Asumamos que $v \in  {\SymbEFairpre}^n( S )$ por lo que existe un
       $(v, v') \in E^{\SymbG}$ tal que $v' \in {\SymbEFairpre}^{n-1}( S )$,  pero entonces tenemos que $\zeta(v') \subseteq \EFairpre^{n-1}( S' )$ y también (por definición de $E^{\mathcal{H}_{A,A'}}$) tenemos que 
       $\langle (\pr{0}{v},\pr{1}{v},\pr{2}{v},\pr{3}{v},\pr{4}{v},\mhyphen, \pr{5}{v}),(\pr{0}{v'},\pr{1}{v'},\pr{2}{v'},\pr{3}{v'},\pr{4}{v'},\mhyphen,\pr{5}{v'}) \rangle \in E^{\mathcal{H}_{A,A'}}$,
       como $\zeta(v) = \{(\pr{0}{v},\pr{1}{v},\pr{2}{v},\pr{3}{v},\pr{4}{v},\mhyphen, \pr{5}{v})\}$, por definición de $\EFairpre$ obtenemos que
       $\zeta(v) \subseteq  \EFairpre^n( S' )$. La otra dirección de la prueba es similar. La prueba para los nodos del Verificador es similar.
       
       Para el caso de $v$ siendo un nodo probabilista,   $v \in  {\SymbEFairpre}^n( S )$ si y solo si existe un
       $v' \in  {\SymbEFairpre}^{n-1}( S )$ tal que $\pr{0}{v'} \in \support{\pr{3}{v}}$ y $\pr{2}{v'} \in \support{\pr{4}{v}}$.
       por inducción tenemos que esto es equivalente a $\zeta(v') \subseteq  {\SymbEFairpre}^{n-1}( S' )$, y como todo nodo $u \in \zeta(v')$ satisface
       $\pr{0}{u} \in \support{\pr{3}{v}}$ y $\pr{2}{u} \in \support{\pr{4}{v}}$ obtenemos $\zeta(v) \subseteq \EFairpre^n( S' )$.
       
       
       Ahora, utilizamos las propiedades anteriores para probar el resultado. De la primer propiedad obtenemos que $\zeta({\SymbAFairpre}^*(\ErrorSt)) = \AFairpre^*(\ErrorSt)$ (esto se deduce de la definición de $\SymbAFairpre$ para vértices probabilistas),  
       por lo tanto $V^{\mathcal{H}} \setminus \AFairpre^*(\ErrorSt) \supseteq \bigcup \zeta(V^\SymbG \setminus {\SymbAFairpre}^*(\ErrorSt))$, entonces utilizando la segunda propiedad obtenemos que $\InitVertex \in V^{\mathcal{H}} \setminus \EFairpre^*(V^{\mathcal{H}} \setminus \AFairpre^*(\ErrorSt))$ si y solo si   
       $\InitVertex \in V^\SymbG \setminus {\SymbEFairpre}^*(V^\SymbG \setminus {\SymbAFairpre}^*(\ErrorSt))$ (observemos que $\zeta(\InitVertex)$ es un conjunto unitario).
\end{proof}
\fi
Como $\Eq(v,C)$ puede ser computado en tiempo polinomial, también es el caso para los conjuntos predecesores $\SymbEFairpre(C)$ y $\SymbAFairpre(C)$.  Consecuentemente, el problema de decidir si un juego de enmascaramiento estocástico falla casi-seguramente bajo \emph{fairness} también es polinomial.


\section{Experimentos}
\label{sec:experimental_eval_prob}
La Tabla~\ref{table:resultsMEM} reporta los resultados obtenidos para nuestro ejemplo de la memoria.
$M_{t}$ y $M_{r}$ son los resultados de medir las acciones de tick y refresh, consideradas como hitos respectivamente.

Algunas palabras son útiles para interpretar los resultados. Notemos que, tanto el incremento de la redundancia, como el aumento de la frecuencia de resfresco, tienen efectos positivos sobre las medidas. En la práctica, estos valores se pueden tener en cuenta a la hora de diseñar un componente tolerante a fallas que provea un balance óptimo entre eficiencia y costos de hardware. Por ejemplo, asumiendo una probabilidad de falla de $0.05$, uno podria preferir $3$ bits y un refresco mas frecuente, por sobre $5$ bits con menos refrescado,a pesar del overhead de software.


\begin{table}
  \centering
  \scalebox{0.7}{
    \begin{tabular}{c!{\ }c!{\ }|>{\ \,}c!{\ }c!{\ }|!{\ }c!{\ }c!{\ }|!{\ }c!{\ }c}
      %% & & \makebox[2.5em][l]{3 bits redund.} & & \makebox[2.5em][l]{5 bits redund.} & & \makebox[2.5em][l]{7 bits redund.} & \\ \hline
      %% Fault & Refresh & \multirow{2}{*}{$M_{t}$} & \multirow{2}{*}{$M_{r}$} & \multirow{2}{*}{$M_{t}$} & \multirow{2}{*}{$M_{r}$} & \multirow{2}{*}{$M_{t}$} & \multirow{2}{*}{$M_{r}$}  \\
      %% Prob. &  Prob.  & & & & & &  \\ \hline
      Prob. & Prob. & \makebox[2.5em][l]{3 redund. bits} & & \makebox[2.5em][l]{5 redund. bits} & & \makebox[2.5em][l]{7 redund. bits} & \\ \cline{3-8}
      Falla &  Refresh  & $M_{t}$ & $M_{r}$ & $M_{t}$ & $M_{r}$ & $M_{t}$ & $M_{r}$  \\ \hline
                  \multirow{3}{*}{$0.5$} 
                  & $0.5$ & $6$ & $3$
                          & $14$ & $7$
                          & $30$ & $15$\\ 
                  & $0.1$ & $4.4$ & $0.44$
                          & $7.28$ & $0.73$
                          & $10.74$ & $1.07$\\
                  & $0.05$ & $4.2$  & $0.21$
                           & $6.62$ & $0.33$
                           & $9.28$ & $0.46$\\ \hline
                  \multirow{3}{*}{$0.1$} 
                  & $0.5$ & $70.01$ & $35$
                          & $430.44$ & $215.22$
                          & $2606.16$ & $1303.08$\\ 
                  & $0.1$ & $30$ & $3$
                          & $70.01$ & $7$
                          & $150.04$ & $15$\\ 
                  & $0.05$ & $25$  & $1.25$
                           & $47.5$ & $2.37$
                           & $81.26$ & $4.06$\\ \hline
                  \multirow{3}{*}{$0.05$} 
                  & $0.5$ & $240.13$ & $120.07$
                          & $2676.52$ & $1338.26$
                          & $31418.29$ & $15709.16$\\ 
                  & $0.1$ & $80.01$ & $8$
                          & $260.11$ & $26.01$
                          & $801.08$ & $80.11$\\ 
                  & $0.05$ & $60$  & $3$
                           & $140.03$ & $7$
                           & $300.13$ & $15.01$\\ \bottomrule
    \end{tabular}
  }
  \vspace{-1.5ex}
%  \caption{Experimental results on a Redundant Memory Cell.}
  \caption{Resultados experimentales sobre la Memoria Probabilista}
  \label{table:resultsMEM}
\end{table}




Las Tablas~\ref{table:resultsNMR} y \ref{table:resultsNMR2} reportan resultados sobre dos casos de estudio adicionales: Redundancia N-Modular (NMR), un ejemplo estándar de tolerancia a fallas \cite{ShoomanBook}; y una arquitectura NMR procesador-memoria con N votantes \cite{KrishnaBook}.

Ya introdujimos NMR en el capítulo~\ref{cap:maskingMeasure}, NMR consiste de N módulos, consiste de N módulos que realizan una tarea independientemente, y cuyos resultados son procesados por un votante perfecto para producir una sola salida.
En este caso, estos módulos pueden exhibir un comportamiento inesperado con una cierta probabilidad, en cuyo caso van a dar una salida incorrecta. Los resultados para este caso de estudio son similares al caso de la celda de memoria cuando hay probabilidad $0$ de refrescado. 

El segundo caso de estudio consiste en N procesadores que dan como salida un valor a un módulo de memoria a través de N votantes. Tanto los votantes como los procesadores pueden dar como salida un valor incorrecto con una cierta probabilidad.
El experimento da como salida los mismos resultados si la probabilidad de falla de los votantes y procesadores son intercambiados. Esto sugiere que, dado que la probabilidad de un par procesador-votante falle se mantenga igual, el sistema es más tolerante cuando las fallas ocurren con una distribución equitativa sobre los votantes y procesadores.
%% An interesting fact,  corroborated by the results,  is that the system is more fault-tolerant when the faults occur uniformly on voters and processors.
Hemos ejecutado nuestros experimentos en una MacBook Air con un procesador 1.3 GHz Intel Core i5 y 
4 GB de memoria.

%\vspace*{-0.9cm}
\begin{table}
\centering
\scalebox{0.7}{
    \begin{tabular}{c!{\ }|!{\ }c|!{\ }c}
      Redundancia & Prob. Falla & $M_{t}$ \\ \hline
      \multirow{4}{*}{$3$}  & $0.5$ & $4$ \\ 
      & $0.3$ & $6.66$ \\
      & $0.1$ & $20$ \\ 
      & $0.05$ & $40$ \\ \cline{1-3}
      \multirow{4}{*}{$5$} & $0.5$ & $6$ \\ 
      & $0.3$ & $10$ \\
      & $0.1$ & $30$ \\ 
      & $0.05$ & $60$ \\ \cline{1-3}
      \multirow{4}{*}{$7$} & $0.5$ & $8$ \\
      & $0.3$ & $13.33$ \\
      & $0.1$ & $40$ \\ 
      & $0.05$ & $80$ \\ \cline{1-3}
      \multirow{4}{*}{$9$} & $0.5$ & $10$ \\
      & $0.3$ & $16.66$ \\
      & $0.1$ & $50$ \\ 
      & $0.05$ & $100$ \\ \cline{1-3}
%      \multirow{4}{*}{$11$} & $0.5$ & $12$ \\
%      & $0.3$ & $20$ \\
%      & $0.1$ & $60$ \\ 
%      & $0.05$ & $120$ \\ \bottomrule
      \hline
    \end{tabular}
    }
    \vspace{0.2cm}
    \caption{Resultados experimentales sobre un Sistema Redundante N-Modular.}
    \label{table:resultsNMR}
    \hfill
\end{table}


    \begin{table}
    \centering
    \vspace{0.6cm}
    \scalebox{0.7}{
    \begin{tabular}{c!{\ }|!{\ }c|!{\ }c|!{\ }c}
      Redundancia & P. Falla Proc. & P. Falla Vot. & $M_{t}$ \\ \hline
      \multirow{5}{*}{$3$}  & $0.09$  & $0.01$ & $21.8$ \\ 
      & $0.07$ & $0.03$ & $24.2$ \\ 
      & $0.05$ & $0.05$ & $25$ \\ 
      & $0.03$ & $0.07$ & $24.2$ \\ 
      & $0.01$  & $0.09$ & $21.8$ \\ \cline{1-4}
      \multirow{5}{*}{$5$} & $0.09$ & $0.01$ & $33.19$ \\ 
      & $0.07$ & $0.03$ & $38.95$ \\ 
      & $0.05$ & $0.05$ & $41.25$ \\
      & $0.03$ & $0.07$ & $38.95$ \\
      & $0.01$ & $0.09$ & $33.19$ \\  \cline{1-4}
      \multirow{5}{*}{$7$} & $0.09$ & $0.01$ & $44.39$ \\
      & $0.07$ & $0.03$ & $53.78$ \\ 
      & $0.05$ & $0.05$ & $58.13$ \\
      & $0.03$ & $0.07$ & $53.78$ \\
      & $0.01$ & $0.09$ & $44.39$ \\ \cline{1-4}
%      \multirow{5}{*}{$9$} & $0.09$ & $0.01$ & $55.54$ \\
%      & $0.07$ & $0.03$ & $68.59$ \\ 
%      & $0.05$ & $0.05$ & $75.39$ \\ 
%      & $0.03$ & $0.07$ & $68.59$ \\ 
%      & $0.01$ & $0.09$ & $55.54$ \\ \bottomrule
      \hline
    \end{tabular}
    }
    \vspace{0.2cm}
    \caption{Resultados experimentales sobre una arquitectura NMR procesador/memoria con N votantes.}
    \label{table:resultsNMR2}
    \vspace{-.5cm}
\end{table}
\section{Discusiones finales} 
\label{sec:final_discussions_prob}

%\section{Related Work} \label{sec:related_work}
%\paragraph{Related Work.}
Los juegos introducidos en \cite{Bacci0LM17,BacciBLMTB19,DesharnaisGJP04,DesharnaisLT11}  están basados en la noción de bisimulación probabilista, por lo que son simétricos. Además, en \cite{DesharnaisGJP04,DesharnaisLT11}, los vértices son modelados utilizando subconjuntos de estados de sistemas de transición probabilistas, 
observemos que nuestros juegos no utilizan subconjuntos de estados. Los juegos definidos en  \cite{Bacci0LM17,BacciBLMTB19} utilizan los \textit{liftings} de Kantorovich y Hausdorff para lidiar con distribuciones probabilistas y no-determinismo, respectivamente; mientras que los vértices de los politopos de transportación son utilizados como vértices probabilistas. En contraste, introducimos una representación simbólica de juegos para evadir la explosión de estados causada por los vértices del politopo. También notemos que las métricas introducidas en \cite{Bacci0LM17,BacciBLMTB19} miden la distancia de bisimulación (probabilista) entre sistemas de transición probabilistas, la cual es siempre $1$ para sistemas casi-seguro terminantes.
		
%% In \cite{LanotteMT17}, the authors introduce a notion of weak simulation quasimetric tailored for reasoning about the evolution of \emph{gossip protocols} in order to compare network protocols that have similar behaviour up to a certain tolerance; being $0$ and $1$ the minimum and maximum distance, respectively.
%% Note that using this quasimetric to compare a network protocol with an almost-sure failing implementation will always return $1$, thus that approach cannot be used to quantify the masking fault-tolerance of almost-sure failing systems.
%
\cite{LanotteMT17}~introduce una simulación débil quasi-simétrica para razonar sobre la evolución de los \emph{ protocolos gossip} para poder comparar protocolos de comportamiento similar hasta una cierta tolerancia.
Sin embargo, a la hora de comparar un protocolo de red con una implementación casi-seguro terminante, esta quasi-métrica siempre retorna $1$ (la máxima distancia) y por lo tanto, no puede cuantificar la tolerancia a fallas enmascarante de la misma forma que en el trabajo presentado aquí.


Las métricas como \emph{Mean-Time To Failure} (MTTF)~\cite{ReliabilityBook}  pueden aplicar para los ejemplos de la Sección~\ref{sec:experimental_eval_prob}.
%%Our framework is more general than such metrics. Indeed, we do not necessarily have to count time
Sin embargo, nuestro \textit{framework} es más general que tales métricas ya que no se limita a contar unidades de tiempo, otros eventos pueden ser considerados hitos.  Adicionalmente, el cómputo de MTTF normalmente requiere que se identifiquen los estados de falla de manera ad-hoc, mientras que las técnicas presentadas aquí nos permiten hacer esto en un nivel de abstracción más alto.
 %% level of abstraction: the failure situation appears in the game as a
 %% result of comparing the implementation model against the nominal
 %% model.




 
% The games introduced in \cite{Bacci0LM17,BacciBLMTB19,DesharnaisGJP04,DesharnaisLT11}  are based on probabilistic bisimulation, so they are symmetric. Furthermore, in 
%\cite{DesharnaisGJP04,DesharnaisLT11} the nodes of the game graph are modeled using subsets of states of the PTSs, in our formulation
%we do not use subsets of states. The games defined in \cite{Bacci0LM17,BacciBLMTB19} use Kantorovich's and Hausdorff's liftings to deal with probabilistic distributions and %non-determinism, respectively. In addition, the authors use the vertices of the transportation polytopes as probabilistic vertices. In contrast, we introduced a symbolic %representation of games to avoid the state explosion caused by the vertices of the polytopes. Also note that the metrics introduced in \cite{Bacci0LM17,BacciBLMTB19} %measure the (probabilistic) bisimulation distance between two PTSs, which for almost-sure failing systems is always $1$.
		
%Another related framework is  defined in \cite{LanotteMT17}. Therein, the authors introduce a notion of weak simulation quasimetric tailored for reasoning about the evolution of %\emph{gossip protocols}. This makes it possible to compare network protocols that have similar behaviour up to a certain tolerance; being $0$ and $1$ the minimum and %maximum distance, respectively.  Note that using this quasimetric to compare a network protocol with an almost-sure failing implementation will always return $1$, thus that %approach cannot be used to quantify the masking fault-tolerance of almost-sure failing systems.

%After the case studies of Section~\ref{sec:experimental_eval},
%\emph{Mean-Time To Failure} (MTTF)~\cite{ReliabilityBook} may come to
%mind.  Though this metric (lifted to games) may be the result of a
%particular case study, we present a much more general framework.
%Indeed, on the one hand, we do not necessarily have to count time
%units, and other events may be set as milestones.  On the other hand,
%the computation of MTTF would normally require the identification of
%failures states in an ad hoc manner, while we do this at a higher
%level of abstraction: the failure situation appears in the game as a
%result of comparing the implementation model against the nominal
%model.






%% Finally, let us compare our approach with a well-known metric in fault-tolerance, \emph{Mean-Time To Failure} (MTTF), which is the expected (or average) amount of time that a system 
%% performs successfully until it fails. In our framework, when we set ticks as milestones, we can calculate the mean time until a masking failure occur.
%% %These metrics are designed for hardware or electronic systems, where we have at hand estimations about the failure rate of the physical components. 
%% Our approach is designed to be used at a higher level of abstraction, where we have a model of the system to be implemented acting as a specification, and several possible implementations of it, described as probabilistic automata. %This level of abstraction makes it possible to use this framework to analyze software fault-tolerance. 
%% Notice that our framework is particularly tailored to deal with masking fault-tolerance, a particular kind of fault-tolerance.
%% In addition, our game formulation allows us to analyse systems on worst-case scenarios.
