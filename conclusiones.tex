\chapter{Conclusiones y Trabajos Futuros}
\label{cap:conclusiones}

En esta tesis presentamos una noción de distancia de masking-tolerancia a fallas
entre sistemas basada en una caracterización de masking-tolerancia a través de relaciones de simulación y una representación correspondiente basada en juegos con objetivos cuantitativos.
Nuestro marco es adecuado para apoyar a los ingenieros en el análisis y
diseño de sistemas tolerantes a fallas. Más precisamente, hemos definido un
función de distancia de enmascaramiento computable tal que un ingeniero
puede medir la tolerancia de enmascaramiento de un determinado
implementación tolerante a fallas, es decir, el número de fallas que se pueden enmascarar.
De este modo, el ingeniero puede medir y comparar la distancia de masking-tolerancia a fallas 
entre implementaciones tolerantes a fallas alternativas, y seleccione una que
se adapte mejor a sus preferencias.

En esta linea de trabajo hay muchas direcciones para el trabajo futuro. Sólo hemos definido un
noción de distancia de tolerancia a fallas para enmascarar la tolerancia a fallas,
nociones similares de distancia pueden definirse para otros niveles de
tolerancia a fallas como \emph{fail-safe} y \emph{non-masking}. Asimismo, nos hemos centrado
en modelos no cuantitativos. Sin embargo, las métricas definidas en probabilidad
modelos, donde la tasa de ocurrencia de fallas se representa explícitamente,
podría dar una noción más precisa de la tolerancia a fallos.

Por otro lado, también hemos investigado propiedades de juegos estocásticos con funciones de payoff de recompensas totales bajo la asunción de que el minimizador (es decir, el entorno) emplea solo estrategias justas al jugar.  %We have proved that determinacy is preserved for stopping games when fairness is assumed. 

Hemos demostrado que, en este escenario, se conserva la determinación y ambos jugadores tienen estrategias óptimas sin memoria y deterministas; además, el valor del juego se puede calcular aproximando un punto fijo mayor de un operador de Bellman. Solo hemos considerado recompensas no negativas en esta tesis. Una forma posible de extender los resultados presentados aquí a juegos con recompensas negativas es adaptar las técnicas presentadas en \cite{DBLP:conf/lics/Baier0DGS18} para MDPs con costos negativos, dejamos esto como un trabajo adicional.

 Para mostrar la aplicabilidad de nuestra técnica, hemos presentado dos ejemplos de aplicaciones y una validación experimental sobre diversas instancias de estos casos de estudio utilizando nuestra herramienta prototipo. Creemos que los supuestos de fairness permiten considerar un comportamiento más realista del ambiente.
%Como trabajo futuro, planeamos construir una herramienta más robusta y así modelar y evaluar otros casos de estudio.

No hemos investigado otras funciones comunes de payoff, como el discounted payoff o el limiting-average payoff. Un beneficio de estas clases de funciones es que el valor de los juegos está bien definido incluso cuando los juegos no se detienen.
A primera vista, la noción de fairness es poco relevante para los juegos con discounted payoff, ya que este tipo de funciones de pago toman la mayor parte de su valor de las partes iniciales de las ejecuciones. Para el limiting-average payoff, la situación es diferente, y las suposiciones de fairness pueden ser relevantes, ya que podrían cambiar el valor de los juegos, dejamos esto como trabajo adicional.