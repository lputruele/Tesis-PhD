\chapter{Conclusiones y Trabajos Futuros}
\label{cap:conclusiones}
In this paper, we  presented a notion of masking fault-tolerance 
distance between systems built on a characterization of masking 
tolerance via simulation relations and a corresponding game 
representation with quantitative objectives. 
Our framework is well-suited to support engineers for the analysis and 
design of fault-tolerant systems. More precisely, we have defined a 
computable masking distance function such that an engineer 
can measure the masking tolerance of a given 
fault-tolerant implementation, i.e., the number of faults that can be masked. 
Thereby, the engineer can measure and compare the masking fault-tolerance 
distance of alternative fault-tolerant implementations, and select one that 
fits best to her preferences.

There are many directions for future work.  We have only defined a
notion of fault-tolerance distance for masking fault-tolerance,
similar notions of distance can be defined for other levels of
fault-tolerance like failsafe and non-masking.  Also, we have focused
on non-quantitative models.  However, metrics defined on probabilistic
models, where the rate of fault occurrences is explicitly represented,
could give a more accurate notion of fault tolerance.