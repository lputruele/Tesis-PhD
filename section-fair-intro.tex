\section{Introducción} \label{sec:intro_fair}
La Teoría de Juegos \cite{MorgensternNeuman42}  ofrece una teoría matemática elegante y profunda. 
En las ultimas décadas, ha recibido gran atención de computólogos ya que tiene importantes aplicaciones en la verificación y síntesis de software. 
La analogía es atractiva, la operación de un sistema bajo un ambiente no cooperativo (hardware defectuoso, agentes maliciosos, canales de comunicación poco confiables, etc.) puede ser modelada como un juego entre dos jugadores (el sistema y el ambiente), en el cual el sistema trata de alcanzar ciertos objetivos, mientras que el ambiente pretende prevenir que esto suceda. 
Esta visión es particularmente útil para \emph{síntesis de controladores}, es decir, generación automática de políticas de toma de decisiones a partir de una especificación de alto nivel. 
Por lo tanto, sintetizar un controlador consiste de computar las estrategias óptimas para un juego dado.
		
En este capítulo nos enfocamos en juegos estocásticos de dos jugadores, suma-cero, por turnos y de información perfecta con recompensas (no negativas)\cite{FilarV96}. 
Intuitivamente, estos juegos son jugados en un grafo por dos jugadores que mueven un token por turnos. Algunos vértices son probabilistas, en el sentido que, si un token está puesto en un vértice probabilista, entonces el próximo vértice a donde se mueve el token se selecciona aleatoriamente. Además, los jugadores deben seleccionar sus movimientos utilizando estrategias. Se asocia una recompensa a cada vértice, el cuál se asume que no es negativo en este trabajo. El objetivo del Jugador $1$ es maximizar la cantidad esperada de recompensa recolectada a lo largo del juego, mientras que el Jugador $2$ intentará minimizar este valor. Esto es lo que~\cite{SvorenovaKwiatkowska16} denomina \emph{objetivo de recompensa total}.
Este tipo de juegos han demostrado ser útiles para razonar sobre diversas clases de sistemas, como vehículos autónomos, sistemas tolerantes a fallas, protocolos de comunicación, plantas de energía, etcétera.  %(See~\cite{SvorenovaKwiatkowska16} for concrete case studies.)
Particularmente, en este capítulo consideramos aquellos juegos en donde uno de los jugadores emplea estrategias \textit{fair}.

Las restricciones de \textit{fairness}, entendiéndolas como resoluciones justas o equitativas de acciones no deterministas, juegan un rol importante en la verificación de software y en síntesis de controladores.
En especial, las asunciones de \textit{fairness} sobre el ambiente hacen posible la verificación de propiedades de \textit{liveness} en sistemas abiertos. 
Varios autores han indicado la necesidad de asunciones de \textit{fairness} sobre el ambiente en el contexto de síntesis de controladores, e.g., \cite{DBLP:conf/fossacs/AsarinCV10, DBLP:conf/icse/DIppolitoBPU11}.
Como un ejemplo simple, consideremos un vehículo autónomo que necesita atravesar un campo donde ciertos objetos en movimiento pueden interferir en su camino. Aunque se puede desconocer el comportamiento preciso de los objetos, es razonable suponer que no obstruirán continuamente los intentos del vehículo por evitarlos. En este sentido, mientras que el comportamiento estocástico puede ser una consecuencia de las fallas del vehículo, solo podemos asumir un comportamiento \textit{fair} de los objetos en movimiento circundantes.
        %% As a simple example consider a communication protocol that transmits bits between two or  more processes via an unreliable channel. 
	%% Clearly, no protocol can guarantee the transmission of information if the environment (es decir, the channel) always loses messages.  Similarly, and to provide more examples, we can assume that a processing module will never receive an infinite number of jobs within a time unit, or that a repairable piece of hardware will not be breaking continuously preventing the development of a particular task, etc.
En este trabajo, consideramos juegos estocásticos en los que se supone que uno de los jugadores (el que hace el papel de entorno) juega solo con estrategias \textit{fair} fuertes.

%% \remarkPC{Los ejemplos que damos aca podrian solucionarse si las fallas tienen probabilidades, habría que buscar un ejemplo que muestre la necesidad de fairness aun con randomized environments.}	
%        In all of these cases, it is sensible to assume that the environments show a \emph{strong fair} behaviour.
 %       In this work, we consider stochastic games in which one of the players (the one playing the environment) is assumed to play only with strong fair strategies.
%	\textcolor{red}{This is true even for randomized controllers. For example, consider a randomized arbiter (as the one described in \cite{DBLP:conf/ifm/Baier10}), the arbiter 
%	grants the access to a mutually exclusive resource to two processes running in an asynchronous manner. When both processes want to access to the resource,  the arbiter tosses an unbiased coin to select one of the competing processes. 
%	The protocol cannot guarantee that both  processes will eventually have access to the resource, unless  fairness assumptions are made over the scheduling policy.
%	In these kinds of scenarios it is sensible to assume that the environments show a \emph{strong fair} behaviour.
%        In this work, we consider stochastic games in which one of the players (the one playing the environment) is assumed to play only with strong fair strategies.
%	}
%	As a simple example consider a communication protocol that transmits bits between two or  more processes via an unreliable channel. 
%	Clearly, no protocol can guarantee the transmission of information if the environment (es decir, the channel) always loses messages.  
%        In this sense, it is reasonable to expect that not every message that is sent will be eventually lost.
%        Similarly, and to provide more examples, we can assume that a processing module will never receive an infinite number of jobs within a time unit, or that a repairable piece of hardware will not be breaking continuously preventing the development of a particular task, etc.
%        In all of these cases, it is sensible to assume that the environments show a \emph{strong fair} behaviour.
 %       In this work, we consider stochastic games in which one of the players (the one playing the environment) is assumed to play only with strong fair strategies.
Para garantizar que el valor esperado de las recompensas acumuladas esté bien definido en jugadas (quizás infinitas), se necesita algún tipo de criterio de parada. Una forma común de hacer esto es obligar a las estrategias a decidir detenerse con alguna probabilidad positiva en cada decisión. Esto corresponde a los llamados juegos estocásticos descontados o \textit{discounted}~\cite{Shapley1095,FilarV96}, y tiene las implicaciones de que las recompensas recolectadas pierden importancia a medida que avanza el juego (la ``reducción de importancia'' viene dada por el factor de descuento). Alternativamente, uno puede estar interesado en conocer la recompensa \emph{total} esperada, es decir, la recompensa acumulada esperada \emph{sin} ninguna pérdida de la misma a medida que avanza el tiempo. Para que este valor esté bien definido, el juego en sí debe detenerse (ser terminante). Es decir, sin importar las estrategias que jueguen los jugadores, la probabilidad de llegar a un estado terminal debe ser $1$~\cite{Condon90,FilarV96}.
	%% In this work we consider stochastic games in which we have fairness assumptions over one of the players (the environment). Usually, some stopping criteria is used for 
	%% guaranteeing that the expected value of total reward is well-defined in (perhaps infinite) plays. This can be done in different ways, either using discounted rewards functions \cite{Shapley1095,FilarV96} or 
	%% assuming that a \emph{terminal} state will be reached with probability $1$ \cite{Condon90,FilarV96}.
Nos centramos en este último tipo de juegos. Sin embargo, aquí estudiamos juegos que pueden no detenerse en general (es decir, para cualquier estrategia), sino que requieren que sean terminantes solo cuando el minimizador juega de manera \textit{fair}.
	%% We study here games that may be not stopping in general, but they become stopping when the minimizer plays in a fair way.
Usamos una noción de \textit{fairness} fuerte (\emph{casi-segura}), principalmente siguiendo las ideas presentadas en~\cite{DBLP:journals/dc/BaierK98} para los procesos de decisión de Markov. Mostramos que este tipo de juegos están determinados, es decir, cada estado del juego tiene un valor definido. Además, mostramos que existen estrategias óptimas sin memoria y deterministas para ambos jugadores. Además, el valor del juego se puede calcular a través del mayor punto fijo de los funcionales correspondientes.
Es importante señalar que la mayoría de las propiedades discutidas en este capítulo se cumplen cuando se hacen los supuestos de \textit{fairness} sobre el minimizador. Es posible que propiedades similares no se mantengan si se cambia el rol de los jugadores.
Sin embargo, estas condiciones abarcan una gran clase de escenarios, donde el sistema pretende maximizar la recompensa total recolectada y el entorno tiene el objetivo opuesto.

%\paragraph{Contributions.} 
En resumen, las contribuciones de este capítulo son las siguientes: (1) introducimos la noción de juego estocástico terminante bajo \textit{fairness}, una generalización de los juegos terminantes que tiene en cuenta entornos \textit{fair}; (2) probamos que se puede decidir en tiempo polinomial si un juego es terminante bajo \textit{fairness}; (3) mostramos que este tipo de juegos están determinados y que ambos jugadores poseen estrategias estacionarias óptimas, que se pueden calcular usando las ecuaciones de Bellman; y (4) implementamos estas ideas en una herramienta prototipo integrada en el conjunto de herramientas {\PrismGames}~\cite{DBLP:conf/cav/KwiatkowskaN0S20}, que usamos para evaluar la viabilidad de nuestro enfoque a través de casos de estudio ilustrativos.

%\paragraph{Structure.}


%%%%%%%%%%%%%%%%%%%%%%%%%%%%%%%%%%%%%%%%%%%%%%%%%%%%%%%%%%%%%%%%%%%%%%%%%
%% PARA LA POSIBLE VERSION FINAL:
%% Considerar revertir este p\'arrafo al que est\'a comentado debajo
%% dado que lo mutil\'e un poco para reducir espacio.
%%%%%%%%%%%%%%%%%%%%%%%%%%%%%%%%%%%%%%%%%%%%%%%%%%%%%%%%%%%%%%%%%%%%%%%%%
El capítulo está estructurado de la siguiente manera. La Sección \ref{sec:mot_example_fair} presenta un ejemplo ilustrativo para motivar el uso de restricciones de \textit{fairness} sobre el minimizador.
En la Sección \ref{sec:stopping_fair} describimos un procedimiento polinomial para verificar si un juego es terminante bajo \textit{fairness},
también demostramos que la determinación se conserva en estos juegos, así como la existencia de estrategias óptimas (sin memoria y deterministas).
Los resultados experimentales se describen en la Sección  \ref{sec:experimental_eval_fair}.
Finalmente, la Sección \ref{sec:related_work_fair} analiza el trabajo relacionado.

%% The paper is structured as follows. Section \ref{sec:mot_example} introduces an illustrating example to motivate the usefulness of having fairness restrictions over the minimizer on stochastic games. Section \ref{sec:background} fixes terminology and introduces background concepts. 
%% In Section \ref{sec:fair-strats} we describe a polynomial procedure to check whether a game stops under fairness assumptions, 
%% we also prove that determinacy is preserved in these games as well as the existence of (memoryless and deterministic) optimal strategies. 
%% Experimental results are described in Section \ref{sec:experimental_eval}. 
%% Finally, in Sections \ref{sec:related_work} and  \ref{sec:conclusions} we discuss related work and draw some conclusions, respectively.  Due to space constraints,  full proofs are gathered in the Appendix.