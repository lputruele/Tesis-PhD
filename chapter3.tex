\chapter{Medidas para Masking-tolerancia a fallas}
\label{cap:maskingMeasure}

Introducimos una noción de distancia de tolerancia a fallas entre sistemas de transición etiquetados. Intuitivamente, esta noción de distancia mide el grado de tolerancia a fallas exhibido por un sistema candidato.
En la práctica, hay diferentes tipos de tolerancia a fallas, aquí nos restringimos al análisis de masking-tolerancia a fallas ya que suele ser una característica muy deseada para sistemas críticos.
En términos generales, un sistema es masking-tolerante a fallas cuando es capaz de enmascarar las fallas completamente, sin permitir que estas fallas tengan consecuencias observables para los usuarios.
Capturamos la masking-tolerancia a fallas mediante una relación de simulación, acompañada por una caracterización correspondiente en términos de juegos.
Enriquecemos además estos juegos con objetivos cuantitativos para así definir la noción de distancia de masking-tolerancia a fallas.
Además, investigamos las propiedades básicas de esta noción de distancia de masking y probamos que es una semi-métrica dirigida.
Demostramos que, en el caso de sistemas deterministas, computar la distancia de masking se puede lograr utilizando un algoritmo de camino más corto.
Por otro lado, para sistemas no deterministas este cómputo se realiza por medio de técnicas de punto fijo. % No se como traducir approach
Estos algoritmos fueron implementados en una herramienta que computa de manera automática la distancia de masking entre un sistema nominal y una versión del mismo que agrega fallas (y mecanismos de tolerancia). Hemos evaluado su desempeño y efectividad sobre varios casos de estudio de diferentes complejidades.

\section{Introducción} \label{sec:intro}

    La tolerancia a fallas es una característica importante del software crítico, y puede ser definida como la capacidad de un sistema para lidiar con eventos inesperados, que pueden ser causados por bugs de programación, interacciones con un ambiente poco cooperativo, mal funcionamiento de hardware, etcétera.

Se pueden encontrar ejemplos de sistemas tolerantes a fallas en casi cualquier parte: protocolos de comunicación, circuitos de hardware, sistemas aviónicas, cripto-monedas, etcétera.

Por lo tanto, el incremento en la relevancia del software crítico en la vida cotidiana ha llevado a que se renueve el interés en la verificación automática de propiedades de tolerancia a fallas. Sin embargo, una de las dificultades principales a la hora de razonar sobre estos tipos de propiedades se da en su naturaleza cuantitativa, lo cual vale incluso en sistemas no probabilistas.
Un ejemplo simple se da con la introducción de redundancia en sistemas críticos. Esta es, sin lugar a dudas, una de las técnicas más utilizadas en tolerancia a fallas.
En la práctica, se sabe que al añadir más redundancia en un sistema incrementa su fiabilidad. Medir este incremento de fiabilidad es un problema central a la hora de evaluar software tolerante a fallas. Por otro lado, no hay un método \emph{de-facto} para caracterizar formalmente propiedades tolerantes a fallas, y por ello se suelen codificar utilizando mecanismos \emph{ad-hoc} como parte del diseño general.

Usualmente el flujo del diseño y verificación de sistemas tolerantes a fallas consiste en definir un modelo nominal (i.e. el programa ``sin fallas'' o ``ideal'') y luego extenderlo con comportamientos defectuosos que se desvían del comportamiento normal prescrito por el modelo nominal.

Este modelo extendido representa la manera en la que el sistema opera bajo la ocurrencia de fallas.
Hay diferentes maneras de extender el modelo nominal, el enfoque típico involucra la \emph{inyección de fallas}  \cite{HsuehTI97,IyerNGK10}, es decir, la introducción automática de fallas en el modelo. Una propiedad importante que cualquier modelo extendido debería satisfacer es la preservación del comportamiento normal ante la ausencia de fallas.

En \cite{DemasiCMA17} se propone un enfoque formal alternativo para tratar con el análisis de la tolerancia a fallas. Este enfoque permite un análisis totalmente automático y distingue apropiadamente comportamientos defectuosos y normales. Además, este framework es sensible a la inyección de fallas. En ese trabajo se definen tres nociones de relaciones de simulación para caracterizar diferentes tipos de tolerancia a fallas:
tolerancia \emph{masking}, \emph{nonmasking}, y \emph{failsafe}, originalmente definidas en \cite{Gartner99}. 

    Por otro lado, en los últimos años, se ha logrando un progreso significativo en pos de definir métricas o distancias apropiadas para diversos tipos de modelos cuantitativos, incluyendo sistemas de tiempo real \cite{HenzingerMP05}, modelos probabilistas \cite{DesharnaisGJP04}, y métricas para sistemas lineales y ramificados \cite{CernyHR12,AlfaroFS09,Henzinger13,LarsenFT11,ThraneFL10}. 

Algunos autores han resaltado que estas métricas pueden ser útiles para razonar sobre la robustez de un sistema, un concepto relacionado con la tolerancia a fallas. Particularmente, en \cite{CernyHR12}, la noción tradicional de relación de simulación se generaliza, y se introducen tres distancias de simulación entre sistemas, concretamente \emph{correctitud}, \emph{cobertura}, y \emph{robustez}.
A estas distancias se las define utilizando juegos cuantitativos con objetivos \emph{discounted-sum} y \emph{mean-payoff}.
%In this article\footnote{This is a revised and expanded version of a conference paper presented at TACAS 2019 \cite{CastroDDP18b}. A detailed description of the extension is given in Section  \ref{sec:related_work}.}, 
En \cite{CastroDDP18b} introducimos una noción de distancia de tolerancia a fallas entre LTS's. Intuitivamente, esta distancia mide el grado de tolerancia a fallas exhibido por un sistema candidato. Como fue mencionado anteriormente, existen varios niveles de tolerancia a fallas, aquí nos restringimos al análisis de \emph{masking-tolerancia a fallas} ya que usualmente se la considera como el tipo de tolerancia mas benigno y por lo tanto es una propiedad altamente deseable en cualquier sistema critico.
%\cite{}.  %% ADD REFERENCE

A grandes rasgos, un sistema es masking-tolerante a fallas cuando es capaz de enmascarar completamente las fallas, no permitiendo que las mismas tengan consecuencias observables para los usuarios. Formalmente, el sistema debe preservar tanto las propiedades de safety como las de liveness del modelo nominal \cite{Gartner99}. A diferencia de la distancia de robustez definida en \cite{CernyHR12}, la cual mide cuantos errores inesperados son tolerados por una implementación, aquí consideramos una colección especifica de fallas dadas en la implementación y medimos cuantas fallas son toleradas por la implementación de tal manera que puedan ser enmascaradas por los estados del sistema.

También requerimos que el comportamiento normal de la especificación se preserve en la implementación cuando no hay fallas. %Otherwise, the distance is $1$.
Formalmente, dado un sistema nominal $N$ y su implementación $I$ y sea $\delta$ nuestra función de distancia, tenemos que $\DeltaMask(N,I)=1$ si y solo si el modelo nominal $N$ y $I\backslash F$ no son bisimilares, donde $I\backslash F$ se comporta como la implementación $I$ cuando todas las acciones en $F$ están deshabilitadas ($\backslash$ es el operador de restricción de  Milner).
Por lo tanto, distinguimos efectivamente entre el modelo nominal, su versión tolerante a fallas  y el conjunto de fallas para el sistema en cuestión.

Para poder medir el grado de masking-tolerancia a fallas de un sistema dado, empezamos caracterizando la masking-tolerancia a fallas por medio de relaciones de simulación entre dos sistemas, como se define en \cite{DemasiCMA17}. El primero cumple el rol de especificación del comportamiento deseado (i.e el modelo nominal) y el segundo tiene el rol de implementación tolerante a fallas (i.e. el modelo extendido con fallas y mecanismos de tolerancia).
La existencia de una relación de masking implica que la implementación enmascara las fallas. Luego, introducimos una caracterización de la simulacion de masking en términos de juegos y enriquecemos los mismos con objetivos cuantitativos para definir así la noción de \emph{distancia de masking-tolerancia a fallas}, donde los valores posibles del juego pertenecen al intervalo  $[0,1]$. La implementación tolerante a fallas es masking-tolerante a fallas si el valor del juego es $0$. Además, mientras mayor el valor, mas lejos esta la implementación de la especificación en términos de la distancia de masking-tolerancia a fallas. De esta forma, una distancia mayor decrementa notablemente el grado de tolerancia a fallas.

Entonces, para un modelo nominal $N$ y dos implementaciones tolerantes a fallas diferentes $I_1$ y $I_2$, nuestra distancia asegura que  $\DeltaMask(N,I_1)<\DeltaMask(N,I_2)$ cuando $I_1$ tolera mas fallas que $I_2$.

También proporcionamos una versión débil de la simulación de masking, la cual hace posible tratar con sistemas mas complejos compuestos por varios componentes que interactúan entre si. Probamos que la distancia de masking es una semi-métrica dirigida, es decir, que satisface dos propiedades básicas de cualquier distancia: reflexividad y la desigualdad triangular.

Finalmente, hemos implementado nuestra técnica en una herramienta que toma como argumentos un modelo nominal y su implementación tolerante a fallas, y computa automáticamente la distancia de masking entre estos.
Hemos utilizado esta herramienta para medir la tolerancia de masking en múltiples instancias de varios casos de estudio: una celda de memoria redundante, una variación del problema de los filósofos comensales, el protocolo de comunicación BRP (Bounded Retransmission Protocol), redundancia N-Modular, el problema de los generales bizantinos y un subproblema del protocolo de consenso Raft para lograr una replicación consistente de datos.

Todos los casos de estudio mencionados son ejemplos típicos de sistemas tolerantes a fallas. Un punto interesante sobre nuestra implementación es que, para sistemas deterministas, la distancia de masking entre dos sistemas puede ser computada recurriendo a un algoritmo de camino mas corto. Mientras que en el caso de sistemas no deterministas, se aplica un algoritmo de punto fijo basado en la búsqueda a lo ancho (breadth first search), lo cual es menos eficiente, sin embargo en ambos casos el algoritmo es polinomial.


%The remainder of the paper is structured as follows. 
%In Section \ref{sec:background}, we introduce preliminaries notions used 
%throughout this paper.
%We present in Section \ref{sec:masking_dist} the definition of strong (resp. weak) masking simulation 
%and the corresponding game characterization. 
%In Section \ref{sec:QuantMask}, we present the formal definition 
%of masking distance built on quantitative simulation games, the algorithms 
%to compute it, and we also prove its basic properties. 
%We describe in Section \ref{sec:experimental_eval} the experimental 
%evaluation on some well-known case studies.
%In Section \ref{sec:related_work} we discuss the related work. 
%Finally, we discuss in Section \ref{sec:conclusions} some conclusions 
%and directions for further work.
%
%Full details and proofs can be found in \cite{CastroDDP18}.


\section{Preliminares} \label{sec:background}

Vamos a introducir algunas definiciones y resultados básicos sobre teoría de juegos que serán necesarios a lo largo del articulo, el lector interesado puede ver \cite{AptG11}.

Un sistema de transición etiquetado (o TS) es una tupla $A =\langle S, \Sigma, \rightarrow, \InitState \rangle$, donde $S$ es un conjunto finito de estados, $\Sigma$ es un alfabeto finito,  $\rightarrow \subseteq S \times \Sigma \times S$ es un conjunto de transiciones etiquetadas, y $s_0$ es el estado inicial. A partir de ahora, denotaremos $s \xrightarrow{e} s'$ en lugar de $(s,e,s') \in \rightarrow$ cuando creamos conveniente.

Con $|S|$ y $|{\rightarrow}|$ denotamos la cantidad de estados y arcos, respectivamente. Definimos $\post(s) = \{s' \in S \mid s \xrightarrow{e} s' \}$ como el conjunto de sucesores de $s$. Similarmente,  $\pre(s') = \{s \in S \mid s \xrightarrow{e} s' \}$ denota el conjunto de predecesores de $s'$.
Además, $\post^{*}(s)$ denota los estados alcanzables desde $s$.
Sin perder generalidad, requerimos que todo estado $s$ tenga un sucesor, i.e., $\forall s \in S : \post(s) \neq \emptyset$. Decimos que un TS es \emph{determinista} si, para toda terna de estados  $s,t, t'$ y para toda etiqueta $e$, $s \xrightarrow{e} t$ y $s \xrightarrow{e} t'$ implica $t=t'$. 

Una \emph{corrida} en un sistema de transición de estados $A$ es un camino infinito $\rho = \rho_0 \sigma_0 \rho_1 \sigma_1  \rho_2 \sigma_2 \dots \in (S \cdot \Sigma)^{w}$ 
donde $\rho_0 = \InitState$ y para todo $i$, $\rho_i \xrightarrow{\sigma_i} \rho_{i+1}$. De ahora en mas, dada una tupla $(x_0,\dots,x_n)$, denotaremos a $x_i$ con  $\pr{i}{x_0,\dots,x_n}$.

Un \emph{grafo de juego} $G$ es una tupla $G = \langle V, V_1, V_2, E, \InitVertex \rangle$ donde $V$ es un conjunto de vértices, $E\subseteq V \times V$ es un conjunto de arcos, $\InitVertex$ es el vértice inicial del juego, y $(V_1, V_2)$ es una partición de $V$. Intuitivamente, el grafo representa un juego entre dos jugadores, donde un token se coloca inicialmente en el vértice inicial, y luego sigue por rondas. Si el token esta en un vértice $v \in V_1$, entonces el jugador $1$ elige un arco $(v, w) \in E$ y mueve el token al vértice $w$. En caso de que el token este en un vértice $v \in V_2$, entonces la elección es del jugador $2$.

Un \emph{grafo de juego con costos} es un grafo de juego mas una función de costo (o recompensa) $\reward^G:V \rightarrow \mathbb{R}$. Un camino maximal en el grafo de juego $G$ se denomina una \emph{jugada}. El conjunto de todas las jugadas se denota con $\Omega$. Los \emph{grafos de juego deterministas} se definen de la misma manera que en sistemas de transición de estados. Además, utilizaremos la notación $\post(v)$ para denotar los sucesores del vértice $v$, y $\pre(v)$ para denotar a los predecesores del vértice $v$.

Dado un grafo de juego $G$, una \emph{estrategia} para el jugador $1$ es una función $\pi: V^{*} V_1 \rightarrow V$ tal que para todo  $\rho_0  \rho_1 \dots \rho_i \in V^{*} V_1$, tenemos que si $\pi(\rho_0  \rho_1\dots \rho_i) = \rho $, entonces $(\rho_i, \rho) \in E$. Una estrategia para el jugador $2$ se define similarmente. El conjunto de todas las estrategias para el jugador $p$ (para $p \in \{1,2\}$) se denota por $\Pi_{p}$.
Una estrategia para el jugador $p$ se dice \emph{sin memoria} (o posicional) si puede ser definida por un mapping $f:V_p \rightarrow V$, de tal forma que si $v \in V_p$ entonces $(v, f(v)) \in E$.
A grandes rasgos, estas estrategias no necesitan memoria del historial de movimientos. Además, una jugada $\rho_0 \rho_1 \rho_2 \dots$ se conforma a una estrategia $\pi$ del jugador $p$ si $\forall i \geq 0: (\rho_i \in V_p) \Rightarrow  \pi(\rho_0 \rho_1 \dots \rho_i) = \rho_{i+1}$. El \emph{resultado} de una estrategia $\pi_{1}$ del jugador $1$ y una estrategia $\pi_{2}$ del jugador $2$ es la jugada única, denominada $\out(\pi_1, \pi_2)$, que se conforma tanto a $\pi_1$ como a $\pi_2$.

Un \emph{juego} se compone de un grafo de juego y un objetivo booleano o cuantitativo. Un \emph{objetivo booleano} es una función $\Phi: \Omega \rightarrow \{0, 1\}$ y la meta del jugador $1$ en un juego con un objetivo $\Phi$ es seleccionar una estrategia de tal forma que el resultado del juego sea $1$, independientemente de lo que haga el jugador $2$. Por el contrario, la meta del jugador $2$ es asegurarse que el resultado sea $0$. Dado un objetivo booleano $\Phi$, una jugada $\rho$ es \emph{ganadora} para el jugador $1$ (resp. jugador $2$) si  $\Phi(\rho) = 1$ (resp. $\Phi(\rho) = 0$). Una estrategia $\pi$ es una \emph{estrategia ganadora} para el jugador $p$ si toda jugada que se conforme a $\pi$ es ganadora para el jugador $p$. Decimos que un juego con objetivo booleano esta \emph{determinado} si algún jugador tiene una estrategia ganadora, y decimos que está \emph{determinado sin memoria} si esa estrategia ganadora es sin memoria. Los juegos de alcanzabilidad son aquellos juegos cuyos objetivos están definidos como $\Phi(\rho_0 \rho_1 \rho_2 \dots) = (\exists i : \rho_i \in B)$ para algún conjunto $B \subseteq V$, un resultado estándar de los juegos de alcanzabilidad es que son determinados sin memoria.

Un \emph{objetivo cuantitativo} esta dado por una función de \emph{payoff} $f: \Omega \rightarrow \mathbb{R}$ y la meta del jugador $1$ es maximizar el valor $f$ de la jugada, mientras que la meta del jugador $2$ es minimizar este valor. Para un objetivo cuantitativo $f$, el valor del juego para la estrategia $\pi_1$ del jugador $1$, denotado por $v_1(\pi_1)$, se define como el ínfimo sobre todos los valores resultantes de estrategias del jugador $2$, i.e., $v_1(\pi_1) = \inf_{\pi_2 \in \Pi_2} f(\out(\pi_1, \pi_2))$. El valor del juego para el jugador $1$ se define como el supremo de los valores de todas las estrategias del jugador $1$, i.e., $\sup_{\pi_1 \in \Pi_1} v_1(\pi_1)$. Análogamente, el valor del juego para una estrategia $\pi_2$ del jugador $2$ y el valor del juego para el jugador $2$ se definen como $v_2(\pi_2) = \Sup_{\pi_1 \in \Pi_1} f(\out(\pi_1, \pi_2))$ 
y $\inf_{\pi_2 \in \Pi_2} v_2(\pi_2)$, respectivamente.
Una estrategia es \emph{optimal} para un jugador si el valor de la estrategia para ese jugador es igual al valor del juego para ese jugador. Decimos que un juego esta determinado si los valores del juego para ambos jugadores es igual, es decir:
$\sup_{\pi_1 \in \Pi_1} v_1(\pi_1) = \inf_{\pi_2 \in \Pi_2} v_2(\pi_2)$. En este caso denotamos con $\val(\mathcal{G})$ al valor del juego $\mathcal{G}$.
    El siguiente resultado de \cite{Martin98} caracteriza un conjunto grande de juegos determinados.
    \begin{thm} Cualquier juego con una función cuantitativa $f$ que esté acotada y sea Borel-medible esta determinado.
\end{thm}


\section{Simulación de Masking} \label{sec:masking_dist}
Empezamos definiendo la simulación de masking. En \cite{DemasiCMA17}, se definió una simulación basada en estados para la masking-tolerancia a fallas, aquí reutilizamos esta definición usando sistemas de transición etiquetados. Primero, vamos a introducir algunos conceptos que son necesarios para definir masking-tolerancia a fallas. Para todo vocabulario $\Sigma$, y todo conjunto de etiquetas $\Faults = \{F_0, \dots, F_n\}$ no pertenecientes a $\Sigma$, consideramos 
$\SigmaF = \Sigma \cup \Faults$, donde $\Faults \cap \Sigma = \emptyset$. Intuitivamente, los elementos de $\Faults$ indican la ocurrencia de una falla en una implementación defectuosa. Además, a veces sera útil considerar el conjunto $\Sigma^i = \{ e^i \mid e \in \Sigma\}$, que contiene los elementos de $\Sigma$ indexados con el superíndice $i$.

%, $\Sigma^i = \{\sigma^i \mid \sigma \in \Sigma\}$.

\subsection{Simulación de Masking fuerte}
\begin{defi} \label{def:masking_rel}
  Sean $A =\langle S, \Sigma, \rightarrow, s_0\rangle$ y $A' =\langle S', \SigmaF, \rightarrow', s_0' \rangle$ dos sistemas de transición etiquetados.
  $A'$ es \emph{fuertemente masking-tolerante a fallas} con respecto a $A$ si existe una relación 
$\M \subseteq S \times S'$ entre $A$ y $A'$ tal que:

\begin{enumerate}[(A)]
  \item $s_0 \M s'_0$, y
  \item para todo $s \in S, s' \in S'$ con $s \M s'$ y todo $e \in \Sigma$ vale lo siguiente:

  \begin{enumerate}[(1)]
    \item 
    si $s \xrightarrow{e} t$ entonces
    $\exists\; t' \in S': s' \xrightarrowprime{e} t'  \wedge t \M t'$;

      \item si $s' \xrightarrowprime{e} t'$ entonces
      $\exists \; t \in S: s \xrightarrow{e} t \wedge t \M t'$;

      \item si $s' \xrightarrowprime{F} t' $ para algún $F \in \Faults$ entonces
      $s \M t'$.
      
  \end{enumerate}
\end{enumerate}

Si tal relación existe decimos que $A'$ es una \emph{implementación fuertemente masking-tolerante a fallas} de $A$, denotado como $A \Masking A'$. 
\end{defi}

 Intuitivamente, la definición establece que, partiendo de $s'$, las fallas pueden ser enmascaradas de tal forma que el comportamiento exhibido es el mismo que el observado al partir de $s$ ejecutando transiciones sin fallas. 
 En otras palabras, una relación de masking asegura que cualquier comportamiento defectuoso en la implementación puede ser simulado por la especificación. Mas específicamente, note que las condiciones (A), (B.1) y (B.2) implican que tenemos una bisimulación cuando $A$ y $A'$ no exhiben comportamientos defectuosos.
Particularmente, la condición (B.1) dice que la ejecución normal de $A$ puede ser simulada por una ejecución de $A'$. Por otro lado, la condición (B.2) dice que la implementación no agrega mas comportamientos normales (no defectuosos). Por ultimo, la condición (B.3) establece que toda transición defectuosa ($F$) que salga de $s'$ debe ser correspondida por un movimiento en el mismo lugar desde $s$.

\subsection{Simulación de Masking Débil}

Para analizar sistemas no triviales se necesita una relación de simulación de masking débil. La idea principal es que una simulación de masking débil se abstrae del comportamiento interno, el cual es modelado por una acción especial $\tau$. Note que las transiciones internas son comunes en tolerancia a fallas: las acciones ejecutadas como parte de un procedimiento tolerante a fallas en un componente usualmente no son observables para el resto del sistema.

Las \textit{relaciones de transición débiles} ${\Rightarrow} \subseteq S
\times (\Sigma \cup \{\tau\} \cup \Faults) \times S$ consideran el paso \emph{silencioso} $\tau$ que se define como sigue: 

\[
\xRightarrow{e} = 
       \begin{cases}
            \xrightarrowstar{\tau} \circ \xrightarrow{e} \circ \xrightarrowstar{\tau} & 
            \text{si } e \in \Sigma,  \\ 
            \xrightarrowstar{e} & \text{si } e = \tau,  \\
            \xrightarrow{e} & \text{si } e \in \Faults.\\
       \end{cases}
\]
%
El símbolo $\circ$ representa la composición de relaciones binarias y $\xrightarrowstar{\tau}$ es la clausura reflexo-transitiva de la relación binaria $\xrightarrow{\tau}$. 

Intuitivamente, si $e \notin \{\tau\}\cup\Faults$, entonces $s\xRightarrow{e}s'$ significa que existe una secuencia de cero o mas transiciones $\tau$ empezando en $s$, seguido de una transición etiquetada por $e$, seguida luego por cero o mas transiciones $\tau$ eventualmente llegando a  $s'$.
$s \xRightarrow{\tau} s'$ establece que $s$ puede transicionar a $s'$ por medio de cero o mas transiciones $\tau$.
%
En particular, $s \xRightarrow{\tau} s$ para todo $s$.
%
Para el caso en que $e\in\Faults$,
$s\xRightarrow{e}s'$ es equivalente a $s\xrightarrow{e}s'$ y por lo tanto
no se permite ningún paso $\tau$ antes o después de la transición $e$.


\begin{defi} \label{def:weak_mask}
  Sean $A =\langle S, \Sigma, \rightarrow, \InitState \rangle$ y $A' =\langle S',
  \SigmaF, \rightarrow', \InitStatePrime \rangle$ dos sistemas de transición etiquetados con $\Sigma$
  conteniendo $\tau$ posiblemente.  $A'$ es \emph{débilmente masking-tolerante a fallas}
  con respecto a $A$ si existe una relación $\M \subseteq S
  \times S'$ entre $A$ y $A'$ tal que:

\begin{enumerate}[(A)]
  \item $\InitState \M \InitStatePrime$
  \item para todo $s \in S, s' \in S'$ con $s \M s'$ y todo $e \in \Sigma \cup \{\tau\}$ vale lo siguiente:

  \begin{enumerate}[(1)]
    \item si $s \xrightarrow{e} t$ entonces 
    $\exists\; t' \in S': s' \xRightarrowprime{e} t' 
    \wedge t \M t'$;

      \item si $s' \xrightarrowprime{e} t'$ entonces  
      $\exists \; t \in S: s \xRightarrow{e} t  
      \wedge t \M t'$;

      \item si $s' \xrightarrowprime{F} t'$ para algún $F \in \Faults$ entonces 
      $s \M t'$.
  \end{enumerate}
\end{enumerate}

%
Si tal relación existe, decimos que $A'$ es una \emph{implementación débilmente masking-tolerante a fallas} de
$A$, denotado por $A \WeakMasking A'$.
\end{defi}

El siguiente teorema conecta la simulación de masking fuerte y débil. El teorema establece que la simulación de masking débil se vuelve una simulación de masking fuerte cuando la transición $\xrightarrow{}$ es reemplazada por $\xRightarrow{}$ en la estructura original.

\begin{thm} \label{thm:weak_thm}
  Sean
$A =\langle S, \Sigma, \rightarrow, \InitState \rangle$ y $A' =\langle S', \SigmaF, \rightarrow', \InitStatePrime \rangle$. 
$\M \subseteq S \times S'$ entre $A$ y $A'$ es una simulación de masking débil si y solo si:

\begin{enumerate}[(A)]
  \item $\InitState \M \InitStatePrime$, y
  \item para todo $s \in S, s' \in S'$ con $s \M s'$ y todo $e \in \Sigma \cup \{\tau\}$ vale lo siguiente:

  \begin{enumerate}[(1)]
    \item si $s \xRightarrow{e} t$ entonces $\exists\; t' \in S': s' \xRightarrowprime{\text{e}} t' 
    \wedge t \M t'$;

      \item si $s' \xRightarrowprime{e} t'$ entonces 
      $\exists \; t \in S: s \xRightarrow{e} t \
      \wedge t \M t')$;

      \item si $s' \xRightarrowprime{F} t'$ para algún $F \in \Faults$ entonces 
      $s \M t'$
  \end{enumerate}
\end{enumerate}

\end{thm}

\noindent
\begin{proof}
	Primero note que las condiciones (A), (B.1), (B.2) y (B.3) en este teorema implican las condiciones (A), (B.1), (B.2) y (B.3)
 de la Def.~\ref{def:weak_mask}, entonces la parte ``si'' es directa. Para la otra dirección, la condición (A) es la misma en el teorema y en la definición.
 Supongamos ahora que la condición (B.1) de la Def.~\ref{def:weak_mask} vale, i.e.,
 $s \xRightarrow{e} t$ para $e \in \Sigma \cup \{\tau\}$ y $t \in S$. 
Si $e \in \Sigma$, entonces, por definición de$\Rightarrow$, tenemos que existen $w$ y $v$ tal que $s \xrightarrowstar{\tau} w$, $w \xrightarrow{e} v$
 y $v \xrightarrowstar{\tau} t$. Por lo tanto, por definición de $\Rightarrow$ y por la condición (B.2) de la Def. \ref{def:weak_mask} 
 tenemos que existen estados $s',t',w',v' \in S'$
 tales que $s' \xRightarrow{\tau} w'$, $w' \xRightarrow{e} v'$, y $v' \xRightarrow{\tau} t'$ y, 
 además, $w \M w'$, $v \M v'$, y $t \M t'$ lo cual implica que$s' \xRightarrow{e} t'$. 
 La prueba para la condición (B.2) es similar.
 	Supongamos ahora que $s \M s'$ y $s' \xRightarrow{F} t'$, para algún $t' \in S'$. Entonces, por Def. de 
$\Rightarrow$, tenemos que $s' \xrightarrow{F} t'$ y por Def.~\ref{def:weak_mask}
 tenemos $s \M t'$ . Esto concluye la prueba.
 \qedhere 
 \end{proof} 
 
Una forma natural de verificar bisimilitud débil es por medio de \emph{saturar}
el sistema de transición  \cite{FernandezM91,Milner89} y lucho verificar bisimilitud fuerte sobre el sistema de transición saturado.
Similarmente, el Theorem~\ref{thm:weak_thm} nos permite computar la simulación de masking débil al reducir este problema a simulación de masking fuerte. Note que $\xRightarrow{e}$ puede ser alternativamente definida por medio de las siguientes reglas:

\[
\dfrac{p \xrightarrow{e} q}{p\xRightarrow{e} q} \hspace{2cm} 
\dfrac{}{p\xRightarrow{\tau} p} \hspace{2cm} 
\dfrac{p\xRightarrow{\tau} p_1 \xRightarrow{e} q_1 \xRightarrow{\tau} q}{p\xRightarrow{e} q}~(e \notin \Faults)
\]
	En lo que sigue utilizamos el teorema \ref{thm:weak_thm} para probar propiedades de masking débil, vale la pena recalcar que la mayoría de propiedades de masking fuerte pueden ser también probadas para masking débil recurriendo a este resultado.
	
\begin{exa}
Consideremos el siguiente ejemplo, consideramos una celda de memoria que almacena un bit de información y soporta operaciones de lectura y escritura, presentado en forma basada en estados en \cite{DemasiCMA17}. Un estado en este sistema mantiene el valor corriente de la celda de memoria ($m=i$, para $i=0,1$), escribir le permite a uno cambiar este valor, y leer retorna el valor almacenado.  
Obviamente, en este sistema el resultado de una lectura va a depender del valor almacenado en la celda. 
Por lo tanto, una propiedad que uno podría asociar a este modelo es que el valor leído de la celda coincida con el de la ultima escritura que se haya ejecutado en el sistema.
    
Una falla potencial en este escenario ocurre cuando la celda inesperadamente pierde su carga, y su valor almacenado cambia (e.g., cambia de $1$ a $0$ debido a perdida de carga). Una técnica típica para lidiar con esta situación es \emph{redundancia}: usar tres bits de memoria en lugar de solo uno. Las operaciones de escritura se realizan simultáneamente sobre los tres bits. La lectura, por otro lado, retorna el valor que se repite en la mayoría de bits, esto se conoce como \emph{votación}. 

Tomamos el siguiente enfoque para modelar este sistema. Las etiquetas $\text{W}_0, \text{W}_1, \text{R}_0,$ y $\text{R}_1$
representan las operaciones de escritura y lectura. Específicamente, $\text{W}_0$ (resp. $\text{W}_1$): escribe el valor cero (resp. uno) en la celda de memoria. $\text{R}_0$ (resp. $\text{R}_1$): lee el valor cero (resp. uno) almacenado en la celda de memoria.
La figura ~\ref{figure:exam_1_mem_cell} muestra tres sistemas de transición. El primero de izquierda a derecha representa el sistema nominal para este ejemplo (denotado como $A$).
Tanto el segundo como el tercer sistema de transición son implementaciones tolerantes a fallas de $A$, denotados como $A'$ y $A''$ respectivamente. Note que $A'$ contiene una falla, mientras que $A''$ considera dos fallas. Ambas implementaciones usan redundancia triple; intuitivamente, el estado $\text{t}_0$ contiene tres bits con valor cero y $\text{t}_1$ contiene tres bits con valor uno.
Además, el estado $\text{t}_2$ se alcanza cuando uno de los bits cambió de valor ($001$, $010$ o $100$)
En  $A''$, el estado $\text{t}_3$ se alcanza después de que cambia un segundo bit ($011$, $101$ o $110$) empezando del estado $\text{t}_0$.
\begin{figure}[h] 
\begin{center}
    \includegraphics[scale=0.45]{images/example_1_cell_mem.eps} 
   % \vspace{-1cm}
    \caption{Sistemas de transición para la celda de memoria.}
    %\vspace{-0.8cm}
    \label{figure:exam_1_mem_cell}
\end{center}
\end{figure}
\sloppy Es directo de ver que existe una relación de masking'tolerancia a fallas entre $A$ y $A'$, teniendo como testigo la relación $\M = \{(\text{s}_0, \text{t}_0), (\text{s}_1, \text{t}_1), (\text{s}_0, \text{t}_2)\}$. Es sencillo verificar que $\M$ satisface las condiciones de la Def.~\ref{def:masking_rel}.

Por otro lado, no existe una relación de masking entre $A$ y $A''$ ya que el estado $\text{t}_3$ necesita estar relacionado con el estado $\text{s}_0$ en cualquier relación de masking entre estos modelos. Este estado solo puede ser alcanzado con la ejecución de fallas, las cuales son necesariamente enmascaradas por pasos en si mismo. Sin embargo, note que, en el estado $\text{t}_3$, se puede leer el valor $1$ (la transición $\text{t}_3 \xrightarrow{\text{R}_1} \text{t}_3$) mientras que, en el estado $\text{s}_0$, solo se puede leer el valor $0$.
\end{exa}
 
\subsection{Juego de Simulación de Masking} \label{subsec:mask_sim_game}
	Vamos a definir un juego de simulación de masking para dos sistemas de transición (la especificación del sistema nominal y su implementación tolerante a fallas) que captura la masking-tolerancia a fallas. Primero definimos el grafo de juego de masking para dos jugadores, los cuales denominaremos como \emph{Refutador} ($\Refuter$) y \emph{Verificador}
($\Verifier$).

\begin{defi} \label{def:strong_masking_game_graph}
  Sean $A=\langle S, \Sigma, \rightarrow, \InitState \rangle$ y $A'=\langle S',
  \SigmaF, \rightarrow', \InitStatePrime \rangle$ dos sistemas de transición.
  % and $M \notin \Sigma \cup \SigmaF$.
  El \emph{grafo de juego para masking fuerte} 
  $\StrMaskGG = \langle V^G, V_\Refuter, V_\Verifier, E^G, {\InitVertex}^G \rangle$ 
  para dos jugadores se define de la siguiente manera:

\begin{itemize}
    %\item $\Sigma^G = \Sigma^1 \cup \SigmaF^2$
  \item $V^G = (S \times ( \Sigma^1 \cup \SigmaF^2 \cup\{\#\}) \times S' \times \{ \Refuter, \Verifier \}) 
  \cup \{\ErrorSt\}$
  \item El estado inicial es $\InitVertex^G = \langle \InitState, \#, \InitStatePrime, \Refuter \rangle$, donde el Refutador comienza a jugar.
  \item Los estados del Refutador son $V_\Refuter = \{ (s, \#, s', \Refuter) \mid s \in S \wedge s' \in S' \} 
  \cup \{\ErrorSt\}$
  \item Los estados del Verificador son $V_\Verifier = \{ (s, \sigma, s', \Verifier) \mid s \in S \wedge s' \in S' \wedge \sigma \in ( \Sigma^1 \cup \SigmaF^2 )\}$
\end{itemize}
y $E^G$ es el conjunto minimal que satisface que:
\begin{itemize}
  \item $\{ ( (s, \#, s', \Refuter) , (t, \sigma^{1}, s', \Verifier)) \mid \exists\;\sigma \in \Sigma: s \xrightarrow{\sigma} t \} \subseteq E^G$,

  \item $\{ ((s, \#, s', \Refuter), (s, \sigma^{2}, t', \Verifier))  \mid \exists\;\sigma \in \SigmaF: s' \xrightarrowprime{\sigma} t' \} \subseteq E^G$,

  \item $\{ ((s, \sigma^2, s', \Verifier), (t, \#, s', \Refuter)) \mid \exists\;\sigma \in \Sigma: s \xrightarrow{\sigma} t \} \subseteq E^G$,

  \item $\{ ((s, \sigma^1, s', \Verifier), (s, \#, t', \Refuter)) \mid \exists\;\sigma \in \Sigma: s' \xrightarrowprime{\sigma} t' \} \subseteq E^G$,

  \item $\{ ((s, F^2, s', \Verifier), (s, \#, s', \Refuter)) \} \subseteq E^G$, para cada $F \in \Faults$. 

  \item Si no hay transiciones que salgan de algún estado $v$, entonces, adicionalmente asumimos que $(v, \ErrorSt) \in E^G$ y $(\ErrorSt, \ErrorSt) \in E^G$.
\end{itemize}

\end{defi}

La intuición de este juego es lo siguiente. 
El Refutador elige transiciones de la especificación o de la implementación y hace su movimiento, luego el Verificador trata de igualar el movimiento de su contrincante, esto es similar al juego de bisimulación \cite{Stirling99}. 
Sin embargo, cuando el Refutador escoge una falla (note que solo este jugador puede elegir fallas), el Verificador debe igualar su movimiento con un paso en si mismo.
Esto intuitivamente se puede interpretar como un enmascaramiento de la falla por parte de la implementación tolerante a fallas de tal manera que la falla no puede ser observada del lado del usuario. $\Refuter$ gana si el juego alcanza un estado de error, i.e., $\ErrorSt$; de lo contrario, $\Verifier$ gana el juego. 
Básicamente este juego es un juego de alcanzabilidad \cite{Jurd11}.

Un \emph{grafo de juego de masking débil} $\WeakMaskGG$ se define de la misma manera que su contraparte fuerte de la 
Def.~\ref{def:strong_masking_game_graph}, con la diferencia de que
$\Sigma$ y $\SigmaF$ pueden contener $\tau$, y que el conjunto de transiciones etiquetadas (denotado como $E_W^G$) ahora se define utilizando relaciones de transición débiles (i.e., $\Rightarrow$ y $\Rightarrow'$) de los sistemas de transición respectivos.

La figura ~\ref{figure:exam_2_mem_cell_gg_two_faults} muestra una parte del juego de masking fuerte para el ejemplo que vimos considerando los sistemas de transición $A$ y $A''$. Aquí, los nodos del Refutador están representados gráficamente como cajas mientras que los nodos del Verificador están representados por círculos.
Podemos observar claramente en el grafo de juego que el Verificador no puede imitar la transición $((s_0, \#, t_3, R),(s_0, R_1^2, t_3, V))$
seleccionada por el Refutador, donde lee el valor $1$ en el estado $t_3$ de la implementación tolerante a fallas. Esto se da porque el Verificador solo puede leer el valor $0$ en el estado $s_0$. 
Entonces, el estado $\ErrorSt$ es alcanzado y el Refutador gana.

Como es de esperarse, hay una simulación de masking fuerte entre $A$ y $A'$
si y solo si el Verificador tiene una estrategia ganadora en $\StrMaskGG$.

\begin{thm} \label{thm:wingame_strat}
  Sean $A=\langle S, \Sigma, \rightarrow, \InitState \rangle$ y $A'=\langle S', \SigmaF, \rightarrow', \InitStatePrime \rangle$ dos sistemas de transición.
  Entonces, $A \Masking A'$ si y solo si el Verificador tiene una estrategia ganadora para el grafo de juego de masking fuerte $\StrMaskGG$.
\end{thm}
\begin{proof} 
	``solo si'': Supongamos que existe una simulación de masking $\M \subseteq S \times S'$.
La estrategia del Verificador (denominada $\pi^*$) se construye de la siguiente forma:  para los estados $(t, \sigma^1, s', V)$ (resp. $(s, \sigma^2, t', V)$) tal que el conjunto $\{ z' \in \post(s'): t \M z' \}$
(resp. $\{ z \in \post(s): z \M t' \}$) no es vacío y $\sigma \notin \Faults$, definimos $\pi^*(t, \sigma^1, s', \Verifier) = (t, \#, t', \Refuter)$ 
(resp. $\pi^*(s, \sigma^2, t', \Verifier) = (t, \#, t', \Refuter)$) que se corresponde con el arco $((t, \sigma^1, s', \Verifier), (t, \#, t', \Refuter))$ (resp. $((s, \sigma^2, t', \Verifier),(t, \#, t', \Refuter))$) tal que $t \in \{ z \in \post(s): z \M t' \}$ (resp. $t' \in \{ z' \in \post(s): z' \M t' \}$).  Si $\sigma \in \Faults$, entonces definimos $\pi^*(s, F^2, s', \Verifier) = (s, \#, s', \Refuter)$, para cualquier $F \in \Faults$, correspondiente con el arco $((s, F^2, s', \Verifier),( s, \#, s', \Refuter))$. Si el conjunto $\{ z' \in \post(s'): t \M z' \}$  (resp. $\{ z \in \post(s): z \M t' \}$) 
es vacío devuelve un vértice correspondiente a un arco arbitrario. 

	Probamos que cualquier jugada: $\rho_0 \rho_1 \dots$ (conforme a la estrategia $\pi^*$) cumple con: 
	(1) $\pr{3}{\rho_i}=\Verifier \vee \pr{0}{\rho_i} \Masking \pr{2}{\rho_i}$ y 
	(2) $\rho_i \neq \ErrorSt$, para todo $i\geq 0$. Esto implica que la estrategia es ganadora para el Verificador. 
La prueba es por inducción sobre $i$. El caso base es directo ya que 
$\InitVertex^G \neq \ErrorSt$ y $\InitVertex^G = (\InitState, \#, \InitStatePrime, \Refuter)$ y por suposición tenemos $\InitState \Masking \InitStatePrime$. Para el caso inductivo, asumimos que 
\todo{No entiendo esta frase}
$\rho_i$ holds $\rho_i \neq \ErrorSt$ y que $\pr{3}{\rho_i} = \Verifier$ o $\pr{0}{\rho_i} \Masking \pr{2}{\rho_i}$. 
Si $\rho_i = (s, \#, s', \Refuter)$, entonces esto significa que $s \Masking s'$. Como asumimos que
$\rightarrow$ y $\rightarrow'$ son seriales, tenemos que: $\exists \sigma \in \Sigma: s \xrightarrow{\sigma} t$ y $\exists \sigma \in \SigmaF : s' \xrightarrowprime{\sigma} t'$. Es decir, por definición del juego, $\rho_{i+1} \neq \ErrorSt$. 
Además, $\pr{3}{\rho_{i+1}} = \Verifier$ y entonces las aserciones (1) y (2) valen. 
Si $\rho_i = (s, \sigma^1, t, \Verifier)$ (resp. $\rho_i = (s, \sigma^2, t, \Verifier)$) y $\sigma \notin \Faults$,
tenemos que $\rho_{i-1} = (s, \#, s', \Refuter)$ y tenemos una transición $s \xrightarrow{\sigma} t$ (resp. $s' \xrightarrowprime{\sigma} t'$), por hipótesis inductiva, $s \Masking s'$. 
Por lo tanto, existe una transición $s' \xrightarrowprime{\sigma} t'$ (resp.  $s \xrightarrow{\sigma} t$) tal que $t \Masking t'$. 
La estrategia juega acorde a una de estas transiciones y entonces
$\rho_{i+1} = (t, \#, t', \Refuter)$ y $t \Masking t'$ y también $\rho_{i+1} \neq \ErrorSt$. Si $\sigma \in \Faults$, entonces $\rho_i = (s, F^2, t', \Verifier)$ para algún $F \in \Faults$. 
Por lo tanto, $\rho_{i-1} = (s, \#, s', \Refuter)$ con $s \Masking s'$ (por hipótesis inductiva). Esto significa que hay una transición $s' \xrightarrowprime{F} t'$, y por Def.~\ref{def:masking_rel},
tenemos que $s \Masking t'$. Además, por definición $\pi^*(s, F^2, t', \Verifier) = (s, \#,t',\Refuter)$, 
y además $s \Masking t'$. 

``si'': Supongamos que el Verificador tiene una estrategia ganadora (llamémosla $\pi$) 
desde el estado inicial. Entonces, definimos la relación de simulación de masking de la siguiente manera: 
\[
\M = \{(s,s') \mid \Verifier \text{ tiene una estrategia ganadora para } (s, \#, s', \Refuter) \}.
\]
Vamos a probar que es una simulación de masking. 
Primer, tenemos que $\InitState \M \InitStatePrime$ ya que $\pi$ es ganadora en $\InitVertex^G$. Por la condición (B.1) de la Def.~\ref{def:masking_rel}, si
$s \M s'$ y $s \xrightarrow{\sigma} t$ para algún $s \in S$ y $s' \in S'$, entonces debemos tener que $\pi$ es ganadora para cualquier estado $(s,\#,s',\Refuter)$. Además, por Def.~\ref{def:strong_masking_game_graph} existe un arco 
$((s,\#,s',\Refuter), (t,\sigma^1,s',\Verifier))$. 
Pero como $\pi$ es ganadora, también tenemos que $\pi(t,\sigma^1,s',\Verifier) \neq \ErrorSt$ y 
por lo tanto $\pi(t,\sigma^1,s',\Verifier)=(t, \#, t', \Refuter)$ y $\pi$ es ganadora desde  $(t, \#, t', \Refuter)$.
De este modo, hay una transición $t \xrightarrowprime{\sigma} t'$ tal que
$t \Masking t'$. La prueba para (B.2) es análoga. Para (B.3), asumamos que $s \Masking s'$ y $s' \xrightarrowprime{F} t'$ para algún 
$F \in \Faults$. 
Como antes, desde el estado $(s,\#,s',\Refuter)$, $\pi$ es ganadora donde también tenemos un arco $((s,\#,s',\Refuter), (s,F^2,t',\Verifier))$ por definición del juego. 
Como $\pi$ es ganadora desde $(s,\#,s',\Refuter)$ para el Verificador, existe una jugada $\pi(s,F^2,t',\Verifier) = (s,\#,t',\Refuter)$ tal que $\pi$ gana desde 
$(s,\#,t',\Refuter)$. Por lo tanto, $s \Masking t'$, y el resultado se deduce.
\qedhere

\end{proof} \\

Por el teorema ~\ref{thm:weak_thm}, el resultado también vale para juegos de masking débiles lo cual esta establecido en el siguiente teorema.

\begin{thm} \label{thm:weak_wingame_strat}
  Sean $A=\langle S, \Sigma \cup \{\tau\}, \rightarrow, \InitState \rangle$ y
  $A'=\langle S', \SigmaF \cup \{\tau\}, \rightarrow', \InitStatePrime \rangle$.
  $A \WeakMasking A'$ si y solo si el Verificador tiene una estrategia ganadora para el grafo de juego de masking débil $\WeakMaskGG$.
\end{thm}

\begin{thm}\label{th:game-determined}
  Para cualquier $A$ y $A'$, el grafo de juego de masking fuerte (resp.\ débil) 
  $\StrMaskGG$ (resp.\ $\WeakMaskGG$) puede ser determinado en tiempo $\BigO(|E^G|)$ (resp.\ $\BigO(|E_W^G|)$).
\end{thm}
\begin{proof}
	El conjunto de estados ganadores para el Refutador puede ser computado utilizando una búsqueda primero en ancho de abajo hacia arriba(bottom-up breadth-first) desde el estado de error, como en juegos de alcanzabilidad \cite{Jurd11}. 
Este procedimiento inspecciona cada arco, en el peor caso. Esto significa que el tiempo de ejecución de este algoritmo es $\BigO(|E^G|)$ para el caso de masking fuerte, y $\BigO(|E_W^G|)$ para el caso débil. Para este ultimo caso, hay que tener en cuenta que computar  
$\Rightarrow$ a partir de $\rightarrow$ toma un tiempo polinomial.
\qedhere
\end{proof} \\
\begin{figure} [h]
\begin{center}
    %\vspace{-0.5cm}
   % \includegraphics[scale=0.5]{ex1_cell_mem_game_graph_two_faults.eps} 
    \includegraphics[scale=0.5]{images/ex1_cell_mem_game_graph_two_faults.eps} 
  %  \vspace{-0.7cm}
    \caption{Parte del grafo de juego de masking para la celda de memoria con dos fallas}
    \label{figure:exam_2_mem_cell_gg_two_faults}
     %\vspace{-0.6cm}
\end{center}
\end{figure}

	Se puede solucionar juegos de alcanzabilidad computando conjuntos $\text{Reach}_0$ (estados ganadores para el Refutador) como un punto fijo de conjuntos $\text{Reach}^i_0$  \cite{Jurd11}.
	Estas ideas pueden ser adaptadas a nuestro contexto para tener en cuenta la cantidad de fallas. Esto sera útil en las próximas secciones, donde la cantidad de fallas es importante para razonar sobre la versión cuantitativa de los juegos de masking.
\begin{defi}\label{def:U} Dado un grafo de juego de masking fuerte $\StrMaskGG$, 
los conjuntos $\setsUs$ (para $i,j \geq 0$) se definen de la siguiente manera:
\begin{align*}
  U^0_i =& U^j_0 = \emptyset,  \label{def:of:Uji} \\
  U_1^1 =&  \{\ErrorSt\},
  \hspace{12.5cm} {}\notag
\end{align*}
%\vspace{-0.8cm}
\begin{align*}
  U_{i+1}^{j+1} =&
    \{v' \mid v' \in V_\Refuter \wedge \post(v') \cap U_{i+1}^j \neq \emptyset\} \\
    &\textstyle\cup
    %\{v' \mid v' \in S_V \wedge post(v') \subseteq \bigcup_{j'\leq j} U_{i+1}^{j'} \wedge post(v') \cap U^j_{i+1} \neq \emptyset \wedge \pr{2}{v'} \notin \Faults \} \\
    %NOTE: above is the original condition of TACAS paper
    \{v' \mid v' \in V_\Verifier \wedge \post(v') \subseteq \bigcup_{i'\leq i+1, j' \leq j}U_{i'}^{j'} \wedge \post(v') \cap U^j_{i+1} \neq \emptyset \wedge \pr{1}{v'} \notin \Faults \} \\
    &\textstyle \cup
    \{v' \mid  v' \in V_\Verifier \wedge \post(v') \subseteq \bigcup_{i'\leq i, j' \leq j}U_{i'}^{j'} \wedge \post(v') \cap \setsUs \neq \emptyset \wedge \pr{1}{v'} \in \Faults \}
\end{align*}
Además, $U^k = \bigcup_{i \geq 0} U_i^k$ y $U = \bigcup_{k \geq 0} U^k$.
\end{defi}
Intuitivamente, el subíndice $i$ en $U^k_i$ indica que $\ErrorSt$ es alcanzado después de que hayan ocurrido a lo sumo $i-1$ fallas y $k$ pasos.
El siguiente lema se demuestra de forma directa utilizando técnicas estándar de juegos de alcanzabilidad \cite{AlfaroHK07}.
	Note que estos conjuntos también pueden ser computados para juegos débiles de forma similar utilizando la relación $\Rightarrow$.
\begin{lem} \label{lemma:RefWinStrat} El Refutador tiene una estrategia ganadora en $\mathcal{G}_{A, A'}$ (o $\mathcal{G}^W_{A, A'}$) si y solo si $\InitVertex^G \in U^k$, para algún $k$.
\end{lem}
\begin{proof} 
	``solo si'': La prueba utiliza resultados estándar de juegos de alcanzabilidad. Mas específicamente, considere el conjunto $V^G \setminus U$, este conjunto es una trampa para el Refutador, es decir, si $(s,\sigma, s', \Verifier) \in V^G \setminus U$, entonces existe un $v \in \post((s,\sigma, s', \Verifier))$ tal que 
$v \in V^G \setminus U$, sino
$(s,\sigma, s', \Verifier) \in U^k$ para algún $k$. Similarmente, si $(s,\#, s', R) \in V^G \setminus U$, entonces para todo $v \in \post((s,\#, s', \Refuter))$ 
tenemos $v \in V^G \setminus U$. Esto significa que $V^G \setminus U$ es una ``trampa'' para el Refutador. 
Ahora bien, podemos definir una estrategia para el Verificador $\pi_\Verifier$ de la siguiente manera: si $(s, \sigma, s', \Verifier) \in V^G \setminus U$, entonces
$\pi_\Verifier(s, \sigma, s', \Verifier) = \rho$ para algún $v \in V^G \setminus U$ el (cuya existencia esta garantizada), de lo contrario retorna un nodo arbitrario. 
Esta estrategia es ganadora para el Verificador desde cualquier $v \in V^G$, es decir, para cualquier jugada $\rho_0 \rho_1 \rho_2 \dots$ 
si $\rho_0 \in V^G \setminus U$, entonces $\forall i \geq 0: \rho_i \in V^G \setminus U$ 
(lo cual implica que $\forall i \geq 0: \rho_i \neq \ErrorSt$). 
La prueba es por inducción sobre $i$. Para $i=0$ el resultado es directo ya que $\rho_0 \in V^G \setminus \{\ErrorSt\}$. Para el caso inductivo,
supongamos que $\rho_i \in V^G \setminus U$. En el caso de que $\rho_i = (s, \sigma, s', \Verifier)$, entonces por definición de $\pi_\Verifier$, 
$\rho_{i+1} = \pi_\Verifier(s,\sigma, s', \Verifier) \in V^G \setminus U$.
Además, si $\rho_i = (s, \#, s', \Refuter)$, entonces $\post((s, \#, s', \Refuter)) \subseteq V^G \setminus U$, 
y por lo tanto $\rho_{i+1} \notin U$. 
Ahora bien, como $\InitVertex^G \notin U^k$ para todo $k$, entonces $\InitVertex^G \notin U$ y $\InitVertex^G \in V^G \setminus U$. 
En consecuencia, por la propiedad demostrada arriba $\Refuter$ tiene una estrategia ganadora desde 
$\InitVertex^G$. Pero esto es una contradicción porque el Refutador y el Verificador no pueden tener ambos estrategias ganadoras desde los mismos estados.
Por consiguiente, $\InitVertex^G \in U^k$ para algún $k$.
	
``si'': Considere $\InitVertex^G \in U^k$ para algún $k$, es decir, tenemos  $\InitVertex^G \in U^k_i$ para algún $i$ por Def.~\ref{def:U}. 
Además, para cada $v \in V^G$ definimos $\delta(v) = \min \{ (i,j) \mid v \in U^j_i \}$ (usando orden lexicográfico), por conveniencia asumimos $\min \emptyset = (\infty, \infty)$.
Entonces, la estrategia ganadora $\pi_R$ para el Refutador se define de la siguiente forma. Si $\delta(v) = (i,j)$ (con $(i,j) < (\infty, \infty)$), entonces $\pi_\Verifier(v) = w$, con $w$ siendo un vértice tal que
$\delta(w) = (i,j-1)$ (si $\pr{1}{v} \notin \Faults$) o $\delta(w) = (i-1,j-1)$ (si $\pr{1}{v} \in \Faults$), cuya existencia esta garantizada por Def.~\ref{def:U}. De otra forma, $\pi_\Refuter(v)$ retorna
un vértice arbitrario. Note que para cualquier jugada $\rho_0 \rho_1 \rho_2 \dots$ comenzando en $\InitVertex^G$ tenemos que:  para todo $i \geq 0$, $\delta(\rho_i) > \delta(\rho_{i+1})$, en orden lexicográfico.
	Por lo tanto, para algún $k>0$ tenemos $\delta(\rho_k) = (1,1)$, y por tanto $\rho_k = \ErrorSt$.
\qedhere
\end{proof} \\

Por el teorema \ref{thm:weak_thm}, esta prueba también aplica al juego de masking débil $\WeakMaskGG$.
%By theorem \ref{thm:weak_thm}, the proof also applies for the weak masking game graph $\mathcal{G}^W_{A, A'}$. 

\section{Masking Cuantitativo} \label{sec:QuantMask}
En esta sección, extendemos el juego de simulación de masking fuerte (resp. débil) introducido anteriormente con objetivos cuantitativos para así definir la noción de distancia de masking-tolerancia a fallas.
% In practice, fault-tolerance appears with a quantitative flavor, that is, fault-tolerant systems have %degrees of tolerance. This is particularly true when techniques like redundancy and voting are used.  %take this into account. 
Es importante remarcar que utilizamos el atributo  ``cuantitativo'' en un sentido no probabilista.
\begin{defi}  
  %Let $A=\langle S, \Sigma, E, s_0\rangle$ and $A'=\langle S', \Sigma_{\Faults}, E', s'_0 \rangle$.  T
  Para los sistemas de transición $A$ y $A'$, el \emph{grafo de juego de masking cuantitativo fuerte} 
  $\QMStrGame = \langle V^G, V_\Refuter, V_\Verifier, E^G,  \InitVertex^G,  \reward^G \rangle$ se define de la siguiente manera:
 
\begin{itemize}
\item
  $\mathcal{G}_{A, A'}=\langle V^G, V_\Refuter, V_\Verifier, E^G, \InitVertex^G \rangle$ se define como en Def.~\ref{def:strong_masking_game_graph},
\item
  $ \reward^G(v) = (\chi_{\Faults}(\pr{1}{v}), \chi_{\ErrorSt}(v))$

\end{itemize}
%
donde $\chi_{\Faults}$ es la función característica sobre el conjunto $\Faults$, devolviendo $1$ si $\sigma \in \Faults$ y $0$ de otro modo, y $\chi_{\ErrorSt}$ es la función característica sobre el singleton $\{\ErrorSt\}$.
\end{defi}
Note que la función de recompensa retorna un par de números en lugar de un solo numero. Es directo codificar al par como un único numero pero no lo hacemos por claridad. Remarcamos que el
\emph{grafo de juego de masking cuantitativo débil} $\QMWeakGame$
se define de la misma forma que el grafo de juego definido arriba pero utilizando el grafo de juego de masking débil $\mathcal{G}^W_{A, A'}$ en lugar de 
$\mathcal{G}_{A, A'}$


Dado un grafo de juego de masking cuantitativo fuerte con la función de recompensa $\reward^G$ y la jugada 
$\rho = \rho_0 \rho_1 \rho_2, \ldots$, para todo $i \geq 0$, sea 
%$v_i = v^G(\rho_i \xrightarrow{\sigma_i} \rho_{i+1})$.
$r_i = \reward^G(\rho_i)$.
Definimos la \emph{función de payoff de masking} de la siguiente manera: 
\[%\displaystyle
\FMask(\rho) = \lim_{n \rightarrow \infty}  \frac{\pr{1}{r_n}}{1+ \sum^{n}_{i=0} \pr{0}{r_i}},
%\FMask(\rho) = \liminf_{n \rightarrow \infty}  \frac{\pr{1}{v_n}}{1+ \sum^{n}_{i=0} \pr{0}{v_i}},
\]
la cual es proporcional a la inversa de la cantidad de movimientos de enmascaramiento realizados por el Verificador. Para entender esto, note que el numerador de $\frac{\pr{1}{r_n}}{1+ \sum^{n}_{i=0} \pr{0}{r_i}}$ será $1$
cuando alcancemos el estado de error, es decir, en aquellos caminos que no alcancen el estado de error esta formula devuelve $0$. Además. si el estado de error es alcanzado,  el denominador contara el numero de transiciones de falla tomadas hasta el estado de error. Todas, a excepción de la ultima fueron enmascaradas exitosamente. Sin embargo, la ultima falla tratara de ser enmascarada por el Verificador, pero eventualmente llevará al estado de error.
Es decir, los vértices con valor $(1,\_)$ son aquellos correspondientes a fallas. Los demás se corresponden con $(0,\_)$.
Observe también que si $\ErrorSt$ es alcanzado en $v_n$ sin la ocurrencia de ninguna falla, la parte nominal de la implementación no se corresponde con la especificación nominal, en cuyo caso $\frac{\pr{1}{r_n}}{1+ \sum^{n}_{i=0} \pr{0}{r_i}}=1$.
Entonces, 
el Refutador quiere maximizar el valor de cualquier corrida, es decir, tratará de de ejecutar fallas que lleven al estado $\ErrorSt$. 
Por el contrario, el Verificador quiere evitar $\ErrorSt$ y entonces tratará de enmascarar fallas de tal forma que se aleje del estado de error. 

Mas precisamente, el valor del juego de masking cuantitativo fuerte para el Refutador se define como $\val_\Refuter(\QMStrGame) = \Sup_{\pi_\Refuter \in \Pi_\Refuter} \; \Inf_{\pi_\Verifier \in \Pi_\Verifier} \FMask(\out(\pi_\Refuter, \pi_\Verifier))$. Análogamente, el valor del juego para el Verificador se define como $\val_\Verifier(\QMStrGame) = \Inf_{\pi_\Verifier \in \Pi_\Verifier} \; \Sup_{\pi_\Refuter \in \Pi_\Refuter} \FMask(\out(\pi_\Refuter, 
\pi_\Verifier))$. Entonces, definimos el valor del juego de masking cuantitativo fuerte, denotado por $\val(\QMStrGame)$, como el valor del juego del Refutador o del Verificador, i.e., $\val(\QMStrGame) = \val_\Refuter(\QMStrGame) = \val_\Verifier(\QMStrGame)$. Esto es posible debido a que los juegos de masking cuantitativos fuertes están determinados como lo probaremos mas abajo en el Teorema~\ref{thm:mask_game_det}. \\

\begin{defi} \label{def:mask_dist}
 Sean $A$ y $A'$ unos sistemas de transición. 
La \emph{distancia de masking fuerte} entre $A$ y $A'$, denotada por $\DeltaMask(A, A')$ se define de la siguiente manera:
$\DeltaMask(A, A') = \val(\QMStrGame).$
\end{defi}

	Nos gustaría remarcar que la \emph{distancia de masking débil} $\DeltaMask^W$ se de fine de la misma forma para el grafo de juego de masking cuantitativo débil $\QMWeakGame$.  A grandes rasgos, estamos interesados en medir la cantidad de fallas que pueden ser enmascaradas. El valor del juego está esencialmente determinado por las transiciones de falla en el grafo de juego y como los jugadores pueden encontrar una estrategia que lleve a (o evite) el estado $\ErrorSt$, independientemente de la existencia de acciones silenciosas.

A continuación, estableceremos algunas propiedades básicas de este tipo de juegos. 
Como ya anticipamos, los juegos de masking cuantitativos fuertes están determinados.

\begin{thm} \label{thm:mask_game_det}
  Para cualquier grafo de juego de masking cuantitativo fuerte $\QMStrGame$ con función de payoff $\FMask$:
%  \vspace{-0.3cm}
  \[\textstyle
  \Inf_{\pi_\Verifier \in \Pi_\Verifier} \; \Sup_{\pi_\Refuter \in \Pi_\Refuter} \FMask(\out(\pi_\Refuter, \pi_\Verifier)) = \Sup_{\pi_\Refuter \in \Pi_\Refuter} \;  \Inf_{\pi_\Verifier \in \Pi_\Verifier} \FMask(\out(\pi_\Refuter, \pi_\Verifier))\]
\end{thm}
\begin{proof} Para demostrar que la función de payoff de masking $\FMask$ está determinada tenemos que probar que es acotada y Borel-medible (teorema de Martin \cite{Martin98}). Primero, $\FMask$ está acotada por definición. Segundo, para ver que $\FMask$ es Borel-medible note que $\FMask(\Omega) \subseteq [0,1]$, y entonces es suficiente para probar que, para cada racional $x$, $\FMask^{-1}((-\infty, x])$ es Borel en la topología de Cantor de ejecuciones infinitas. 
Considere $\FMask^{-1}([-\infty,x])$ para un $x$ arbitrario, esto es lo mismo que $\FMask^{-1}([0, \frac{1}{a}])$ para un $a$ dado. Pero, $\FMask^{-1}([0, \frac{1}{a}]) = \bigcup_{b \geq a} A_b$ donde
$A_b = \bigcup_{i >0} A^i_b$ para $A^i_b = \{ \rho_0 \rho_1 \dots \mid \rho_i = \ErrorSt \wedge \sum^{i-1}_{j=0} \chi_{\Faults}(\pr{1}{\rho_j}) =b\}$. Note que 
$A^i_b = \{ C_{\rho_0 \dots \rho_i} \mid \sum^{i-1}_{j=0} \chi_{\Faults}(\pr{1}{\rho_j}) =b\}$ donde $C_{\rho_0 \dots \rho_i}$ es el cono correspondiente al segmento inicial 
$\rho_0 \dots \rho_i$ el cual es Borel-medible, y por lo tanto $A^i_b$, $A_b$ y $\FMask^{-1}((-\infty, x])$ son Borel-medibles.
\qedhere
%is the set of all sequences $x_0 x_1 x_2 \dots$
%such that we have $b$ or more ones, and this is the union $\bigcup A_i$ being $A_i = \{\sigma \mid \mbox{ number of ones of } \sigma[0,i] = b\}$, which are Borel sets, thus $A$ is a Borel set.
\end{proof} \\


\subsection{Cómputo de la Distancia de Masking}

En esta subsección presentamos los algoritmos para computar la distancia de masking para un juego de masking cuantitativo fuerte (resp. débil). Mostramos que, en el caso de sistemas deterministas, computar la distancia de masking puede ser logrado utilizando un algoritmo de camino mas corto. A diferencia de los sistemas no deterministas, en cuyo caso el computo puede ser realizado por un algoritmo de punto fijo.

Comenzamos presentando el próximo teorema que establece como se calcula el valor de un juego de masking cuantitativo fuerte dado.
%
\begin{thm} \label{thm:quant_game}
  Sea $\QMStrGame$ un grafo de juego de masking cuantitativo fuerte.
  \sloppy Entonces, $\val(\QMStrGame) = \frac{1}{\pr{0}{\delta(\InitVertex^G)}}$, con
  $\delta(\InitVertex^G) = \min \{ (i,j) \mid  \InitVertex^G \in \setsUs \}$, siempre que
  $\InitVertex^G \in U$, y $\val(\QMStrGame)=0$ de otro modo, donde los conjuntos
   $\setsUs$ y $U$ están definidos en Def.~\ref{def:U}.

\end{thm}
\begin{proof} La prueba es por casos:
	
	If $\InitVertex^G \notin U$, then we define the following strategy for the Verifier. If $v \in V_\Verifier \setminus U$, then $\pi^*_\Verifier(v)=v'$ 
	for some $v' \in V_\Refuter \setminus U$, and if $v \notin  V_\Verifier \setminus U$, then $\pi^*_\Verifier(v) = v'$ for an arbitrary $v'$. 
	We prove by induction that for any play conforming $\pi^*_\Verifier$: $\rho =\rho_0 \rho_1 \dots$ we 
have $\forall i \geq 0: \rho_i \notin V \setminus U$. The base case is direct. 
For the inductive case, assume that $\rho_i \notin V \setminus U$. In case that $\rho_i \in V_\Verifier$, the property follows by the definition of
$\pi^*_\Verifier$. If $\rho_i \in V_\Refuter$, since $\rho_i \in V^G \setminus U$, then $\post(\rho_i) \subseteq V^G \setminus U$ (by Def.~\ref{def:U}). 
Thus, $\rho_{i+1} \in V^G \setminus U$. 
	 Furthermore, we also have $\rho_i \neq \ErrorSt$ since $\{ \ErrorSt \} \subseteq U$. Also, note that any play avoiding state $\ErrorSt$ has value $0$: by definition of $\QMStrGame$, each transition performed by the Refuter  must be followed by a transition
selected by the Verifier. These transitions (the matches performed by the Verifier) have cost $(1,0)$ since the target of any of these transitions is different from $\ErrorSt$. Because we have an infinite number of these matches, when state $\ErrorSt$ is not reached, the valuation 
of these plays is $\lim_{n\rightarrow \infty} \frac{0}{1+ \sum^n_{i=0} \pr{0}{r_i}} = 0$ (recall that $r_i = \reward^G(\rho_i)$). 
Summing up, $\sup_{\pi_\Refuter \in \Pi_\Refuter} \FMask(\out(\pi_\Refuter, \pi^*_\Verifier)) = 0$, and since $\FMask$ is positive
we have $\val(\QMStrGame) = \inf_{\pi_\Verifier \in \Pi_\Verifier} \sup_{\pi_\Refuter \in \Pi_\Refuter} \FMask(\out(\pi_\Refuter, \pi_\Verifier)) = 0$.

	If $\InitVertex^G \in U$, we have that $\delta(\InitVertex^G) < (\infty, \infty)$, then we define the following strategy for the Refuter. 
	If $v \in U$, then $\pi^*_\Refuter(v)= v'$, where $\delta(v') < \delta(v)$, otherwise $\pi^*_\Refuter(v)=v'$, for an arbitrary $v'$. 
	Strategy $\pi^*_R$ is well-defined because if $v \in \setsUs$, then there is a $v' \in \post(v)$ such that $v \in U^{j-1}_i$ (by Def.~\ref{def:U}). Furthermore, $\delta(v') = (\pr{0}{\delta(v)}, \pr{1}{\delta(v)} - 1)$.  
	Now, we prove that for every strategy $\pi_\Verifier \in \Pi_\Verifier$ such that $\out(\pi^*_\Refuter, \pi_\Verifier) = \rho = \rho_0 \rho_1 \dots$ and $\rho_0 \in U$, we have:
\begin{equation}\label{eq:inf-bound}
	\FMask(\out(\pi^*_\Refuter, \pi_\Verifier)) \geq \frac{1}{ \pr{0}{\delta(\rho_0)}}
\end{equation} 
		To prove that, note that, if $\delta(\rho_0)= (i,j)$, then $\rho_j =\ErrorSt$ 
		since $\pr{1}{\delta(\rho_k)} = \pr{1}{\delta(\rho_{k+1})}+1$ for all $k$. 
That is, in $j$ steps we have $\delta(\rho_{j}) = (1,1)$ and so $\rho_{j} = \ErrorSt$.
	This implies that $\FMask(\rho) =  \lim_{n \rightarrow \infty}  \frac{\pr{1}{r_n}}{1+ \sum^{n}_{i=0} \pr{0}{r_i}} >0$, and also
$ \sum^{\pr{1}{\delta(\rho_0)} + k}_{i=0} \pr{0}{r_i} =  \sum^{\pr{1}{\delta(\rho_0)}}_{i=0} \pr{0}{r_i}$.
	 That is, proving:
\begin{equation}\label{eq:bottom-sum}
 1+ \sum^{\pr{1}{\delta(\rho_0)}}_{i=0} \pr{0}{r_i} \geq \pr{0}{\delta(\rho_0)}
\end{equation} 
we get $\FMask(\rho) \geq \frac{1}{ \pr{0}{\delta(\rho_0)}}$. The proof is by induction on $\delta(\rho_0)$, the base case is direct.
 For the inductive case, if $\rho_0 \in V_\Refuter$, then
\begin{align*}
 	1 + \sum^{\pr{1}{\delta(\rho_0)}}_{i=0} \pr{0}{r_i} & =  1 + \pr{0}{r_0} + \sum^{\pr{1}{\delta(\rho_0)}}_{i=1} \pr{0}{r_i} & \\
					   %  & =   1+\sum^{\pr{1}{\delta(\rho_0)}}_{i=1} \pr{0}{r_i} & \\
					    % & =   1+\sum^{\pr{1}{\delta(\rho_1)}+1}_{i=1} \pr{0}{r_i} & \\
					     & =   1+ \sum^{\pr{1}{\delta(\rho_1)}}_{i=0} \pr{0}{r_{i+1}} &  \text{($\rho_0 \in V_\Refuter$)}\\
					     & \geq \pr{0}{\delta(\rho_1)} & \text{(I.H.)} \\
					     & = \pr{0}{\delta(\rho_0)} & \text{($\rho_0 \in V_\Refuter$)}
\end{align*}
 If $\rho_0 \in V_\Verifier$ and $\pr{1}{\rho_0} \notin \Faults$, then the proof is as above. If $\pr{1}{\rho_0} \in \Faults$, then:
\begin{align*}
 	1+\sum^{\pr{1}{\delta(\rho_0)}}_{i=0} \pr{0}{r_i} & =  1+\pr{0}{r_0} + \sum^{\pr{1}{\delta(\rho_0)}}_{i=1} \pr{0}{r_i} & \\
					 %    & =   2 + \sum^{\pr{1}{\delta(\rho_0)}}_{i=1} \pr{0}{r_i} & \text{($\pr{1}{\rho_0} \in \Faults$)}\\
					  %   & =  2 +  \sum^{\pr{1}{\delta(\rho_1)}+1}_{i=1} \pr{0}{r_i} & \\
					     & =  2 +  \sum^{\pr{1}{\delta(\rho_1)}}_{i=0} \pr{0}{r_{i+1}} &   \text{($\pr{1}{\rho_0} \in \Faults$)}\\
					     & \geq 1 + \pr{0}{\delta(\rho_1)} & \text{(I.H.)} \\
					     & = \pr{0}{\delta(\rho_0)} & \text{($\pr{1}{\rho_0} \in \Faults$)}
\end{align*}
Thus, inequation \ref{eq:bottom-sum} holds. Now, consider the strategy $\pi^*_\Verifier$ for the Verifier defined as follows. For $v \in U$, $\pi^*_\Verifier(v) = v'$ such that $\delta(v') = \max \{\delta(v'') \mid v'' \in \post(v)\}$, otherwise $\pi^*_\Verifier(v) = v'$ for an arbitrary $v'$. 
Let us prove that $\FMask(\out(\pi^*_\Refuter, \pi^*_\Verifier)) = \frac{1}{\pr{0}{\delta(\rho_0)}}$. 
%Let $(\out(\pi^*_\Verifier, \pi^*_\Refuter)) = \rho_0 \rho_1 \dots$.
Note that, by definition of $\pi^*_\Verifier$ and $\pi^*_\Refuter$, we have that $\pr{1}{\delta(\rho_{i+1})} = \pr{1}{\delta(\rho_i)} - 1$, %$\pr{1}{\delta(\rho_i)} = \pr{1}{\delta(\rho_{i+1})}+1$
that is, in at most $\pr{1}{\delta(\rho_0)}$ steps we reach $\ErrorSt$. Thus, $\lim_{n \rightarrow \infty}  \pr{1}{r_n} =1$. 
Furthermore, if $\pr{1}{\rho_i} \in \Faults$, then $\pr{0}{\delta(\rho_{i+1})} = \pr{0}{\delta(\rho_i)} - 1$
by definition of $\pi^*_V$. Thereby, at most $\pr{0}{\rho_0}$ faults occur in $\rho$, and then
$\FMask(\out(\pi^*_\Refuter, \pi^*_\Verifier)) = \frac{1}{\pr{0}{\delta(\rho_0)}}$. Therefore, this and inequation \ref{eq:inf-bound} give us: 
$\inf_{\pi_\Verifier \in \Pi_\Verifier} \FMask(\out(\pi^*_\Refuter, \pi_\Verifier)) = \frac{1}{\pr{0}{\delta(v^G_0)}}$. 


Finally, let us prove that there is no
strategy $\pi^+_\Refuter$  such that $\sup_{\pi_\Verifier} \FMask(\out(\pi^+_\Refuter, \pi_\Verifier)) < \frac{1}{\pr{0}{\delta(v^G_0)}}$. 
Thus, $\pi^*_\Refuter$ is the best strategy. 
For the sake of contradiction, assume that we have such a strategy. Consider $\out(\pi^+_\Refuter, \pi^*_\Verifier) = \rho = \rho_0 \rho_1 \dots$, 
and let $\delta(\rho_0) = (i,j)$. Then, note that, if $\rho_i \in V_\Refuter$, then $\pr{0}{\delta(\rho_i)} = \pr{0}{\delta(\rho_{i+1})}$, 
moreover, if $\rho_i \in V_\Verifier$ then $\pr{0}{\delta(\rho_i)} = 1 + \pr{0}{\delta(\rho_{i+1})}$. 
Thus, $\lim_{n \rightarrow \infty}\sum^n_{0} \pr{0}{r_i} = \frac{1}{\pr{0}{\delta(\rho_0)}}$, which yields a contradiction.
Hence, $\val(\QMStrGame) = \inf_{\pi_\Verifier \in \Pi_\Verifier} \sup_{\pi_\Refuter \in \Pi_\Refuter} \FMask(\out(\pi_\Refuter, \pi_\Verifier)) = \frac{1}{\pr{0}{\delta(v^G_0)}}$.
\qedhere
.

\end{proof} \\

	Interestingly, for deterministic games the value of the game is given by the shortest path (if it exists) to the error state, as shown in the following theorem.
\begin{thm}\label{theorem:det-games} Let  $\QMStrGame = \langle V^G, V_\Refuter,V_\Verifier, E^G, \InitVertex^G, \reward^G \rangle$ be a deterministic 
quantitative strong masking game graph,
and let  $G_{\QMStrGame} = \langle V^G,  E^G, \InitVertex^G, w \rangle$ be its underlying weighted graph, with a weight function $w$ 
defined as:
\[
w(v, v') =
\begin{cases} 1 & \text{if $\pr{1}{v'} \in \Faults$} \\
		      0 & \text{otherwise}	 
\end{cases}
\]
	If $\rho = \rho_0 \rho_1  \dots \ErrorSt$ is the shortest path (w.r.t. function $w$)
to the error state in $G_{\QMStrGame}$, and $k$ is the number of faults occurring in $\rho$, then: 
\[
	\val(\QMStrGame) = \frac{1}{1+k}
\]
\end{thm}
\begin{proof}
	Let us define a strategy $\pi_\Refuter \in \Pi_\Refuter$ for the Refuter such that:
\[
	\inf_{\pi_\Verifier \in \Pi_\Verifier} f_m(\out(\pi_\Refuter, \pi_\Verifier)) = \sup_{\pi_\Refuter' \in \Pi_\Refuter} \inf_{\pi_\Verifier \in \Pi_\Verifier} f_m( \out(\pi_\Refuter',\pi_\Verifier))
\]	
	where this strategy conforms the shortest path in the underlying graph. Let us assume that: 
$\rho_0 \rho_1 \dots \ErrorSt$ is the shortest path. We can safely assume that no vertex appears twice in $\rho$, otherwise we can reduce the shortest path by removing the cycle.
	Keeping in mind that remark, we define the strategy $\pi_\Refuter$ as follows:
\[
	\pi_\Refuter(v) = \begin{cases}
								\rho_{j+1} & \text{if $v = \rho_j$ for some $j \geq 0$,} \\
								v' & \text{for arbitrary $v' \in V_\Verifier$.}
						   \end{cases}
\] 
First, let us prove that $\inf_{\pi_\Verifier \in \Pi_\Verifier} f_m (\out(\pi_\Refuter, \pi_\Verifier)) = \frac{1}{1+k}$. Let $\pi_\Verifier$ be an arbitrary strategy for the Verifier. Let us prove that
the play $\rho'_0 \rho'_1 \dots$ conforming $\pi_\Refuter$ and $\pi_\Verifier$ is equal to $\rho$. We prove it by induction 
on $\rho_i$, it holds for $\rho_0 = \InitVertex = \rho'_0$. Assume that it holds 
for $i=n$, that is, $\rho'_n = \rho_n$, if $\rho_n = (t, \sigma^1, s', \Verifier)$ (resp.  $\rho_n = (s, \sigma^2, t', \Verifier)$), since $A$ and $A'$ are deterministic, there is a unique edge $((t, \sigma^1, s', \Verifier), (t, \#, t', \Verifier)) \in E^G$ (resp. $((s, \sigma^1, t', \Verifier), (t, \#, t', \Verifier)) \in E^G$) and therefore $\rho'_{n+1} = \pi_\Verifier( \rho'_n) = (t, \#, t', \Verifier) = \rho_{n+1}$. If 
$\rho'_n = \rho_n = (s, \#, s', \Refuter)$ then $\rho'_{n+1} = \pi_\Refuter(\rho'_n) = \pi_\Refuter(\rho_n) = \rho_{n+1}$. 
	
	In addition,  the valuation of $\rho$ is $\frac{1}{1+k}$, that is, we obtain:
\begin{equation} \label{eq1}
\begin{split}
 f_m (\out(\pi_\Refuter, \pi_\Verifier))  =  \frac{1}{1+k},
\end{split}
\end{equation}	
and since this holds for any $\pi_\Verifier$, we obtain:
\begin{equation} \label{eq1}
\begin{split}
\inf_{\pi_V \in \Pi_V} f_m (\out(\pi_\Refuter, \pi_\Verifier))  =  \frac{1}{1+k}
\end{split}
\end{equation}

Now, assume that there is a strategy $\pi_\Refuter'$ for the Refuter such that 
$\inf_{\pi_V \in \Pi_V} f_m (\out(\pi_\Refuter', \pi_\Verifier)) > \inf_{\pi_\Verifier \in \Pi_\Verifier} f_m (\out(\pi_\Refuter, \pi_\Verifier))$.
From (\ref{eq1}) we have that $\inf_{\pi_\Verifier \in \Pi_\Verifier} f_m(\out(\pi_\Refuter', \pi_\Verifier)) > \frac{1}{1+k}$. 
Therefore, the number of faults observed in 
$\inf_{\pi_\Verifier \in \Pi_\Verifier} f_m(\out(\pi_\Refuter', \pi_\Verifier))$ should be less than $k$. 
But, it would be a shorter path in the underlying graph, which is a contradiction. Thereby, we have:
\[	
	\inf_{\pi_\Verifier \in \Pi_\Verifier} f_m(\out(\pi_\Refuter, \pi_\Verifier)) = \sup_{\pi_\Refuter' \in \Pi_\Refuter} \inf_{\pi_\Verifier \in \Pi_\Verifier} f_m( \out(\pi_\Refuter',\pi_\Verifier)).
\]	
\qedhere
\end{proof} \\

	Let us note that Theorems~\ref{thm:mask_game_det}, \ref{thm:quant_game} and \ref{theorem:det-games} apply as well to 
$\QMWeakGame$.
	Theorem~\ref{theorem:det-games} allows us to provide efficient procedures for deciding quantitative games in the 
case of deterministic systems. For non-deterministic systems, the procedure is computationally more
expensive (in terms of time and memory) than the deterministic case, because 
the sets $\setsUs$ (Def.~\ref{def:U}) need to be computed. \\

It is important to remark that, for non-deterministic systems, the shortest path does not always determine winning strategies. This can be seen by means of a simple example as follows. Let us consider the two transition systems shown in Fig.~\ref{fig:two-nondet-systems}.

\begin{figure} [h]
\begin{center}
    \includegraphics[scale=0.5]{nondeterm.eps} 
    \caption{Two non-deterministic systems}
    \label{figure:nondeterm}\label{fig:two-nondet-systems}
\end{center}
\end{figure}
\begin{figure} [h]
\begin{center}
    \includegraphics[scale=0.5]{nondeterm-game.eps} 
    \caption{The quantitative game for systems $M$ and $M'$}
    \label{figure:nondeterm-game}
\end{center}
\end{figure}

In this example, the non-determinism is provided by action $b$. Note also that there is a fault in  $M'$ connecting states $t_2$ and $t_3$. The quantitative game for these systems is shown in Fig.~\ref{figure:nondeterm-game}. Note that any shortest path of this graph w.r.t. the weight 
function $w$ has value $1$ (any path in the graph avoiding the faults). However, the value of this game is $\frac{1}{2}$ due to the best 
strategy for the Refuter is to take the Verifier to state $(s_2,b^1,t_1,\Verifier)$. 
From there, since the Verifier wants to minimize the value of the game, her best play  is to move to state $(s_2,\#,t_2,\Refuter)$. Then, the Refuter can choose a fault which leads to the error state. \\
%Note that the value of this game is $\frac{1}{2}$.\\

The above observation suggests that a different algorithm for computing the masking distance for 
non-deterministic systems is needed. 
The main idea is to compute the sets $\setsUs$ as defined in Def.~\ref{def:U} via 
a fix-point approach. More precisely, the sets $\setsUs$ can be calculated using a 
bottom-up breadth-first search from the error state. 
Algorithm \ref{Alg:MainAlg} shows the pseudo-code for solving any quantitative 
strong (resp. weak) masking game. 
It takes as input a quantitative strong masking game graph $\QMStrGame = \langle V^G, V_\Refuter, V_\Verifier,E^G, \InitVertex^G, \reward^G \rangle$ 
and computes the value of the game, i.e., $\val(\QMStrGame)$.
%Let us describe the essential steps of the algorithm for solving any quantitative strong (resp. weak) masking game. 
The algorithm annotates the nodes of the graph with numbers $i$ and $j$. 
%The algorithm can be understood as follows. 
The main steps of the algorithm are as follows:


\begin{enumerate}
  \item Every vertex $v$ is annotated with a pair $(i, j)$ representing that $v \in U_{i}^{j}$. Initially, $\ErrorSt$ is labeled with $(1, 1)$ and every other   
  node is labeled with $(\infty, \infty)$ (lines \ref{Alg:InitB}-\ref{Alg:InitE}). 
  Moreover, we use the assignment $\Q \gets \emptyset$ to denote the creation of an empty queue. 
  Subsequently, we enqueue the error state (lines \ref{Alg:CreateQ}-\ref{Alg:LoopB}). We also maintain a set $\STB$ with the stabilized  states, i.e. states whose labels are final. The states that are not reachable from the error state are ignored.
  \item A bottom-up breadth-first search is performed from the error state by means of the priority queue $\Q$ (lines \ref{Alg:LoopB}-\ref{Alg:LoopE}), where the state with the lowest label has the highest priority. 
  Let $v'$ be the vertex with the highest priority in the queue. Then, for all predecessor $v$ of $v'$ the following steps 
  are executed (lines \ref{Alg:LoopBPre}-\ref{Alg:LoopEPre}):
  \begin{itemize}
    \item If $v$ is a Verifier's node (resp. a Refuter's node), then we get the 
    maximum (resp. minimum) label $(i,j)$ from all of its successors w.r.t. lexicographical order. In addition, if $v$ is a Verifier's vertex and all its successors are in $\STB$, then $v$ is added to the set. In case that $v$ is a Refuter's vertex, 
    the vertex is simply added without any restrictions. 
    \item Update $(i,j)$ by incrementing $j$ and incrementing $i$ only if $\pr{1}{v} \in \Faults$. Let  $(i',j')$ be this new label.
    \item If $i'$ is different from the first component of the current label of $v$, then update it and add $v$ to the queue. 
    Observe that the same node could be added to the queue
     more than once as its successors' labels change.
  \end{itemize}
  \item When the queue is empty, 
  the algorithm terminates and returns the value of the game (lines \ref{Alg:EndB}-\ref{Alg:EndE}). 
  Intuitively, the procedure halts as soon as the labels of all states in the game graph  reach a final value. 
  Let $k$ be the first component of the initial state's label, then the value of the game is $0$ if 
  $k = \infty$, and $\frac{1}{k}$ otherwise.\\
\end{enumerate}
\begin{algorithm}[t]
\SetAlgoLined
\KwIn{Quantitative Strong Masking Game $\QMStrGame = \langle V^G, V_\Refuter, V_\Verifier,E^G, \InitVertex^G, \reward^G \rangle$}
\KwOut{$\val(\QMStrGame)$}
 Label $\ErrorSt$ with $(1,1)$\; \label{Alg:InitB}
 Label every vertex in $V^G \setminus \{\ErrorSt\}$ with $(\infty,\infty)$\; \label{Alg:InitE}
 %$\Q \gets \emptyset$ \tcp*{$\Q$ is a priority queue}  \label{Alg:CreateQ}
 $\Q \gets \emptyset$ \tcp*{$\Q$ is a priority queue}  \label{Alg:CreateQ}
 $\STB \gets \{\ErrorSt\}$\; \label{Alg:CreateSTB}
 %$\mathcal{VE} \longleftarrow \emptyset$\; \label{Alg:CreateVE} 

 $\enqueue(\Q, \ErrorSt)$\; \label{Alg:LoopB} 
 \While{$\Q$ is not empty}{ \label{Alg:While} 
 	 $s' \gets \dequeue(\Q)$\; \label{Alg:Dequeue-Node}
 	 %\lIf{$\{s'\xrightarrow{\sigma}t \mid t \in post(s')\} \subseteq \mathcal{VE}$}{ \label{Alg:UpdateSTB}
	%	$\mathcal{STB} \longleftarrow \mathcal{STB} \cup \{s'\}$
	 %}
	 
	 \For{$v \in \pre(v') \wedge v \notin \STB$ }{ \label{Alg:LoopBPre}
	 %\For{$s\xrightarrow{\sigma}s' \in E^G$}{ \label{Alg:LoopBPre}
	 	\If{$v \in V_\Verifier$}{
	 		$\lbl \gets \maxlabel(\post(v))$\; \label{Alg:V-Update}
	 		\lIf{$\post(v) \subseteq \STB$}{ \label{Alg:UpdateSTB}
	 		$\STB \gets \STB \cup \{v\}$ 
	 		}
	 		%$enq \longleftarrow post(s) \subseteq \mathcal{STB}$
	 	}
	 	\Else{
	 		$\lbl \gets \minlabel(\post(v))$\; \label{Alg:R-Update}
	 		%$enq \longleftarrow post(s) \cap \mathcal{STB} \neq \emptyset$
	 		$\STB \gets \STB \cup \{v\}$ \label{Alg:R-STB-Update}
	 	}
	 	$\pr{1}{\lbl} \gets \pr{1}{\lbl} + 1$ \tcp*{we assume $\infty+1=\infty$} \label{Alg:Inc_j}
	 	\lIf{$\pr{1}{v} \in \Faults$}{
	 		$\pr{0}{\lbl} \gets \pr{0}{\lbl} + 1$  \label{Alg:Inc_i}
	 	}
	 	\If{$\pr{0}{\Label(v)} \neq \pr{0}{\lbl}$}{ \label{Alg:Insert-Cond}
	 		Label $v$ with $\text{lbl}$\;
	 		$\text{Enqueue}(\Q, v)$  \tcp*{$v$ has priority $\Label(v')$} \label{Alg:Enqueue-Vertex}
	 		%$\mathcal{VE} \longleftarrow \mathcal{VE} \cup \{s\xrightarrow{\sigma}s'\}$
	 	}
	 } \label{Alg:LoopEPre}
 } \label{Alg:LoopE}
 %\lIf{$\pr{0}{Label(s_{0}^G)} \neq \infty$}{ \label{Alg:EndB}
 \lIf{$\InitVertex^G \in \STB$}{ \label{Alg:EndB}
	$\result \gets \frac{1}{\pr{0}{\Label(\InitVertex^G)}}$
 }
 \lElse{
	$\result \gets 0$ \label{Alg:Ret0}
 }
 \Return{$\result$}\; \label{Alg:EndE}
 \caption{Computing the value for a quantitative strong masking game} \label{Alg:MainAlg} 
\end{algorithm}


	 Let us prove the termination and correctness of this algorithm.

\sloppy \begin{thm}\label{th:alg-termination}  \textbf{(Termination)} Algorithm \ref{Alg:MainAlg} terminates.
\end{thm}
\begin{proof}
	Let us prove that a vertex cannot be added an unbounded number of times to $\Q$. 
First, note that lines \ref{Alg:LoopBPre} and \ref{Alg:R-STB-Update} of Algorithm \ref{Alg:MainAlg} imply that Refuter's vertices are 
added only once to the queue. 
Thus, only Verifier's vertices can be added an unbounded number of times to $\Q$. 
We prove by contradiction that this is not the case. 
Assume that $v$ is a Verifier's vertex that is added infinitely often to the queue. 
This implies that $v \notin \STB$ is always true inside the loop, and so 
$\post(v) \cap (V^G \setminus \STB) \neq \emptyset$ (otherwise $v$ would be inserted into $\STB$, line \ref{Alg:UpdateSTB}). 
That is,  some Refuter's node  $w \in \post(v)$ must satisfy $\post(w) \cap \STB = \emptyset$. 
Furthermore, we must also have $\Label(w) = (\infty, \infty)$ because this is the initial value assigned to $w$, 
and the only way of modifying this is by executing line \ref{Alg:R-Update}, 
in such a case, $w$ will be added to $\Q$, contradicting our assumption. 
But the label of $v$ is always calculated taking the maximum of its successors (line \ref{Alg:V-Update}), that is, 
$\Label(v) = (\infty, \infty)$ must hold inside the loop, and since we defined $\infty + 1 = \infty$, the condition in line \ref{Alg:Insert-Cond} is always false.
Thus $v$ is never added to the queue, contradicting our initial assumption. 
Summing up, no node is added an unbounded number of times to the queue. Hence, 
the queue eventually gets empty and the algorithm terminates.
\qedhere
\end{proof} 
\sloppy \begin{thm}\label{th:alg-correctness} \textbf{(Correctness)} On Termination, Algorithm \ref{Alg:MainAlg} 
returns the value of the quantitative strong (resp. weak) masking game graph $\QMStrGame$.
\end{thm}
\begin{proof}
	Let us consider the function  $\delta(v) = \min \{(i,j) \mid v \in \setsUs\}$, where $\min \emptyset = (\infty, \infty)$.
We also consider the extension of this function to sets, that is, $\delta(W) = \{\delta(w) \mid w \in W \}$. 
Furthermore, the following properties of $\delta$ are a consequence of Def. \ref{def:U}: \\

\noindent
if $v$ is a Refuter's vertex:
\begin{equation}\label{prop:delta-1}
 \delta(v)  = \min \{(i,j+1) \mid (i,j) \in \delta(\post(v)) \}
\end{equation}
If $v$ is a Verifier's vertex and $\pr{1}{v} \in \Faults$ then:
\begin{equation}\label{prop:delta-2}
 \delta(v)  = \max \{(i+1,j+1) \mid (i,j) \in \delta(\post(v)) \}
\end{equation}
If $v$ is a Verifier's vertex and $\pr{1}{v} \notin \Faults$ then:
\begin{equation}\label{prop:delta-3}
 \delta(v) = \max \{(i,j+1) \mid (i,j) \in \delta(\post(v)) \}
\end{equation}
	Now, let us prove that the following predicates always hold after line \ref{Alg:While}.
\begin{equation}\label{Alg:inv0}
	\forall v \in \STB : \Label(v) < (\infty, \infty)
\end{equation}
\begin{equation}\label{Alg:inv1}
	\forall w, v \in S^G : v \in \STB \wedge \Label(w) < \Label(v) \Rightarrow w \in \STB
\end{equation}
%and 
\begin{equation}\label{Alg:inv2}
	\forall v \in \STB : \Label(v) = \delta(v)
\end{equation}

	Proof of Property \ref{Alg:inv0}: it holds for the initialization (line \ref{Alg:CreateSTB}). We proceed by contradiction. 
Let $v$ be the first node added to $\STB$ such that $\Label(v) = (\infty, \infty)$.
If it is a Refuter's vertex, then $\post(v) \cap \STB \neq \emptyset$ and since $v$ was the first node added to $\STB$ with 
$\Label(v) = (\infty, \infty)$ we have that $\forall w \in \post(v) \cap \STB: \Label(w) < (\infty, \infty)$. But then, in line \ref{Alg:R-Update},
$v$ is labelled with some label different from $(\infty, \infty)$ which is a contradiction. 
Similarly, if $v$ is a Verifier's vertex, then $\post(v) \subseteq \STB$ and 
therefore it is labelled in line \ref{Alg:V-Update} with a value different from $(\infty, \infty)$, contradicting our initial assumption. 
Hence, $\forall v \in \STB : \Label(v) < (\infty, \infty)$.

	Proof of Property \ref{Alg:inv1}: $\STB$ is initialized with $\{\ErrorSt \}$ and therefore the property holds before line \ref{Alg:While}. 
For the sake of contradiction, assume that $v$ is the first node added to $\STB$ that does not hold the property, that is, there is a $w$ such that $\Label(w) < \Label(v)$ and
$w \notin \STB$; furthermore, let us assume that $w$ is a vertex with minimum label satisfying this. 
Since $v \in \STB$ we have $\Label(v)<(\infty, \infty)$ (by Prop.~\ref{Alg:inv0}) and therefore $\Label(w)< (\infty, \infty)$. 
In fact, $w$ cannot be a Refuter's vertex, otherwise, when labelled, it would be added to $\STB$ (line \ref{Alg:R-STB-Update}), 
contradicting our assumptions. 
Now, if $w$ is a Verifier's vertex and $\Label(w) < (\infty, \infty)$
since $\Label(w) = \max\{\Label(z) \mid z \in \post(w)\}$ (line \ref{Alg:R-Update}), we have: $\forall z \in \post(w) : \Label(z) < (\infty, \infty)$.
Moreover, we have by transitivity that $\forall z \in \post(w) : \Label(z) < \Label(v)$. But since we assumed that 
$w$ was the vertex with minimum label satisfying $\Label(w) < \Label(v)$ and $w \notin \STB$, we have 
that $\forall z \in \post(v) : z \in \STB$. But then when $v$ was inspected in line \ref{Alg:UpdateSTB} it was added to $\STB$, 
which yields a contradiction.

	Proof of Property \ref{Alg:inv2}: the property holds for $\ErrorSt$. Let $v$ be the first vertex added to $\STB$ such 
	that $\Label(v) \neq \delta(v)$. In case that $v$ is a Verifier's vertex, then, 
	since $v \in \STB$, we have $\post(v) \subseteq \STB$. Moreover,  we assumed that $v$ was the first vertex violating the property,  
	so we have that $\forall w \in \post(v):\delta(w)=\Label(w)$. Furthermore, if $\pr{1}{v} \notin \Faults$, then: 
\begin{align*}
\Label(v) & = \max \{(i,j+1) \mid (i,j) \in \Label(\post(v))\}&  \text{(Lines \ref{Alg:V-Update} and \ref{Alg:Inc_j})}\\
	      & =  \max \{(i,j+1) \mid (i,j) \in \delta(\post(v))\}&  \text{(Assumption)}\\
	      & = \delta(v)& \text{(Prop. \ref{prop:delta-3})}
\end{align*}
which is a contradiction. Now, in case that $\pr{1}{v} \in \Faults$ we have that:
\begin{align*}
\Label(v) & = \max \{(i+1,j+1) \mid (i,j) \in \Label(\post(v))\}&  \text{(Lines \ref{Alg:V-Update}, \ref{Alg:Inc_j}, and \ref{Alg:Inc_i})}\\
	      & =  \max \{(i+1,j+1) \mid (i,j) \in \delta(\post(v))\}&  \text{(Assumption)}\\
	      & = \delta(v)& \text{(Prop. \ref{prop:delta-2})}
\end{align*}
which also contradicts our assumption. Therefore, $v$ cannot be a Verifier's vertex. 
In case that $v$ is a Refuter's vertex, since $v \in \STB$ we have that $\Label(v) < (\infty, \infty)$ by Prop.~\ref{Alg:inv0}. 
Thus, when $v$ was added to $\STB$, $\Label(v) = \min\{ \Label(w) \mid w \in \post(v)\}$. Therefore, 
there exists a $w \in \post(v)$ such that $\Label(w) \leq \Label(v)$, but then by Prop.~\ref{Alg:inv1} 
we have $w \in \STB$. Thereby:
\begin{align*}
\Label(v) & = \min \{(i,j+1) \mid (i,j) \in \Label(\post(v))\}&  \text{(Lines \ref{Alg:R-Update} and \ref{Alg:Inc_j})} \\
	      & =  \min \{(i,j+1) \mid (i,j) \in \delta(\post(v))\}&  \text{(Assumption)}\\
	      & = \delta(v)& \text{(Prop. \ref{prop:delta-2})}
\end{align*}
contradicting our assumptions. Hence, $\Label(v) = \delta(v)$.

	Now, we prove that when the loop of line \ref{Alg:LoopBPre} terminates we have:
\begin{equation}\label{Alg:inv3}
 \forall v \in S^G \setminus \STB : \delta(v)=(\infty, \infty)
\end{equation}

We proceed by contradiction. Let $v \in S^G \setminus \STB$ such that $\delta(v) < (\infty, \infty)$ and, furthermore, assume 
%$\delta(v) = \min \{w \in S^G \setminus \STB \mid \delta(w) < (\infty, \infty) \}$.
$\delta(v)$ is the minimum label satisfying this.
If $v$ is a Refuter's vertex, then there exists some $w \in \post(v)$ such that $\delta(w) < \delta(v)$. 
Thus, we have that $w \in \STB$ (by our assumption) which means that $v$ was added at least once in the queue since it 
was initialized with $(\infty, \infty)$. 
Thereby, it was added to $\STB$ in line \ref{Alg:R-STB-Update}, contradicting the initial assumption. 
Therefore, $v$ must be a Verifier's vertex. 
If $\pr{1}{v} \in \Faults$, then $\delta(v) = \max \{(i+1,j+1) \mid (i,j) \in \delta(\post(v))\}$ 
and therefore $\delta(v) > \delta(w)$ for all $w \in \post(w)$.
Thus, by our assumptions, $w \in \STB$ for every $w \in \post(v)$.
Moreover, when the last $w \in \post(v)$ was added to $\STB$, $v$ was in the queue because of the BFS policy.
In fact, when condition $\post(v) \subseteq \STB$ was checked for $v$ (line \ref{Alg:UpdateSTB}), it was true and then $v$ 
should be added to $\STB$, contradicting our assumption. Hence, $v \in \STB$.

Finally, the result computed by Algorithm \ref{Alg:MainAlg} follows from properties \ref{Alg:inv2}, \ref{Alg:inv3}, 
and Theorem \ref{thm:quant_game}. 
In more details, if $\delta(s_{0}^G) = (\infty, \infty)$, then by Prop. ~\ref{Alg:inv0} and \ref{Alg:inv3}, we 
have upon termination that $\Label(s_{0}^G) = (\infty, \infty)$. 
Thereby, the algorithm returns $0$ in line \ref{Alg:Ret0} which is correct by Theorem~\ref{thm:quant_game}. 
In case that $\delta(s_0^G)< (\infty, \infty)$, then $s_0^G \in \STB$ and also by Prop.~\ref{Alg:inv2} we 
have $\Label(v) = \delta(v)$. Therefore, the algorithm returns the correct answer in line \ref{Alg:EndB} by Theorem \ref{thm:quant_game}. 
\qedhere
\end{proof}




It remains to discuss the running time complexity for computing the value of quantitative masking games. 
The following theorem states the time complexity of determining the value of any kind of quantitative masking games.
%
%\noindent
%Finally, winning regions (and strategies) can be computed in lineal time.  
\sloppy \begin{thm}\label{th:qgame-determined} Any quantitative strong (resp. weak) masking game graph 
$\QMStrGame = \langle V^G, V_\Refuter, V_\Verifier, E^G, \InitVertex^G, \reward^G \rangle$   can  be determined in 
time  $\BigO(|E^G|*\log |V^G|)$ (resp. $\BigO(|E_W^G|*\log |V^G|)$).
\end{thm}
\begin{proof}
	Note that the loop in line \ref{Alg:LoopBPre} inspects the edges of the graph. Furthermore, note that, no edge $(v, v')$ can be inspected twice:
	if $v'$ is a Refuter's vertex, then it will be added to $\STB$ and the edge cannot be processed any longer. 
	If $v$ is Verifier's vertex, then $v'$ is a Refuter's
	vertex, thus once dequeued (line \ref{Alg:Dequeue-Node}), it will not be added to the queue again, and therefore the edge  $(v, v')$ will not
	processed again. Thereby,  the loop of line \ref{Alg:LoopBPre} will be executed $|E^G|$ times in the worst case. 
	Furthermore,  if some efficient implementation of priority queues is used for storing the states (line \ref{Alg:Enqueue-Vertex}), 
	then it takes $\BigO(\log |V^G|)$ steps to insert or remove an element from the queue, that is, in total the loop of line \ref{Alg:LoopBPre} takes $\BigO(|E^G|*\log |V^G|)$ steps.
	In addition, note that the \emph{while loop} of line \ref{Alg:While} is executed until the $\Q$ gets empty, and nodes are enqueued only in line \ref{Alg:Enqueue-Vertex}. Thus, the number of times that the \emph{while loop} is executed is 
	bounded by the number of times that the \emph{for loop} 	is executed, which implies that
	the outmost loop is executed at most $O(|E^G|)$ times. Similar remarks hold for the quantitative weak masking game.	

\qedhere
\end{proof} \\
% TBD similar for weak games

Theorems \ref{th:game-determined} and \ref{th:qgame-determined} describe the complexity of solving the quantitative and standard masking games. However, in practice, one needs to bear in mind that $|V^G| = |S|*|S'|$ and $|E^G| = |{\rightarrow}|+|{\rightarrow'}|$, so constructing the game takes 
$\BigO(|S|^2*|S'|^2)$ steps in the worst case. Additionally, for weak games, the transitive closure of the original model needs to be computed, which for the best known algorithm yields $\BigO(\max(|S|,|S'|)^{2.3727})$ \cite{Wil12}.

	Interestingly, deterministic games can be solved in linear time by using the fact proven in Theorem \ref{theorem:det-games}.
\begin{thm}\label{th:deterministic-qgame-complexity} 
  Any quantitative strong (resp. weak) deterministic masking game can be solved in 
time $\BigO(|E^G|)$ (resp. $\BigO(|E_W^G|)$).
\end{thm}
\begin{proof} By Theorem \ref{theorem:det-games}, the value of the game is given by the shortest path to the error state in the game graph where the weight function assigns  $1$ to faults, and $0$ to other transitions. Dial's algorithm \cite{Dial69} can be used to obtain the shortest path in this graph, which runs in $\BigO(|E^G|)$  (resp. $\BigO(|E^G_W|)$ in the case of weak games).
\qedhere
\end{proof} 

Let us note that using the sets $\setsUs$ we can define optimal strategies for the 
Refuter and the Verifier, without taking into account the history of the play. 
That is, we have the following theorem.
 
\begin{thm} \label{thm:memoryless} Let $\QMStrGame$ be a quantitative strong masking game graph.
  Players $\Refuter$ and $\Verifier$ have memoryless optimal strategies for $\QMStrGame$.
\end{thm}
\begin{proof} Giving a game we can compute the sets $\setsUs$ using Algorithm~\ref{Alg:MainAlg}.
Thereby, it is straightforward to define an optimal strategy for the Refuter. 
For a given node $v$, if it belongs to some $\setsUs$, then the Refuter chooses some node in $U^{j'}_{i'}$ where: 
$(i', j') = \text{max}\{(i'', j'') \mid post(v) \cap U^{j}_{i} \neq \emptyset \wedge (i'',j'') < (i,j) \}$. 
By Definition of $\setsUs$, we know that such a pair exists, and also that this 
strategy is winning for the Refuter. If $v \notin \setsUs$, then any choice by the Refuter will lead to a winning play of the Verifier. 
Hence, the Refuter moves to any successor of $v$.
In a similar way, we can define an optimal strategy for the Verifier.
\qedhere 
\end{proof} \\

%% \noindent
%% Now, we present some basic properties of the masking distance.

By using $\QMWeakGame$ instead of $\QMStrGame$ in
Def.~\ref{def:mask_dist}, we can define the \emph{weak masking
  distance} $\DeltaMask^W$.  The next theorem states that $A$ and
$A'$ are at distance $0$ if and only if there exists a strong (or weak) masking
simulation between them.

\begin{thm}\label{theorem:ref}
  For any transition systems $A = \langle S, \Sigma, \rightarrow, \InitState \rangle$ and $A' = \langle S', \Sigma_{\mathcal{F'}}, \rightarrow', \InitStatePrime \rangle$, it holds that:
  \begin{enumerate}[(i)]
  \item  $\DeltaMask(A,A') = 0$ iff $A \Masking A'$, and
   \item $\DeltaMask^W(A,A') = 0$ iff $A \WeakMasking A' $.
  \end{enumerate}
\end{thm}
%
This follows from Theorem \ref{thm:quant_game}.
%That is, the masking distance between two systems is $0$ if and only if there is a masking simulation between them.
Noting that $A \Masking A$ (and $A \WeakMasking A$) for 
any transition system $A$, we obtain that $\DeltaMask(A,A)=0$ (resp.\ $\DeltaMask^W(A,A)=0$) by Theorem \ref{theorem:ref}, i.e., 
both distances are reflexive.

For our running example, the masking distance is $1/3$ 
with a redundancy of $3$ bits and considering two faults. 
This means that only one fault can be masked by this implementation. \\

We can prove a version of the triangle inequality for our notion of distance. 
%
\begin{thm} \label{thm:triang_ineq}
  Let $A = \langle S, \Sigma, \rightarrow, s_0 \rangle$, $A' = \langle S', \Sigma_{\mathcal{F'}}, \rightarrow', s'_0 \rangle$,  and  $A'' = \langle S'', \Sigma_{\mathcal{F''}},\rightarrow'', s''_0 \rangle$ be transition systems such that $\Faults' \subseteq \Faults''$, and the corresponding games $\mathcal{Q}_{A,A'}$, $\mathcal{Q}_{A',A''}$ and $\mathcal{Q}_{A,A''}$.
  Then $\DeltaMask(A,A'') \leq \DeltaMask(A,A') + \DeltaMask(A', A'')$ and
  $\DeltaMask^W(A,A'') \leq \DeltaMask^W(A,A') + \DeltaMask^W(A', A'').$
\end{thm}
\begin{proof} 
We will prove that for every node $(s,\#,s'',\Refuter)$ in $\mathcal{Q}_{A,A''}$ and  pair of nodes
$(s,\#, s', R)$ in $\QMStrGame$ and $(s',\#,s'', R)$ in $\mathcal{Q}_{{A'},A''}$ (for any state $s'$ in $A'$), it holds that:
\[
\frac{1}{\pr{0}{\delta((s,\#, s'', \Refuter))}} \leq \frac{1}{\pr{0}{(\delta(s,\#, s', R))}} + \frac{1}{\pr{0}{\delta((s',\#, s'', R))}}
\] 
where $\delta(v) = \min \{(i,j) \mid v \in U^i_j\}$, calculated in the corresponding game. 
As in Theorem \ref{thm:quant_game}, we just define $\max \emptyset = (\infty, \infty)$. 
The result follows from this fact and Theorem \ref{thm:quant_game}. The proof is by induction on $\delta((s,\#, s'', \Refuter))$.

First, let us observe that for any node $(s, \#, s'', \Refuter)$ of game $\mathcal{Q}_{A,A''}$ we must have $\delta((s, \#, s'', \Refuter)) \geq (1,3)$.
In fact, this node cannot be the error state, which means that $j \neq 1$. Moreover, after the movement of the Refuter we have at least one movement from the Verifier and thus $j \geq 3$. That is the base case is $\delta((s,\#, s'', \Refuter)) = (1,3)$.
\begin{description}
\item [Base Case] If $\delta((s,\#, s'', \Refuter)) = (1,3)$. Assume that $(s, \#, s'', R) \in U^3_1$. This means that we have a transition 
$((s, \#, s'', \Refuter), (w, \sigma^t, w'', \Verifier))$, where $t \in \{1,2\}$, that cannot be matched by the Verifier. 
In case that $t=1$, then this move is a transition $((s, \#, s'', \Refuter), (w, \sigma^1, s'', \Verifier))$ from $A$. 
Now, let $(s, \#, s', \Refuter)$ and $(s', \#, s'', \Refuter)$ be a pair of states of $\QMStrGame$ and $\mathcal{Q}_{A',A''}$, respectively. 
By definition, we have a transition $((s, \#, s', \Refuter), (w, \sigma^1, s', \Verifier))$ in $Q_{A,A'}$. 
In case that the Verifier cannot match this move in that game, we have that $(s, \#, s', \Refuter) \in U^3_1$. 
This finishes the proof since $1 \leq 1 + k''$, regardless of the value of $k''$. 
Otherwise, the Verifier chooses a matching move, that is, $((w, \sigma^1, s', \Verifier), (w, \#, w', \Refuter))$ in $\mathcal{Q}_{A,A'}$. 
Moreover, we have a transition $((s', \#, s'', \Refuter), (w', \sigma^1, s'', \Verifier))$ in $\mathcal{Q}_{A',A''}$. 
But, this cannot be matched due to our initial assumption. Therefore, $\delta((s', \#, s'', \Refuter))= (1,3)$, and we have that:
\begin{align*}
	\frac{1}{\pr{0}{\delta((s,\#, s'', \Refuter))}}  & = 1\\
							& \leq \frac{1}{\pr{0}{\delta((s, \#, s', \Refuter))}} + 1\\
							& =   \frac{1}{\pr{0}{\delta((s, \#, s', \Refuter))}}  + \frac{1}{\pr{0}{\delta((s', \#, s'', \Refuter))}}
\end{align*}
For $t=2$,  the reasoning is similar using the transitions of $A''$. 

\item [Induction Step] For $(i,j) > (1,3)$ the proof is as follows. 
Assume that $\delta((s, \#, s'', \Refuter)) = (i,j)$. Because of $1<i \leq j$, we have a transition 
$((s, \#, s'', \Refuter), (w, \sigma^t, w'', \Verifier))$ in $\mathcal{Q}_{A,A''}$. We proceed by cases.

  Case $\sigma^t = F^2$ for some $F \in \Faults''$. Since $\delta((s, \#, s'', \Refuter)) > (1,3)$,
we must have a transition $((s, F^2, w'', \Verifier), (s, \#, w'', \Refuter)) \in Q_{A,A''}$ and $\delta((s, \#, w'', \Refuter)) = (i-1,j-2)$. 
Thus, by definition of $\mathcal{Q}_{A',A''}$, we have a transition 
$((s', \#, s'', \Refuter), (s', F^2, w'', \Verifier))$ in $\mathcal{Q}_{A',A''}$. 
In case that $F \in \Faults'$, then $F \in \Sigma_{\Faults'}$. 
If it cannot be matched, then:
\begin{align*}
\frac{1}{\pr{0}{\delta((s,\#, s'', \Refuter))}}  & \leq 1	\\
							     &= \frac{1}{\pr{0}{\delta((s', \#, s'', \Refuter))}}  \\
							     &\leq  \frac{1}{\pr{0}{\delta((s, \#, s', \Refuter))}} +  \frac{1}{\pr{0}{\delta((s', \#, s'', \Refuter))}}   
\end{align*}
and the result follows.  Otherwise, we have a collection of transitions  $((s', F^2, w'', \Verifier), (w', \#, w'', \Refuter))$ in $\mathcal{Q}_{A',A''}$. 
So, in game $\QMStrGame$, we have at least an edge $((s, \#, s', \Refuter), (s, F^2, w', \Verifier))$. 
Then, since $((s, F^2, w'', \Verifier), (s, \#, w'', \Refuter)) \in Q_{A,A''}$, we also have  a transition $((s, F^2, w', \Verifier), (s, \#, w', \Refuter))$ in $Q_{A,A'}$. 
By induction hypothesis, we have $\delta((s, \#, w', \Refuter))= i'$ and $\max \{ \delta(v) \mid v \in \post((s', F^2, w'', \Verifier)) \}=i''$ such that:
\begin{equation}\label{eq:hi1}
\frac{1}{i-1} \leq \frac{1}{i'} + \frac{1}{i''}.
\end{equation} 
and thus:
\begin{equation}\label{eq:hi1a}
\frac{1}{i} \leq \frac{1}{i'+1} + \frac{1}{i''}.
\end{equation}
Furthermore, note that $\delta((s, F^2, w', \Verifier)) \geq (i'+1,j'+1)$ and we have a unique (masking) transition from $(s, F^2, w', \Verifier)$. 
Thereby, 
\begin{equation}\label{eq:hi2}
	\delta((s, \#, s', \Refuter)) \geq (i'+1,j'+2)
\end{equation}
Similarly, since $F \in \Sigma_{\Faults'}$ we have that:
\begin{equation}\label{eq:hi3}
	\delta((s', \#, s'', \Refuter)) \geq (i'',j'+2)
\end{equation}
Hence, taking into account inequations \ref{eq:hi1a}, \ref{eq:hi2} and \ref{eq:hi3}, we 
obtain:
\begin{align*}
	\frac{1}{\pr{0}{\delta((s,\#, s'', \Refuter))}} & = \frac{1}{i} & \text{(Assumption)} \\
									      & \leq \frac{1}{i'+1} + \frac{1}{i''} & \text{Ineq.~\ref{eq:hi1a}} \\
									      &  = \frac{1}{\pr{0}{\delta((s, \#, s', \Refuter))}} +  \frac{1}{\pr{0}{\delta((s', \#, s'', \Refuter))}} &
\end{align*}
	If $F \notin \Faults'$, we must also have a transition $((s', F^2, w'', \Verifier), (s', \#, w'', \Refuter))$ in $\mathcal{Q}_{A',A''}$. 
Now, by induction hypothesis, we have $\delta((s, \#, s', \Refuter)) = (i',j')$ and $\delta((s', \#, w'', \Refuter)) = (i'',j'')$ such that 
$\frac{1}{i-1} \leq \frac{1}{i'} + \frac{1}{i''}$. Thereby, $\delta((s', \#, s'', \Refuter)) = (i''+1, j''+2)$ and similarly as above the result follows.

	Case $\sigma^t \neq F^2$. If $t=1$, then we have a transition $((s, \#, s'', R), (w, \sigma^1, s'', V))$ in $\mathcal{Q}_{A,A''}$, and 
	since $\delta((s, \#, s'', \Refuter)) > (1,3)$,
we must have a transition $((s, \sigma^1, w'', \Verifier), (w, \#, w'', \Refuter))$ in $Q_{A,A''}$ such that $\delta((s, \#, w'', \Refuter)) = (i,j-2)$. 
By definition of game $Q_{A, A'}$, we have a transition
$((s, \#, s', \Refuter), (w, \sigma^1, s', \Verifier))$. In case that it cannot be matched, 
then it leads to the desired result. Otherwise, let us define $(i',j') = \max \{\delta(v) \mid v \in \post((w, \sigma^1, s', \Verifier)) \}$. 
Since $\post((w, \sigma^1, s', \Verifier)) \neq \{\ErrorSt\}$, there must be also a transition $((s', \#, s'', R), (w', \sigma^1, s'', V))$ in $Q_{A',A''}$. 
In case that this cannot be matched, then we have $\delta((s', \#, s'', R)) = (1,3)$ and the proof finishes. 
In other case, let us define $i'' = \max \{\delta(v) \mid v \in \post((w', \sigma^1, s'', V)) \}$ where by induction hypothesis we have
\begin{equation}
	\frac{1}{i} \leq \frac{1}{i'} + \frac{1}{i''}
\end{equation}	
	Hence, 
\begin{equation}
	\frac{1}{\pr{0}{\delta((s, \#, s'', R))}} \leq \frac{1}{\pr{0}{\delta((s, \#, s', R))}} + \frac{1}{\pr{0}{\delta((s', \#, s'', R))}}
\end{equation}	
	For the case $t=2$ the proof is similar.
\end{description}


\qedhere
\end{proof}\\

The proof for $\DeltaMask^W$ is similar to the $\DeltaMask$ 
but using $\QMWeakGame$ instead of $\QMStrGame$ and by Theorem~\ref{thm:weak_thm}.

Reflexivity and  triangle inequality imply that both masking distances are directed semi-metrics \cite{CharikarMM06,AlfaroMRS08}.  Moreover, it is interesting to note that the triangle inequality property has practical applications. When developing critical software is quite common to develop a first version of the software taking into account some possible anticipated faults. 
Later, after testing and running of the system, more plausible faults could be observed. 
Consequently, the system is modified with additional fault-tolerant capabilities to be able 
to overcome them. 
Theorem \ref{thm:triang_ineq} states that incrementally measuring the masking distance between these different versions of the software provides an upper bound to the actual distance between the nominal system and its last fault-tolerant version. That is, if the sum of the distances obtained between the different versions is a small number, then we can ensure that the final system will 
exhibit an acceptable masking tolerance to faults w.r.t. the nominal system.





\section{Related Work} \label{sec:related_work}

This article is a revised and expanded version of a conference paper presented at 
TACAS 2019 \cite{CastroDDP18b} and it extends it as follows:  
(1) it presents formal proofs for all theorems; (2) it describes the problem of computing  
the masking distance considering deterministic and 
non-deterministic systems; (3) it includes a more efficient implementation 
of the \MaskD~tool; and (4) it proposes a more thorough experimental analysis, including more 
instances for each case study and a new case study (Raft).

In recent years, there has been a growing interest in the quantitative 
generalizations of the boolean notion of correctness and the 
corresponding quantitative verification questions \cite{BokerCHK14,CernyHR12,Henzinger10,Henzinger13}.
The framework described in \cite{CernyHR12} is the closest related work to our approach. 
The authors generalize the traditional notion of 
simulation relation to three different versions of simulation distance: \emph{correctness}, \emph{coverage}, and \emph{robustness}.
These are defined using quantitative games with \emph{discounted-sum} 
and \emph{mean-payoff} objectives, two well-known cost functions.
Similarly to that work, we also consider distances between purely 
discrete (non-probabilistic, untimed) systems.

Correctness and coverage distances are concerned with the nominal part of the systems, 
and so faults play no role on them. On the other hand, robustness distance measures how many unexpected errors can be performed by the implementation in such a way that the resulting behavior is tolerated by the specification. So, it can be used to analyze the resilience of the implementation. Note that, robustness
distance can only be applied to correct implementations, that is, implementations that preserve the behavior of the specification but perhaps do not cover all its behavior. As noted in~\cite{CernyHR12}, bisimilarity sometimes implies a distance of $1$. In this sense a greater grade of robustness (as defined in~\cite{CernyHR12}) is achieved by pruning critical points from the specification. Furthermore, the  errors considered in that work are transitions mimicking the original ones but with different labels. In contrast to this, 
 in our approach we consider that faults are injected into the fault-tolerant 
 implementation, where their behaviors are not restricted by the nominal system. 
 This follows the idea of model extension in fault-tolerance where faulty behavior is added 
 to the nominal system. Further, note that when no faults are present, the masking distance between the specification and the implementation is $0$ when they are bisimilar, and it is $1$ otherwise.
It is useful to note that robustness distance of~\cite{CernyHR12} is not reflexive. We believe that all these definitions of distance between systems capture different notions useful for software development, and they can be used together, in a complementary way, to obtain an in-depth
evaluation of fault-tolerant implementations.





